\documentclass[conference]{IEEEtran}
\ifCLASSINFOpdf
  % \usepackage[pdftex]{graphicx}
  % declare the path(s) where your graphic files are
  % \graphicspath{{../pdf/}{../jpeg/}}
  % and their extensions so you won't have to specify these with
  % every instance of \includegraphics
  % \DeclareGraphicsExtensions{.pdf,.jpeg,.png}
\else
  % or other class option (dvipsone, dvipdf, if not using dvips). graphicx
  % will default to the driver specified in the system graphics.cfg if no
  % driver is specified.
  % \usepackage[dvips]{graphicx}
  % declare the path(s) where your graphic files are
  % \graphicspath{{../eps/}}
  % and their extensions so you won't have to specify these with
  % every instance of \includegraphics
  % \DeclareGraphicsExtensions{.eps}
\fi

\usepackage{booktabs} % For formal tables
\usepackage{multirow}
\usepackage{algorithm}
\usepackage[noend]{algpseudocode}
\usepackage[pdftex]{graphicx}
\usepackage[T1,hyphens]{url}
%\usepackage[colorlinks,urlcolor=blue]{hyperref}
\usepackage[]{hyperref}

\graphicspath{{./figures/}}
\algrenewcommand\textproc{}

%\algsetup{linenosize=\small}

% correct bad hyphenation here
\hyphenation{op-tical net-works semi-conduc-tor}

\begin{document}
%
% paper title
% Titles are generally capitalized except for words such as a, an, and, as,
% at, but, by, for, in, nor, of, on, or, the, to and up, which are usually
% not capitalized unless they are the first or last word of the title.
% Linebreaks \\ can be used within to get better formatting as desired.
% Do not put math or special symbols in the title.
\title{OBFS: OpenCL Based BFS Optimization on Software Programmable FPGAs}
% Ironing out the irregularity in BFS for efficient OpenCL implementation on Xeon-FPGA

% make the title area
\maketitle

% As a general rule, do not put math, special symbols or citations
% in the abstract
\begin{abstract}
    Breadth First Search (BFS) is a key building block of graph processing 
	and there have been considerable efforts devoted to accelerating BFS on FPGAs
	for the sake of both performance and energy efficiency. Prior work 
	typically built the BFS accelerator through handcrafted circuit design using 
	hardware description language (HDL). Despite the relatively good performance, 
	the HDL based design leads to extremely low design productivity, and incurs 
	high portability and maintenance cost. While the evolving high level synthesis (HLS) 
	tools make it convenient to create a functional correct BFS accelerator, 
	the performance of the baseline design remains much lower. 

	To obtain both the near hand-crafted design performance and the software-like features, 
	we propose OBFS, an OpenCL based BFS accelerator on software programmable FPGAs, 
	and explore a series of high-level optimizations to the OpenCL design. 
	With the observation that OpenCL based FPGA design is rather inefficient on 
	irregular memory access (random and short burst), we focus on the optimization 
	of irregular memory access in BFS. 
	First of all, we convert the low-efficient irregular edge reading into batched 
	memory access with graph reordering. Then we use an on-chip bitmap to avoid random 
	visiting status access over DDR. On top of the graph reordering and 
	on-chip bitmap, we further build conflict-free parallel data paths 
	to make best use of the on-chip memory bandwidth. In addition, we 
	shift the random BFS level update from the main BFS processing and 
	hide it with overlapped execution of different BFS processing. 
	According to the experiments on a set of representative graphs, 
	OBFS achieves up to 12X performance speedup compared to the reference design 
	in Spector benchmark on Intel Harp-v2. When compared to prior handcrafted design on 
	similar FPGA cards, it achieves comparable performance or even better on some R-MAT graphs. 
\end{abstract}

% For peer review papers, you can put extra information on the cover
% page as needed:
% \ifCLASSOPTIONpeerreview
% \begin{center} \bfseries EDICS Category: 3-BBND \end{center}
% \fi
%
% For peerreview papers, this IEEEtran command inserts a page break and
% creates the second title. It will be ignored for other modes.
\IEEEpeerreviewmaketitle

\section{Introduction} \label{sec:intro}
Breadth-first search (BFS) is the basic building component of many graph algorithms 
and is thus of vital importance to high-performance graph processing. Nevertheless, 
it is notoriously difficult for accelerating on FPGAs because of the 
irregular memory access and the low computation-to-memory ratio. 
At the same time, BFS on large graphs also involves tremendous 
parallelisms which indicate great potential for acceleration. 
With both the challenge and the parallelization potential, 
BFS has attracted a number of researchers exploring its acceleration on FPGAs 
\cite{attia2014cygraph, betkaoui2012reconfigurable, Dai2017foregraph, Ma2017fpga,
umuroglu2015hybrid, oguntebi2016graphops, engelhardt2016gravf, zhou2016high}. 

Previous work have shown that BFS accelerators on FPGAs can provide competitive  
performance and superior energy efficiency when given comparable memory bandwidth. 
However, these work typically optimize BFS or relatively general graph processing 
with dedicated circuit design using hardware description language (HDL). The HDL 
based designs with customized circuits are beneficial to the resulting performance 
and save resource consumption, but it usually takes long time for development, 
upgrade, maintenance and porting to a different FPGA device, which are all 
important concerns from the perspective of the accelerator developers. Another 
engineering yet non-trivial problem is the high barrier to use the FPGA powered graph 
processing accelerators in high-level applications such as 
big data analytics, which is mostly caused by the lack of 
well-defined high level interface and user-friendly SDK supporting 
various hardware systems. Improving the ease of using the 
HDL based accelerators requires a lot of design efforts 
such as driver and runtime environment to support newer 
devices and diverse computing systems. This is also one of 
the key obstacles hindering the widespread adoption of 
the FPGA accelerators despite the great performance-energy 
efficiency advantages.

The limitation of the conventional HDL design method in combination with the 
rapid advancements of the HLS techniques makes the HLS tools attractive. 
HLS tools are increasingly adopted in both industry and academia for rapid FPGA prototyping and 
application acceleration. Software programmable FPGAs \cite{koch2016fpgas, xilinx-sdaccel} 
gets widespread acceptance. Nevertheless, the current HLS based design tools are mostly used for 
applications with relatively regular memory access patterns and data paths. 
It remains challenging for the HLS tools to accelerate BFS with irregular 
memory access patterns and complex data paths. In general, the main reasons lies in the 
following aspects. First of all, the HLS tools nowadays can support only very limited 
on-chip buffer optimizations, so it is rather difficult to handle irregular memory accesses 
especially random memory accesses. Secondly, hardware pipelining strategy in HLS tools 
is usually conservative to ensure the functional correctness, while 
this also leads to inefficient hardware implementations when the data path is data 
dependent or dynamic. Under such a context, we explore the use of Intel OpenCL for 
efficient BFS acceleration on software programmable FPGAs.

To cope with the irregular memory accesses and dynamic data paths in BFS, we proposed 
a series of optimization methods to regularize 
both the data path and memory accesses for efficient HLS implementation. 
We start with graph edge reordering. Basically, the edges 
are shuffled based on its destination vertices and divided into batches. In each batch, 
the edges point to different segments of the vertices. Then we have the vertex visiting 
status stored into multiple on-chip buffer banks while each bank stores the visiting status 
of vertices in different vertex segments. In combination with the edge batching, we can read 
and process the edges in the granularity of batches without any stall. In addition, we also have 
the coupled CPU to gather the scattered frontier vertices' edge location and combine them as 
as an sequential array such that the data path on the FPGA can be pipelined smoothly. 
According to the experiments on a set of big graphs, the optimized high level BFS 
accelerator achieves up to 70X performance speedup when compared to the design 
in Spector benchmark. It achieves around 80\% of the handcrafted design on 
similar FPGA boards. 

The major contributions of this work are summarized as follows.
\begin{itemize}
    \item As far as we know, this is the first highly optimized and open-sourced 
		HLS based BFS accelerator on FPGAs targeting portability and ease of 
		use on top of performance. 
    \item We proposed a set of combined methods to regularize the irregular
		memory accesses and dynamic data paths of BFS. This may shed light on similar 
		irregular application acceleration on FPGAs using HLS tools.
    \item The resulting accelerator shows significant performance speedup 
        over a baseline HLS design and gets close to state-of-art handcrafted 
		design on a set of representative graphs.
\end{itemize}

The rest part of the paper is organized as follows. In Section \ref{sec:relatedwork}, 
we brief the background of software programmable FPGAs and related work of 
BFS acceleration especially on FPGAs. In Section \ref{sec:motivation},  
we analyze the performance of basic BFS implementations with best-effort HLS 
optimizations and demonstrate the challenge of BFS acceleration using OpenCL. 
In Section \ref{sec:bfs-opt}, we present the overview of the BFS accelerator 
design using OpenCL and detail the major optimization methods.
In Section \ref{sec:experiment}, we present comprehensive experiments of the 
BFS accelerator. Finally, we conclude this work in Section \ref{sec:conclusion}.



\section{Background}\label{sec:background}
Hardware faults in the DNN accelerators are the major sources of 
the unreliability. The influence of faults is closely related with 
the micro architecture of the DNN accelerators. To help understand and 
investigate the fault tolerance of the accelerators, we take a typical 
DNN accelerator with a regular 2D processing element (PE) array as an example 
and elaborate its architecture in this section.

The representative DNN accelerator is illustrated in 
Figure ~\ref{fig:npu-arch}. It adopts output stationary data flow 
to map computing such as convolution to the 2D computing array. 
Each PE in the array performs all the operations 
required to yield an output activation. While each PE has only a 
single multiplier and accumulator, it accumulates all the input 
activations in a filter window sequentially. To that end, weights 
are fed to the first column of the PE array in parallel and flow 
through PEs from left to right to ensure that all PEs operate in full 
scale. While input activations are organized in a batch and sent 
to one column of PE every cycle, each PE in the column shares the 
same input data through broadcasting and it takes 
each PE multiple cycles to complete the accumulation. 
During this period, more batched input activations 
can be read and sent to the next column of PEs along with 
the movement of the weights. Output activations flow 
from right to left in column-wise. Eventually, each row of the PEs
array produces a set of sequential output activations in 
the same row of one output feature map on y-axis. 
Each column of the PEs produces the output activations 
belonging to different output feature maps but the 
same position in z-axis. The architecture along with the 
compact data flow achieves high data reuse under limited 
on-chip buffer bandwidth provision. 

\begin{figure}
    \center{\includegraphics[width=0.8\linewidth]{npu-arch}}
    \caption{Typical DNN accelerator architecture}
\vspace{-0.5em}
\label{fig:npu-arch}
\end{figure}

Both convolution layer and full connection layer can be mapped to the 
array efficiently. While pooling and other non-linear activation functions 
such as sigmoid will be performed right after the computing-intensive 
layers like convolution layer in a module named XPE such that 
the data movement between the two layers can be reduced. 
All the neural network operations can be mapped to the 
accelerator. To map diverse neural 
network models with different combination of layers and parameters such as 
stride size, kernel size, and input/output feature map size, we define 
a set of instructions to generate appropriate control signals 
for different neural network operations. Each neural network 
will be compiled to a series of instructions and executed sequentially. 
In addition, neural network input features and weights 
are usually larger than the on-chip buffers and PE array size, 
so they must be tiled and the tiles need to be scheduled to 
obtain efficient execution on the accelerator. To enable fine-grained 
optimizations, each instruction only handles operations of 
a single tile. Thus, tiling is performed during model
compilation and it is transparent to the instructions.

Table ~\ref{tab:instrction-set} shows the instruction set of the neural 
network accelerator. It adopts 64-bit fixed length encoding 
and consists of four types of instructions including parameter 
setup, calculation, data movement and control. The parameter setup 
category defines the input/output feature size, kernel size, 
Q-code, and DMA parameter. Calculation 
category includes different operations in neural networks such as 
convolution, full connection, pooling, addition, softmax, 
dot-accumulation and activation function etc. Data movement category
includes three instructions which move a block of data 
from DRAM to buffer, buffer to DRAM and buffer to buffer 
respectively. Finally, control category includes three instructions which are 
Jump, Stop and Nop. Jump instruction is mainly used for repeated 
execution. Stop is used to terminate the execution of the accelerator.
Nop is used to resolve the data dependency between sequential 
instructions.

\begin{table}
    \centering
    \caption{Instruction Set of the DNN Accelerator}
    \label{tab:instrction-set}
    \begin{tabular}{cp{0.6\columnwidth}}
        \toprule
        Instruction Type & Description \\
        \midrule
        Parameter setup & Setup parameters for the computing operations such as the input/output feature size, kernel size, Q-code, and DMA options\\
        \midrule
        Calculation & Performs various neural network operations such as convolution, full connection, pooling, addition, softmax, dot-accumulation and activation function \\
        \midrule
        Data movement & Move a block of data from buffer to buffer, buffer to DRAM and DRAM to buffer.\\
        \midrule
        Control & Control the execution of the accelerator such as Jump, Nop and Stop\\
    \bottomrule
    \end{tabular}
    \vspace{-1em}
\end{table}

The neural network accelerator architecture is general enough to support 
various neural network models. In addition, it typically works along 
with a general purposed processor and has an AXI slave port that allows 
configuration and controlling from the attached processor. It assumes 
the input data, weight and output data are stored in DRAM that can be 
accessed directly. 
\section{Motivation} \label{sec:motivation}
Clock frequency determines the accelerator operation speed 
and directly affects the performance. Accordingly, it also has influence on the 
neural network runtime and energy efficiency. In this section, we take 
an open-sourced CNN accelerator named PipeCNN \cite{pipecnn_2} as an example and analyze the 
influence of clock frequency on the neural network performance and energy efficiency.

The accelerator is implemented on KCU1500 and attached to a desktop computer with 
Intel i7-6700@3.40GHz. The basic convolution structure is shown in Fig \ref{fig:cnn-arch}. 
The computing array consists of 16 dot production units which is also named as 
processing elements (PE). Each PE allows a parallel dot production of two 4-data vectors.
The optimized clock frequency according to the SDAccel compilation 
is 200 MHz. 

\begin{figure}
	\center{\includegraphics[width=0.75\linewidth]{accelerator}}
    \caption{Baseline CNN accelerator architecture.}
\label{fig:cnn-arch}
\vspace{-1em}
\end{figure}


In order to evaluate the influence of clock 
frequency, we set the clock to 50 MHz, 100 MHz, and 150 MHz respectively.
A set of neural networks including LeNet, AlexNet, VGG-16 and VGG-19 are used as the benchmark.
Both the runtime and enenrgy efficiency are normalized to the 50 MHz implementation.
Normalized runtime of the the neural network benchmark executed on the accelerators are 
shown in Fig \ref{fig:computing-bound}. It shows that the 
overall performance of the neural network benchmark
almost increases proportional to the clock frequency. 
It can be predicted that the processing remains 
computing-bound and high-frequency CNN accelerator 
design will be beneficial.

\begin{figure}
	\center{\includegraphics[width=0.75\linewidth]{relative_time}}
    \caption{Normalized runtime of neural networks executed on CNN accelerators with different clock frequency.}
\label{fig:computing-bound}
\vspace{-1em}
\end{figure}

To analyze the enenrgy consumption, we obtain the power consumption from SDAccel report. 
The power estimation setup assumes 50\% activity. Given the power consumption and the runtime, 
we calcualted the energy delay product (EDP) which can be used as an energy efficiency metric.
The experiment result is presented in Fig \ref{fig:edp}.
\begin{figure}
	\vspace{-1em}
	\center{\includegraphics[width=0.75\linewidth]{relative_energy}}
    \caption{Normalized energy-delay product of neural networks executed on CNN accelerators with different clock frequency.}
\label{fig:edp}
\vspace{-1em}
\end{figure}

According to the above experiments, it can be conclcuded that higher clock frequency 
can be beneficial to both the neural network performance and energy efficiency. The 
potential performance and energy efficiency improvement indicates that it is 
worthwhile to boosting the clock of FPGA based CNN accelerators with some minor 
overhead. Detailed overclocking on CNN accelerators will be investigated in 
detail in the rest part of this paper. 

\section{HLS based BFS optimization} \label{sec:bfs-opt}
This work centers the HLS based BFS optimization on FPGAs. 
In order to guide the HLS tools to generate efficient hardware, 
we propose a series of methods to regularize the irregular memory 
accesses and dynamic data paths in BFS algorithm. 

\subsection{Irregularity in BFS Structure}
The basic pipelined BFS structure with classical top-down traverse 
is presented in Figure \ref{fig:base-bfs}. It can be roughly 
divided into four pipeline stages. In the first stage, it reads 
frontier from memory. Then it passes the frontier to the second stage
via the OpenCL channel for further inspection. In the second stage, 
frontier neighbors will be inspected from the graph data. While the 
graph is stored as compressed sparse row (CSR) format which has a row 
pointer array (RPA) containing the edge index starting position of each 
vertex and a column index array (CIA) which is essentially the incoming/outgoing 
neighboring vertex indices, the second stage must go through the RPA read and 
CIA read sequentially. When the frontier neighbors are 
drained from memory, the second stage then forwards them to the third stage.
In the third stage, the each neighboring vertex will be checked if it is 
already visited in previous BFS iterations. If the vertex is unvisited, 
it will be considered as frontier in next BFS iteration. The corresponding 
vertex status will be set and the vertex index will be sent to the last stage.
In the last stage, the vertex indices will be written to memory one 
after another.

\begin{figure}
\center{\includegraphics[width=0.75\linewidth]{base-bfs}}
    \caption{Baseline pipelined BFS}
\label{fig:base-bfs}
\end{figure}

The basic BFS structure is well pipelined, but we notice that it involves many 
random memory accesses. In the second stage, as the vertex indices in the frontier 
are usually not continuous, thus the RPA read becomes random. For vertices with 
larger degree, the CIA read can be considered as sequential memory access. Nevertheless, 
the vertex degree can be a random integer, so the CIA read is aligned to 
the vertex index data width (4 bytes in this work) and the bandwidth utilization of 
the sequential memory access remains limited. In the third stage, when the status array 
is located in the memory, the status read is also random. When the vertex is un-visited, 
random write is also required. In the last stage, frontier vertices write is performed one 
after another, it is also random memory access by default. These random memory access 
essentially leads to low meory bandwidth utilization. 

While the data paths in the first, second and fourth pipeline stages are parallel,
the third pipeline stage can not be easily parallelized. Basically, the neihgbors of 
the different frontier vertices may be overlapped. When the frontier vertices are 
processed in parallel, they may update the same vertex status and result in write 
conflict. Worse still, more parallel random accesses will not improve the memory 
bandwidth utilization and increase the average memory access latency instead. 
This stage soon becomes the performance bottleneck. Keeping the vertex status on-chip 
may alleviate the memory access bottleneck, but it is difficult to synchronize the 
vertex status in parallel using HLS. This also prevents the parallelization in the 
other pipeline stages. The authors xx proposed to build a crossbar based shuffling 
between the second stage and the thrid stage. However, the complex logic increases the 
initiation interval (II) dramatically and the overall accelerator performance degrades.

In order to address this problem, we proposed the following optimization methods 
to regularize the memory accesses as well as the data paths so that it can be 
efficiently implemented by the HLS tools. 1) We reorder the edges of the graphs and insert 
padding data with pre processing. It essentially batches the edges and ensures 
no write conflict in each batch. 2) We use bitmap to store the vertex visiting status and 
keep the bitmap in on-chip memory. Meanwhile the bitmap is divided into parallel banks 
based on the pre processing and parallel data paths can operate on the different 
banks independently. 3) We have the coupled CPU to gather the frontier vertices' RPA into 
a sequential array. Then the accelerator can start with sequential RPA read and ensures 
efficient data processing the following pipeline stages. 4) The fourth pipeline stages 
are parallelized and each parallel path is optimized with batch write. Each optimization method
will be detailed in the rest part of this section.

\subsection{Graph reordering and padding}
As analyzed in Section xx, parallel data path in the third pipeline stage will cause 
write conflict. To avoid the write conflict, we divide the vertices of the graph into 
different segments based on modular operation such that each data path can operate on 
the different segments independently. 

We reorder the edges of the graph and 
add padding to the edge data as shown in Figure \ref{fig:graph-reorder}. Basically, 
we divide the edges (i.e. CAI array) into small batches. In each batch, the destination 
vertices of the edges will be evenly distributed into different segments of the vertex. 
evenly split 
divided based on their Take the outgoing edge as an example,

\begin{figure}
\center{\includegraphics[width=0.75\linewidth]{graph-reorder}}
    \caption{CSR layout after the graph reordering and padding}
\label{fig:graph-reorder}
\end{figure}

\subsection{Bitmap partition and parallel access}
\subsection{CPU assisted data reorganization}
\subsection{Batch write}

\subsection{Optimized BFS structure}
Optimized BFS structure is presented in Figure \ref{fig:opt-bfs}.
\begin{figure}
	\center{\includegraphics[width=0.85\linewidth]{opt-bfs}}
    \caption{Optimized BFS pipeline}
\label{fig:opt-bfs}
\end{figure}

\subsection{Theoretical performance analysis}
In this sub section, we analyzed the theoretical performance.

\section{Experiments} \label{sec:experiment}
We measure the performance of the HLS based BFS 
accelerator on Alpha Data ADM-7v3 and KU115 using a set of 
representative graphs and compare them to both a 
baseline design and previous handcrafted BFS accelerators. 
The baseline design refers to a design with HLS pragmas 
added to the native C based level synchronous BFS implementation. 
Then we briefly evaluate the design optimization 
methods including pipelining, redundancy removal, 
caching and data path duplication 
proposed in this work. Based on the design on ADM-7v3, 
we further ported the design 
to KU115 on Nimbix Cloud and explored the portability of the 
BFS accelerator. 

\subsection{Experiment Setup}
The graph benchmark used in this work includes three real-world graphs and 
two synthetic graphs generated using R-MAT model \cite{chakrabarti2004rmat} 
as listed in Table \ref{tab:graph}. The real-world graphs are from social network 
\cite{snapnets} while the R-MAT graphs are generated 
using the Graph 500 benchmark parameters ($A=0.59, B=0.19, C=0.19$). To make the 
presentation easier, the five benchmark graphs are shorted as Youtube, 
LJ, Pokec, R-MAT\uppercase\expandafter{\romannumeral1}, 
R-MAT\uppercase\expandafter{\romannumeral2} respectively. We refer 
to an R-MAT graph with scale $S$ ($2^{S}$ nodes) and edge factor $E$ ($E\times 2^{S}$). 
In order to avoid trivial search, we only choose vertices from the largest 
connected component as the BFS starting point.

\begin{table}
    \centering
  \vspace{-0.3em}
  \caption{Graph Benchmark}
  \label{tab:graph}
  \vspace{-0.3em}
  \begin{tabular}{cccc}
    \toprule
      Name & \# of vertex & \# of edge & Type \\
    \midrule
      Youtube & 1157828 & 2987624 & Undirectional \\
      LJ & 4847571 & 68993773 & Directional \\
      Pokec & 1632804 & 30622564 & Directional \\
      R-MAT\uppercase\expandafter{\romannumeral1} & 524288 & 16777216 & Directional \\
      R-MAT\uppercase\expandafter{\romannumeral2} & 2097152 & 67108864 & Directional \\
  \bottomrule
\end{tabular}
\vspace{-1em}
\end{table}

\subsection{Performance comparison}
We use the million traverse per second (MTEPS) as 
the performance metric. The performance of the proposed BFS 
accelerator on the graph benchmark is 
presented in Table \ref{tab:performance-summary}. 
The implementation on ADM-7v3 achieves up to 
82.16 MTEPS on the R-MAT\uppercase\expandafter{\romannumeral1} graph and 
38.83 MTEPS on average. When compared to a baseline HLS based 
BFS accelerator, the proposed design shows 24.7X to 77.5X performance 
speedup on the benchmark. 
With the comparison, it is clear that straightforward HLS optimizations 
are far from sufficient and dedicated high level optimizations are critical to 
the performance of the resulting BFS accelerator.
\begin{table}
  \vspace{-0.3em}
    \centering
  \caption{Performance summary}
  \vspace{-0.3em}
  \label{tab:performance-summary}
  \begin{tabular}{cccccc}
    \toprule
      Benchmark & Youtube & LJ & Pokec & RMAT\uppercase\expandafter{\romannumeral1} & RMAT\uppercase\expandafter{\romannumeral2} \\
    \midrule
      MTEPS & 14.35 & 28.05 & 36.94 & 82.16 & 32.67 \\
      Speedup & 77.50 & 36.82 & 38.83 & 62.18 & 24.70 \\
  \bottomrule
\end{tabular}
\vspace{-1em}
\end{table}

We also compare this work to a set of existing BFS accelerators on FPGAs. 
As the platforms and graph benchmarks used in these work are mostly different and it is 
difficult to make a complete fair end-to-end comparison. Here we provide two implementations 
on Alpha-Data ADM-7v3 and KU115 respectively. A rough comparison result is listed 
in Table \ref{tab:compare}. The best HLS based BFS implementation on KU115 is 
getting close to that in \cite{zhang2017boosting} 
and \cite{nurvitadhi2014graphgen}, though the peak memory bandwidth 
is relatively higher. When compared to design on high-end 
FPGA computing system such as Convey HC-2 with highly optimized memory sub systems, 
the performance is still much lower. 

Since different FPGAs may have diverse memory bandwidth, we also measure the 
per bandwidth BFS performance i.e. MTEPS/GB. According to the experiments, we 
can see that the HLS based BFS accelerator on Alpha-Data achieves higher 
MTEPS/GB. The comparison shows that the memory bandwidth on KU115 is not 
fully explored. This is mainly caused by the fact that only 16 global memory 
ports are allowed to be implemented in the SDAccel design and 
limited parallel data paths can be instantiated on the FPGAs as 
mentioned in previous section. We believe the performance of the 
proposed design can be further improved given more parallel data paths.

\begin{table}
  \vspace{-0.3em}
  \caption{FPGA based BFS accelerator comparison}
  \label{tab:compare}
    \setlength{\tabcolsep}{4pt} % Default value: 6pt
    %\renewcommand{\arraystretch}{1.5} % Default value: 1
  \vspace{-0.3em}
  \begin{tabular}{cccccc}
    \toprule
      Work & Platform & Graph & MTEPS & BW(GB/s) & MTEPS/GB \\
    \midrule
      \cite{betkaoui2012reconfigurable} & Convey HC-2 & R-MAT & 1600 & 80  & 20 \\
      \cite{attia2014cygraph} & Convey HC-2 & R-MAT    & 1900 & 80  & 23.8 \\
      \cite{zhang2017boosting} & Micro-AC510       & R-MAT  & 166.2  & 60  & 2.8 \\
      \cite{nurvitadhi2014graphgen} & VC707 Kit & Twitter & 148.6 & 12.8 & 11.6 \\
      \cite{dai2016fpgp}  & VC707 Kit & Twitter & 12  & 12.8 & 0.95 \\
      this work & ADM-7v3 & R-MAT & 57.41 & 10.8 & 5.3 \\
      this work & ADM-7v3 & Table \ref{tab:graph} & 38.8 & 10.8 & 3.6 \\
	  this work & KU115 & R-MAT & 120.84 & 76.8 & 1.57\\
	  this work & KU115 & Table\ref{tab:graph} & 77.98 & 76.8 & 1.02\\
  \bottomrule
\end{tabular}
\vspace{-1em}
\end{table}

\subsection{Design Configuration and Resource Overhead}
With the software emulation based tuning, 
we can decide the design configurations rapidly. The graph specific configuration 
of the BFS accelerator targeting ADM-7v3 is summarized in 
Table \ref{tab:parameter-setup}. The hash tables for LJ and R-MATII 
as highlighted in the table are shrunk to fit for the on-chip memory constraints. 
Note that the \textit{depth} read and write cache 
are set to be the same and the cache size in the table refers to the capacity of 
one cache size.

\begin{table}
  \vspace{-0.3em}
  \caption{Memory optimization parameter setup on ADM-7v3}
  \label{tab:parameter-setup}
  %\setlength{\tabcolsep}{4pt} % Default value: 6pt
  %\renewcommand{\arraystretch}{1.5} % Default value: 1
    \centering
  \vspace{-0.3em}
  \begin{tabular}{ccccccc}
    \toprule
      Benchmark & Hash Table & Cache Size & Prefetch Buffer \\
    \midrule
      Youtube  & 256K  & 16K $\times$ 64B & 64B \\
      LJ       & \textbf{512K} & 32K $\times$ 64B & 64B \\
      Pokec    & 1024K & 16K $\times$ 64B & 64B \\
      R-MATI   & 512K  & 8K $\times$  64B & 64B \\
      R-MATII  & \textbf{512K} & 32K $\times$ 64B & 64B \\
  \bottomrule
\end{tabular}
\vspace{-1em}
\end{table}

The corresponding FPGA resource consumption is 
presented in Table \ref{tab:mem-resource}. 
FF and LUT consumption don't change much with the different 
design configurations and they take up only a small portion 
of the total FPGA resources. Block RAMs turns out to be the major 
resource bottleneck, and it leads to the adoption of sub optimal 
design configurations.

\begin{table}
  \vspace{-0.5em}
  \caption{FPGA resource consumption on ADM-7v3}
  \label{tab:mem-resource}
  \vspace{-0.3em}
  %\setlength{\tabcolsep}{4pt} % Default value: 6pt
  %\renewcommand{\arraystretch}{1.5} % Default value: 1
    \centering
  \begin{tabular}{ccccccc}
    \toprule
      Config. & FF & \% & LUT & \% & RAMB18K & \% \\
    \midrule
      Youtube  & 65244 & 7 & 108810 & 25 & 1515  & 51 \\
      LJ       & 65266 & 7 & 108829 & 25 & 2784  & 94 \\
      Pokec    & 65262 & 7 & 108812 & 25 & 2155 & 73 \\
      R-MATI   & 65244 & 7 & 108808 & 25 & 1217 & 41 \\
      R-MATII  & 65266 & 7 & 108829 & 25 & 2784 & 94 \\
  \bottomrule
\end{tabular}
\vspace{-1em}
\end{table}

\subsection{Optimization evaluation}
In this section, we evaluate the 
performance of the BFS accelerators with the different optimizations. 
Basically we start from the baseline design and 
add the optimizations including pipelining, hash redundancy removal, 
prefetching, caching and data path duplication in order. The performance improvement 
can be found in Figure \ref{fig:opt-performance}. 
In general, the performance of the BFS accelerator improves 
significantly when more optimization techniques are applied. Particularly,
pipelining and data path duplication enhance the performance most. 
The performance improvement brought by the hash table based filtering 
seems to be trivial, but it actually boosts the performance by over 20\% on average. 
In addition, it also affects the cache efficiency as observed in Section \ref{sec:observation}
and is thus critical to the overall accelerator performance.

\begin{figure}
\center{\includegraphics[width=0.85\linewidth]{opt-performance}}
    \caption{BFS accelerator optimization technique evaluation. The performance on 
    all the graphs improves when more optimizations including pipelining, 
    redundancy removal, prefetching, caching, and data path duplication are 
    gradually applied to the design.}
\label{fig:opt-performance}
\vspace{-1em}
\end{figure}

There is only one memory bank available in ADM-7v3, so we 
evaluate the memory bank-aware data layout strategy by porting the design to KU115.
Without the bank-aware layout optimization, porting the design from ADM-7v3 to 
KU115 achieves less performance improvement despite the much larger memory 
bandwidth on KU115. When the optimization is applied, the multiple-bank memory 
on KU115 can be utilized. The performance of the BFS accelerator on KU115 improves 
significantly as shown in Table \ref{tab:porting-summary} especially for the 
graphs with more edges. This experiment also 
demonstrated the portability of the proposed HLS based BFS accelerator.
\begin{table}
	\vspace{-0.5em}
    \centering
	\caption{Memory-bank aware data layout optimization influence on the BFS accelerator performance (MTEPS)}
  \label{tab:porting-summary}
  \vspace{-0.3em}
  \begin{tabular}{cccccc}
    \toprule
	Benchmark & Youtube & LJ & Pokec & RMAT\uppercase\expandafter{\romannumeral1} & RMAT\uppercase\expandafter{\romannumeral2} \\
    \midrule
	ADM-7v3 & 14.35 & 28.05 & 36.94 & 82.16 & 32.67 \\
	KU115 with bank opt. & 18.69 & 61.49 & 68.04 & 122.48 & 119.2 \\
	KU115 no bank opt. & 14.15 & 41.74 & 40.82 & 85.18 & 63.31\\
  \bottomrule
\end{tabular}
\vspace{-1em}
\end{table}

\section{Discussion}
According to our experience using SDAccel for BFS accelerator design, 
we have come up with a few insights on accelerating complex applications 
with HLS design tools. 1) The high level design tools are able to produce competitive 
hardware implementations for not only regular applications but 
also irregular applications. 2) Optimizing the HLS design for higher performance is non-trivial. It is not 
so much friendly to software programmers as expected. Hardware design 
experiences are important for optimizing the HLS based design. 
3) The HLS tools are still not as mature as the software compilers. 
It is not rare to come across a design which is functionally correct 
in software emulation but wrong in hardware implementation. The bugs are 
even more difficult to address than the HDL design bugs. It may be caused by 
many different factors such as concurrent memory access conflict or deadlock. 
4) Applications with dynamic memory accesses can be found in many applications. 
And customized hardware architectures such as hash table, and cache, channels etc are 
classical wisdom that can be applied to optimize these applications. However, 
these building blocks in existing HLS design tools are still quite limited. More 
support on these features will be beneficial to complex application 
acceleration with HLS tools.

\section{Conclusions} \label{sec:conclusion}
Handcrafted HDL based BFS accelerators usually suffer 
high portability and maintenance cost 
as well as ease of use problem despite the relatively 
good performance. HLS based BFS accelerator can greatly 
alleviate these problems, but it is 
difficult to achieve satisfactory performance due to 
the inherent irregular memory access and 
complex nested loop structure. In this work, we developed 
a series of HLS based optimizations such as redundancy removal, 
prefetching, caching and data path duplication. 
With the optimizations, BFS performance can be greatly improved. 
According to the experiments on 
a representative graph benchmark, the resulting HLS based BFS accelerator 
achieves up to 70X speedup compared to a baseline HLS design. 
When compared to the existing HDL based BFS accelerators on 
similar FPGA cards, the proposed HLS based BFS accelerator on KU115 
gets over 70\% of the MTEPS while it still preserves the 
software-like features including portability and ease of use 
and maintenance.  


\section{Related Work}\label{sec:relatedwork}
Despite their promising performance advantage, the relatively low design productivity of developing
FPGA applications remains a major obstacle that hinders widespread adoption of FPGAs as commodity
computing devices. To address this problem, the design of QuickDough was inspired by the recent success in HLS tools.
It also took advantage of modern FPGAs' capabilities to allow for an additional overlay architecture
be employed for productivity sake.

\subsection{High-Level Synthesis}
To bridge the design productivity gap between software and hardware application development, many researchers have turned to the use of HLS techniques \cite{cong2011high}.
By raising the abstraction level of the physical hardware, HLS allows designers to express hardware designs using familiar high-level, software-like description languages such as C, Java, or Python \cite{cardoso2010compiling,Canis:2011:LHS:1950413.1950423}.
The low-level hardware implementations are then left to the tools to synthesize and optimize.
Indeed, with decades of research, some early results in HLS have already found their ways into FPGA vendors' commercial tools in recent years \cite{chen2005xpilot, zhang2008autopilot, VivadoHLS}.

Unfortunately, when considering the overall design productivity of developing hybrid software-gateware applications, the raised abstraction provided by HLS is only addressing part of the problem.
While the high-level abstraction makes expressing complex functionalities as FPGA gateware easier, the lengthy low-level compilation time spent in synthesis, mapping, placing and routing remains a bottleneck to the overall design productivity for an application designer.
Such long compilation time is particularly challenging for novice designers who are accustomed to the high speed of software compilation.
Most importantly, it is significantly impacting the possible compile-debug-edit cycle achievable per day by a designer, negatively impacting the productivity of the designer.

\subsection{Overlay Architectures}
To improve the speed of low-level implementation tools, researchers have explored various approaches over the past decades.
Inspired by application specific integrated circuit (ASIC) design flows, researchers and vendors have developed modular design flow and explored the use of pre-compiled hard macros \cite{lavin2010using,lavin2011} as implementation library.
In addition, researchers have also exploited the use of dynamic partial reconfiguration capabilities in FPGAs \cite{Frangieh2010} as a way to improve productivity.
In recent years, there has been an increased interest in applying the concept of \emph{overlay architectures} as a way to address this productivity challenge.  


An overlay architecture is a virtual intermediate architecture that is overlaid on top of the physical configurable fabric of an FPGA.  They are employed during the FPGA application implementation process for purposes such as to improve portability, security, and also productivity.
%Depending on the design goal, overlays have manifested in various forms, including HDL models, pre-synthesized or pre-implemented coarse-grained circuits, or even arrays of processing elements with various granularity. 

One of the most familiar categories of overlay consists of virtual FPGAs \cite{zuma2013carl,Grant2011Malibu,Coole2010Intermediate,Koch2013CI}. They are built either virtually or physically on top of off-the-shelf FPGA devices and typically feature coarser configuration granularity than the physical device.
Similar to virtual machines running on a typical computer, such virtual FPGA provides an additional layer that improves application portability and security.
Furthermore, because of the coarser-grained configurable fabric, implementing designs on such overlay is relatively easier than on a fine-grained device.
However, the additional layer imposes restrictions on the underlying fabrics' capability and usually results in moderate hardware overhead and timing degradation.

Another category of overlay architecture commonly employed is in the form of coarse-grained reconfigurable arrays (CGRAs).
The use of CGRAs provides unique advantages of performance especially for compute intensive applications as demonstrated by numerous ASIC CGRAs \cite{tessier2001reconfigurable} \cite{compton2002reconfigurable}.
Indeed, CGRAs on FPGA and ASIC have many similarities in terms of the scheduling algorithm and array structure.
However, they have quite different trade-offs in terms of configuration flexibility, overhead and performance.
In a nutshell, CGRAs on ASIC emphasize more on configuration capability to cover more applications, while FPGAs' inherent programmability greatly alleviates the concern.
Instead, CGRAs on FPGA may take advantage of the configurability of the underlying fabric to allow more intensive customization tailored to the target application.

The authors in \cite{kissler2006dynamically} developed WPPA (weakly programmable processor array), a VLIW architecture based parameterizable CGRA overlay. It featured an interconnection wrapper unit for each processing element (PE) that could be used for dynamic CGRAs topology customization. Unfortunately, programming and compilation on WPPA were not presented. The authors in \cite{ferreira2011fpga} proposed a heterogeneous CGRA overlay with a global multi-stage interconnection on FPGA. Compiling applications onto the overlay took only milliseconds for smaller DFGs. However, the global multi-stage interconnection required multiple stages for communication between each pair of PEs and resulted in either low implementation frequency or large communication latency in terms of cycles. In addition, there was no intermediate storage except the pipeline registers in the CGRA and it limited the performance of the operation scheduling.
In \cite{shukla2006quku}, a customized CGRA overlay called QUKU was developed for DSP algorithms. It had two-level configuration capability, while the low-speed configuration was used for operator reuse within an application and high-speed reconfiguration was used for optimization between different applications. Nevertheless, the hardware infrastructure was consist of simple operation elements which can only be adapted to a few specified DSP algorithms.
The authors in \cite{capalijia2013pipelined} built a more generic high speed mesh CGRA overlay using the elastic pipeline technique to achieve the maximum throughput. It adopted a data-driven execution flow and was suitable for smaller pipelined DFG execution, while it would be difficult to handle applications with random IO access. 

In general, previous CGRA overlays have demonstrated the promising performance acceleration capability for compute intensive applications. They typically take DFG as design entry and focus on hardware infrastructure design as well as corresponding mapping and scheduling. However, they are still lack of consideration on proper loop unrolling for DFG generation, on-chip buffering, the communication with host and even end-to-end performance which are essential for FPGA accelerator design especially from a HW/SW co-design engineer's perspective. 


Finally, a third category of overlay features soft-processor-like architectures with high degree of
control and data parallelism suitable for FPGA accelerations.  For example, in the work of MARC
\cite{Lebedev2010}, a many-core overlay with customizable data path was proposed.  Similarly, a
GPU-like overlay was proposed in \cite{Jeffrey2011potential}.


In this work, we opted to utilize a fully pipelined synchronous soft coarse-grained reconfigurable
array (SCGRA) as an overlay to facilitate rapid FPGA accelerator generation in a hybrid CPU-FPGA
system. Compared to previously proposed CGRAs, our overlay is designed to be \emph{soft} as the size,
processing element designs, as well as the interconnect topologies may all be customized as needed
providing just enough resource for an application specifically. Moreover, the design of our overlay
is regular and design parameters such as loop unrolling factor and overlay size have
relatively predictable influence on the overlay performance and overhead, which makes the
customization much easier and more efficient. Finally, it also takes advantage of the large number
of on-chip distributed memory on the FPGA for intermediate data storage and can handle large DFGs
with thousands of nodes. 

%On top of the above approaches, the use of \emph{overlays} in the form of HDL Model, pre-synthesized or pre-implemented coarse-grained reconfigurable circuits over the fine-grained FPGA devices, promises both to raise the abstraction level and reduce the compilation time.
%Recent years have seen a number of overlay designs being developed with granularities ranging from multi-processors to highly configurable logic arrays \cite{Lebedev2010,kissler2006dynamically,unnikrishnan2009application,Yiannacouras2009FPS,Guy2012VENICE,Jeffrey2011potential}. 

%% Not so much overlay, removed for clarity sake.
%Soft processors, which allow customization for target applications or application domains, have already been demonstrated to be efficient overlays on FPGA. A great number of work use embedded processors as FPGA overlays with micro-architecture parameters such as pipeline depth configurable \cite{Yiannacouras2007Exploration,microblaze,nios} and 


%instruction set architecture (ISA) customizable \cite{grad2009woolcano, }. 


%multi-processor overlay with both micro-architecture and interconnection customizable \cite{unnikrishnan2009application}, 

% vector processors overlay \cite{Guy2012VENICE,Yiannacouras2009FPS}


 
\section{Conclusion} \label{sec:Conclusion}
In this paper, we propose to take the CNN accelerator’s ‘undeterministic’ behaviors into consideration 
at training and have the CNN model to learn the accelerator’s behaviors. To that end, we further build 
an open-sourced training system based on Caffe on a hybrid CPU-FPGA architecture. Then use the training 
system to deal with an overclocked CNN accelerator and an accelerator with soft errors. According to our 
experiments, the proposed training can improve the prediction accuracy of four CNN models up to 3.4\% when 
the CNN accelerator is overclocked on the extreme situation. This method is also beneficial to the CNN 
accelerators with soft errors. In the case with most soft errors, it improves the prediction accuracy up 
to 6.8\% and by 3.58\% on average. The disadvantage is the much longer training time due to the frequent 
data transfer between host memory and device memory. This problem can be resolved when porting the system 
to closely coupled CPU-FPGA architectures with shared memory.

%\appendix
%\section{Acknowledgement}

%\begin{acks}
%  The authors would like to thank Sam Ho for providing the suggestions on
%  HLS design debugging and optimization as well as the SDAccel usage. 

%\end{acks}


\bibliographystyle{IEEEtran}
\bibliography{refs} 


% trigger a \newpage just before the given reference
% number - used to balance the columns on the last page
% adjust value as needed - may need to be readjusted if
% the document is modified later
%\IEEEtriggeratref{8}
% The "triggered" command can be changed if desired:
%\IEEEtriggercmd{\enlargethispage{-5in}}

% references section

% can use a bibliography generated by BibTeX as a .bbl file
% BibTeX documentation can be easily obtained at:
% http://mirror.ctan.org/biblio/bibtex/contrib/doc/
% The IEEEtran BibTeX style support page is at:
% http://www.michaelshell.org/tex/ieeetran/bibtex/
%\bibliographystyle{IEEEtran}
% argument is your BibTeX string definitions and bibliography database(s)
%\bibliography{IEEEabrv,../bib/paper}
%
% <OR> manually copy in the resultant .bbl file
% set second argument of \begin to the number of references
% (used to reserve space for the reference number labels box)
%\begin{thebibliography}{1}

%\bibitem{IEEEhowto:kopka}
%H.~Kopka and P.~W. Daly, \emph{A Guide to \LaTeX}, 3rd~ed.\hskip 1em plus
%  0.5em minus 0.4em\relax Harlow, England: Addison-Wesley, 1999.

%\end{thebibliography}




% that's all folks
\end{document}


