\section{Limitations and Conclusion} \label{sec:conclusion}
\subsection{Limitations}
In this work, we make use of the host CPU for BFS, but it is not an optimized co-design. The processing power is not fully utilized.
In fact, we mainly take advantage of the host processor cache. Most of the time, it is not used at all. We may further 
explore the heterogeneous architecture.

We assume the bitmap should be able to fully buffered on the FPGA on-chip RAM blocks. Basically, we limit the number of vertices in the graph that 
can be processed using this work. While the latest FPGAs has over 500Mb on-chip memory, we are supposed to handle graphs with hundreds 
of millions of vertices in theory. And we do not constrain the size of the edges in the graph. To handle larger graphs, we need to further explore 
the graph partitions. While this work already promises the efficiency of a big partition.

\subsection{Conclusion}
Handcrafted BFS accelerators with HDL usually suffer high portability and maintenance cost 
as well as ease of use problem despite the relatively 
good performance. HLS based BFS accelerator can greatly alleviate these problems, but it is 
difficult to achieve satisfactory performance due to the inherent irregular memory access and 
complex nested loop structure. In this work, we regularize the basic BFS algorithm with  
a series of optimizations. The regularized BFS can be implemented using OpenCL efficiently.
Compared to a reference design in Spector benchmark, it is up to 9X performance speedup.
When compared to the prior HDL based BFS accelerators on similar FPGA cards, 
the proposed OpenCL based BFS accelerator achieves comparable performance.


