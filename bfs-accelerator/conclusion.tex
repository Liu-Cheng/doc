\section{Limitations and Conclusion} \label{sec:conclusion}
\subsection{Limitations}
In this work, we mainly explore the OpenCL based BFS accelerator 
design on FPGAs, though Harp-v2 is a heterogeneous CPU-FPGA computing platform. 
We may further explore co-design on the heterogeneous architecture.

We assume that the bitmap can be fully buffered on the FPGA on-chip RAM blocks. 
Basically, we limit the number of vertices in the graph that 
can be processed using this work. While the latest FPGAs has over 
500Mb on-chip memory, we are supposed to handle graphs with hundreds 
of millions of vertices in theory on a single FPGA card. 
And we do not constrain the size of the edges in the graph. 
To handle larger graphs, we need to further explore 
the graph partitioning, while this work exhibits 
the efficiency of the BFS processing of a single graph partition.

\subsection{Conclusion}
Handcrafted BFS accelerators with HDL usually suffer high portability and maintenance cost 
as well as ease of use problem despite the relatively 
good performance. OpenCL based BFS accelerator can greatly alleviate these problems, but it is 
difficult to achieve satisfactory performance due to the inherent irregular memory access. 
In this work, we center the irregular memory access in BFS with  
a series of high-level optimizations. The resulting BFS can be implemented using OpenCL efficiently.
Compared to a reference design in Spector benchmark, it achieves up to 12X performance speedup.
When compared to the prior HDL based BFS accelerators on similar FPGA cards, 
the proposed OpenCL based BFS accelerator achieves competitive performance.


