\section{Related work} \label{sec:relatedwork}
The growing importance of efficient BFS traverse on large graphs 
have attracted attentions of many researchers. In the past few years, 
many BFS optimization algorithms and accelerators have been proposed 
on almost all the major computing platforms including multi-core processors, 
distributed systems, GPUs and FPGAs. In this work, we will 
particularly focus on the FPGA based BFS acceleration. 

The researchers tried to explore BFS acceleration 
on FPGAs from many various angles.
To alleviate the memory bandwidth bottleneck of the 
BFS accelerators, the authors in \cite{zhang2017boosting, khoram2018accelerating} 
explored the new Hybrid Memory Cube (HMC) which provides 
much higher memory bandwidth as well as flexibility for BFS 
acceleration, while the authors in \cite{attia2014cygraph} 
proposed to change the compressed sparse row (CSR) format slightly. 
Different from the first two work, the authors in \cite{umuroglu2015hybrid} 
choosed to perform some redundant but sequential memory access for higher memory bandwidth 
utilization based on a spare matrix-vector multiplication model.
In addition, they particularly took advantage of the 
hybrid CPU-FPGA architecture offloading only highly parallel 
BFS iterations for FPGA acceleration while leaving the rest 
on host CPU.  

Most of the BFS accelerators are built on a vertex-centeric 
processing model, while the authors 
in \cite{zhou2016high} explored the edge-centeric graph processing and demonstrated 
significant throughput improvement. On top of the single FPGA board acceleration, 
the authors in \cite{attia2014cygraph, betkaoui2012reconfigurable} also explored 
BFS acceleration on a FPGA based high performance computing system with multiple 
FPGAs and memory instances. There are also work exploring customized soft processors 
for graph processing and building a distributed solution on 
top of a group of embedded FPGA boards \cite{kapre2015custom, wang2010message}.

The researchers opted to develop 
more general graph processing accelerator framework or library 
recently \cite{engelhardt2016gravf, jun2018grafboost, yao2018efficient, oguntebi2016graphops, Dai2017foregraph, dai2016fpgp}. 
They can also be utilized for BFS acceleration despite the lack of 
specialized optimization for BFS. Meanwhile, this is also a way to improve 
the ease of use FPGAs for graph processing acceleration.


Prior BFS acceleration work have demonstrated the potential benefits of accelerating 
BFS on FPGAs. These accelerators were mainly developed for the sake 
of performance and generality for more graph processing algorithms. 
However, they were all handcrafted HDL designs. Developing the HDL based accelerators 
takes long time and applying these accelerators 
on high level applications for software designers still requires a lot of 
efforts especially when the target computing platforms are different. 
To that end, we opt to develop BFS accelerators with OpenCL
such that the accelerator can be easily ported to diverse FPGA devices 
and easily utilized in high level applications 
by a \textit{software designer}.
