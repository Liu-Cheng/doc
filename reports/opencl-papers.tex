%%%%%%%%%%%%%%%%%%%%%%%%%%%%%%%%%%%%%%%%%
% Simple Sectioned Essay Template
% LaTeX Template
%
% This template has been downloaded from:
% http://www.latextemplates.com
%
% Note:
% The \lipsum[#] commands throughout this template generate dummy text
% to fill the template out. These commands should all be removed when 
% writing essay content.
%
%%%%%%%%%%%%%%%%%%%%%%%%%%%%%%%%%%%%%%%%%

%----------------------------------------------------------------------------------------
%	PACKAGES AND OTHER DOCUMENT CONFIGURATIONS
%----------------------------------------------------------------------------------------

\documentclass[12pt]{article} % Default font size is 12pt, it can be changed here

\usepackage{geometry} % Required to change the page size to A4
\geometry{a4paper} % Set the page size to be A4 as opposed to the default US Letter

\usepackage{graphicx} % Required for including pictures

\usepackage{float} % Allows putting an [H] in \begin{figure} to specify the exact location of the figure
\usepackage{wrapfig} % Allows in-line images such as the example fish picture

\usepackage{lipsum} % Used for inserting dummy 'Lorem ipsum' text into the template

\usepackage{url}

\linespread{1.2} % Line spacing

%\setlength\parindent{0pt} % Uncomment to remove all indentation from paragraphs

\newcommand{\figref}[1]{Figure~\ref{#1}}
\graphicspath{{./figures/}} % Specifies the directory where pictures are stored

\begin{document}

%----------------------------------------------------------------------------------------
%	TITLE PAGE
%----------------------------------------------------------------------------------------

\begin{titlepage}

\newcommand{\HRule}{\rule{\linewidth}{0.5mm}} % Defines a new command for the horizontal lines, change thickness here

\center % Center everything on the page

\HRule \\[0.4cm]
{ \huge \bfseries OpenCL Related FPGA Paper Review}\\[0.4cm] % Title of your document
\HRule \\[1.5cm]

\begin{minipage}{0.4\textwidth}
    \centering {Cheng Liu} \\ 
    \vspace{2em}
    \centering {\large \today}
\end{minipage}

\end{titlepage}

%----------------------------------------------------------------------------------------
%	TABLE OF CONTENTS
%----------------------------------------------------------------------------------------

%\tableofcontents % Include a table of contents

%\newpage % Begins the essay on a new page instead of on the same page as the table of contents 

%----------------------------------------------------------------------------------------
%	INTRODUCTION
%----------------------------------------------------------------------------------------
\section{Paper Review} % Major section
Here is a list about recent papers on OpenCl based FPGA acceleration.
\begin{itemize}
    \item Improving the performance of OepnCL-based FPGA accelerator for convolutional neural network \cite{zhang2017improving}:
        1) The authors argue that a direct opencl-based OpenCL-based CNN design is insufificient to achieve 
        the desired performance. To that end, an analytical performance model is used to identify the 
        performance bottleneck of CNN on FPGAs.
        2) With the analysis, the authors thus propose a new kernel design to effectively 
        address the bandwidth optimization. Finally the resulting performance is even better 
        than previous design (mostly due to the much higher implementation frequency)
    \item A study of data partitioning on OpenCl based FPGAs \cite{wang2015study}:
        1) Explore the challenges of porting the opencl design for GPU to opencl design on FPGAs.
        2) Optimize the data paritioning using opencl optimization techniques such as on chip buckets.
        3) Develop of a cost model for design parameters optimization.

    \item An OpenCL Deep Learning Accelerator on Arria 10 \cite{aydonat2017opencl}: 
        1) proposed a methodology to minimize the bandwidth of convolutional layer and 
        full-connected layer by caching intermediate feature-map in stream buffers.
        2) proposed a design space exploration that leverages analytical models for 
        resource consumption and throughput and provides optimized architectural configuration.
        3) Introduce the Winograd transformation to reduce the amount of convolution operations.

    \item Energy efficient scientific computing on FPGAs using OpenCL: \cite{weller2017energy}:
        1) Implement partial differential equations (PDE) on FPGA and provide a series of 
        optimizations including vendor-dependent optimizations and vendor-independent optimizations.
        2) Discuss the portability of the opencl optimization between different vendors i.e. Xilinx 
        FPGA and Intel-Altera FPGA.
        3) Compare the performance and energy efficiency over CPU and GPUs.

    \item Accelerating Database query processing on opencl-based FPGAs \cite{wang2016accelerating}: 
        1) Implement the SQL query operators using OpenCL on FPGA.
        2) Build the performance model of the opecl based SQL operator performance. 
        With the models, optimized query plan can be generated.

    \item Evaluation of an OpenCl-based FPGA platforms for Particle Filter \cite{tatsumi2016evaluation}:
        1)The authors implemented the particle filter using OpenCL on FPGA.
        2) It allows the designers to achieve significant performance speedup without 
        much hardware circuit design expertises. 
    \item OpenCL-Based FPGA Accelerator for 3D FDTD with periodic and absorbing boundary conditions \cite{waidyasooriya2017opencl}
        This work also focuses on the opencl based algorithm acceleration.
\end{itemize}


%----------------------------------------------------------------------------------------
%	BIBLIOGRAPHY
%----------------------------------------------------------------------------------------
\bibliographystyle{plain}
\bibliography{refs}
%----------------------------------------------------------------------------------------

\end{document}
