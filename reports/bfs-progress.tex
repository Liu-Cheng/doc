%%%%%%%%%%%%%%%%%%%%%%%%%%%%%%%%%%%%%%%%%
% Simple Sectioned Essay Template
% LaTeX Template
%
% This template has been downloaded from:
% http://www.latextemplates.com
%
% Note:
% The \lipsum[#] commands throughout this template generate dummy text
% to fill the template out. These commands should all be removed when 
% writing essay content.
%
%%%%%%%%%%%%%%%%%%%%%%%%%%%%%%%%%%%%%%%%%

%----------------------------------------------------------------------------------------
%	PACKAGES AND OTHER DOCUMENT CONFIGURATIONS
%----------------------------------------------------------------------------------------

\documentclass[12pt]{article} % Default font size is 12pt, it can be changed here

\usepackage{geometry} % Required to change the page size to A4
\geometry{a4paper} % Set the page size to be A4 as opposed to the default US Letter

\usepackage{graphicx} % Required for including pictures

\usepackage{float} % Allows putting an [H] in \begin{figure} to specify the exact location of the figure
\usepackage{wrapfig} % Allows in-line images such as the example fish picture

\usepackage{lipsum} % Used for inserting dummy 'Lorem ipsum' text into the template

\usepackage{url}

\linespread{1.2} % Line spacing

%\setlength\parindent{0pt} % Uncomment to remove all indentation from paragraphs

\newcommand{\figref}[1]{Figure~\ref{#1}}
\graphicspath{{./figures/}} % Specifies the directory where pictures are stored

\begin{document}

%----------------------------------------------------------------------------------------
%	TITLE PAGE
%----------------------------------------------------------------------------------------

\begin{titlepage}

\newcommand{\HRule}{\rule{\linewidth}{0.5mm}} % Defines a new command for the horizontal lines, change thickness here

\center % Center everything on the page

\HRule \\[0.4cm]
{ \huge \bfseries bfs experiment progress}\\[0.4cm] % Title of your document
\HRule \\[1.5cm]

\begin{minipage}{0.4\textwidth}
    \centering {Cheng Liu} \\ 
    \vspace{2em}
    \centering {\large \today}
\end{minipage}

\end{titlepage}

%----------------------------------------------------------------------------------------
%	TABLE OF CONTENTS
%----------------------------------------------------------------------------------------

%\tableofcontents % Include a table of contents

%\newpage % Begins the essay on a new page instead of on the same page as the table of contents 

%----------------------------------------------------------------------------------------
%	INTRODUCTION
%----------------------------------------------------------------------------------------
\section{bfs optimizations} % Major section
In order to accelerate bfs on FPGAs using OpenCL, we reorganize 
the bfs algorithm to a stream manner such that it can fit well 
with the hardware optimization.

Using the youtube dataset, a straightforward bfs takes 40s while a basic stream bfs 
takes 3.8s.

Since memory bandwidth is the bottleneck of the bfs performance, we developed a number of 
strategetic optimizations to the bfs accelerator to reduce the memory accesss and improve the 
memory bandwidth utilization.

\begin{center}
    \begin{tabular}{c|c|c}
        Optimization Strategies & Optimizaed Runtime & Baseline Runtime \\ \hline
        top-down & 3.8s & 3.8s \\ \hline 
        wide data width & bug & 3.8s \\ \hline
        customized data path & ? & 3.8s \\ \hline
        duplicate data path & ? & 3.8s \\ \hline
        cache hub vertex & 3.4s & 3.8s \\ \hline
        hw-sw codesign & 2.8s & 3.8s \\ \hline
        codesign + bottom up & 2.2s & 3.8s \\ \hline
        codesign + bottom up + pipeline & 1.6s & 3.8s \\ \hline
        codesign + top down + pipeline & 1.2s (bug) & 3.8s \\ 
    \end{tabular}
\end{center}





%----------------------------------------------------------------------------------------
%	BIBLIOGRAPHY
%----------------------------------------------------------------------------------------
\bibliographystyle{plain}
\bibliography{refs}
%----------------------------------------------------------------------------------------

\end{document}
