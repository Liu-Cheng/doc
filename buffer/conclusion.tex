\section{Limitations and Future Work} \label{sec:limitations}
Despite the promising advantages of the proposed customization framework, there 
are a few limitations that must be acknowledged and hopefully addressed 
in future work.

First of all, the proposed design framework mainly targets compute intensive 
nested loops. As such, it is not a generic method to perform HLS on random logic.

Secondly, the grouping strategy used in the framework makes the IO buffer partitioning 
difficult, because the input/output data of the DFGs in the same group may be located in
diverse partitioned banks and the DFG scheduling may not be reused among these DFGs. 
This limitation will be fixed by introducing an additional buffering stage in future.
Currently, input/output buffer partitioning is not supported in this framework and the 
accelerators developed only have a single input buffer and a single output buffer. 

Thirdly, data transmission and the computation are sequentially repeated to complete 
the nested loop on the accelerator. However, it is possible to pipeline the two 
processing stages for better performance while performance models need to be 
adjusted accordingly.

\section{conclusion} \label{sec:conclusion}
In this work, we have presented an automatic nested loop acceleration 
framework based on a homogeneous SCGRA overlay built on top of 
off-the-shelf FPGA devices. Given high level user constraints and design goals, 
the framework performs an intensive system customization specifically 
to a nested loop. When the optimized design configuration is 
acquired after the customization, the HDL model of the resulting accelerator  
as well as the communication interface can be generated. Finally, the accelerator 
and the software running on the host CPU can further be compiled rapidly to 
the hybrid CPU-FPGA system.

Particularly, by taking advantage of the regularity of 
the SCGRA overlay, various metrics of the accelerator such as overhead 
and energy consumption can be effectively evaluated via analytical models. 
This unique characteristic allows us to reduce the complex system 
customization to a simpler sub design space exploration and a 
straightforward search over the sub design space. Compared to an 
exhaustive search through the entire design space, the proposed 
two-step customization method reduces runtime by 2 orders of magnitude 
on average while achieving quite similar energy-performance Pareto-optimal 
curve and the same customized architecture that produces 
minimum energy consumption. When compared to the implementations using 
a state-of-the-art HLS tool, the customized design exhibits competitive 
performance. 


