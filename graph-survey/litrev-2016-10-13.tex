\section{Gunrock}
Gunrock \cite{wang2016gunrock} is a data-centric graph processing framework which mainly handles
graphs that can be expressed with iterative convergent processes. It is also
based on the BSP model. Basically, the graph
problems are divided into steps and each step must be synchronized.
Advance-filter-compute are the typical primitive steps used in Gunrock. 
These operations are performed on top of graph frontiers which is used in many
graph processing framework. Major optimization strategies used in Gunrock are
listed below.

\begin{itemize}
    \item Kernel fusion: Inspired by the primitive GPU optimization strategy
        which leverages the producer-consumer locality between operations,
        Gunrock tries to integrate advance, filter and user-specific functor
        into a thread to improve performance as well as memory access
        efficiency.
    \item Workload balance: Gunrock's advance step which is applied on graph
        frontier result in severe load balancing problem especially for graphs
        with power-law distribution just as any other similar framework. To
        solve this problem, Gunrock roughly divides the vertices in the
        frontier into three groups based on the size of the associated neighbor
        lists. Vertices with larger size of neighbor lists will be executed in
        parallel in a cooperative thread array (CTA). Vertices with medium size
        will be executed in parallel in a warp. Vertices with small size will be
        executed in separate threads. When multiple vertices are assigned to a
        CTA or warp, the vertex with largest size of associated neighbor list
        will be distributed to all the threads of the CTA or warp and be executed
        first. Then all the vertices will be executed sequentially with the same
        logic.
    \item Idempotent vs. non-idempotent operations: As vertices in current
        frontier may share the same neighbors, there are duplicate vertices when
        producing the next frontier. Gunrock removes some of the duplicate
        vertices with inexpensive heuristics for applications that allows
        duplicated vertices in frontier. This strategy helps to improve
        performance. For applications that can't tolerate duplicated vertices,
        it can also remove duplicate vertices completely. 

    \item Pull and push traversal: This is the same with the top-down
        and bottom-up traversal strategy used in Ligra.

    \item Priority queue: Usually all the vertices in the frontier are treated
        equally in BSP model while Gunrock divides them into two queues
        based on a criterion to save work.  

\end{itemize}

\section{Medusa}
The goal of Medusa framework in the first place is to ease the general graph applications on GPUs.
Here are the major contributions of Medusa.
\begin{itemize}
    \item Edge-Message-Vertex (EMV) model: it provides 6 APIs which used for fine-grained edge and vertex processing.
    \item Memory optimization: Graph aware buffer scheme which avoids the message grouping overhead. 
    \item Multi-GPU support: With vertex and edge replication, the graph is equally partitioned. Multi-hop replication is also discussed. 
\end{itemize}

\section{Dynamic Graph Processing}
Problems of processing dynamic graph with its continuous updates in real-time manner
\begin{itemize}
\item Graph storage: Modification of graph structure is costly
\item Fast response: Widely used global manner cannot achieve real-time processing
\item Workload imbalance: Some vertices update more frequently and consume more computing resource in a certain period of time.
\end{itemize}

Ideas:
\begin{itemize}
\item Random-access graph structure: Hash-based graph partition strategy to enable fine-grained graph updates
\item Incremental graph processing: Vertex-based incremental graph computing model
\item Workload rebalance: Detect hotspots and evaluate their workload, and then rebalance them with greedy algorithms.
\end{itemize}
