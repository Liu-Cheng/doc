\section{Ligra}
Ligra \cite{Shun2013ligra} is a light-weight graph processing framework
targeting shared-memory multi-core processing system. It simplifies graph
traversal algorithm implementation on top of a few parallel libraries including
cilkplus and openmp. Given a graph traversal algorithm, the user only needs to
provide the implementation of a few half-done routines (i.e. Algorithm specific
compute routine, F routines) and the framework will handle the rest. On top of
the ease of the parallel graph algorithm development, this work contributes on
the following aspects:
\begin{itemize}
    \item It uses vertexSubset to support the frontier based graph traversal. 
    \item It supports both edgeMap and vertexMap to meet requirements of
        different algorithms.
    \item It allows both graph traversal with a bottom-up and top-down approach
        for the sake of performance. Basically, iterations with larger frontier
        will prefer the bottom-up approach while the rest will adopt the
        top-down approach. 
    \item Edge/vertex representations including sparse and dense are both
        supported.
\end{itemize}

Communication and synchronization relies on the atomic operation
(compare-and-swap) over the shared memory. 

\section{Ligra+}
This work \cite{Shun2015ligra+} is an extension of the Ligra framework. It
focuses on the graph compression on top of Ligra.

Inspired by the byte compression used in sparse matrix-vector multiplication,
this work proposed a run-length byte encoding method. The basic idea is to
encode the edges with variable length instead of the fixed byte block used in
previous work. To support the run-length encoding, a head byte is used to
specify the lengthy of each encoding block as well as other controlling
information. The overall encoding roughly consists of three steps. 1) Edges
associated with each vertex are
sorted. 2) Difference encoding is applied on the sorted edges. 3) run-length
encoding is further used. Due to the variation of edge numbers associated on
each vertex, only verteies with large associated edges are encoded. Then the encoding
and decoding are further parallelized. 
