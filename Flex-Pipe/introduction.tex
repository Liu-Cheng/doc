\section{Introduction}\label{sec:introduction}
The use of FPGAs as compute accelerators has been demonstrated by numerous researchers as an effective solution to meet the performance requirement across many application domains. However, FPGA accelerator design using the conventional HDL based design flow suffers extremely low design productivity due to the low-level design entry, lengthy compilation and poor design reuse, especially compared to the corresponding software solution. 

Overlay which can be parametric HDL model, pre-synthesized or pre-implemented circuits promises to address the above challenges. It raises the abstraction level by using either high level language or data flow graph (DFG) as the design entry both of which are closer to software design. As the configuration granularity of an overlay is usually coarser, the compilation process is simplified and the compilation time can be reduced. Moreover, it can be resued easily across the FPGA devices and parts, which is essentially beneficial to FPGA accelerator design productivity.

In this work, an SCGRA is built as an FPGA overlay. Using this overlay, QuickDough a rapid FPGA accelerator design methodology is proposed. It translates compute kernel to DFG and then statically schedules the DFG to the target SCGRA. Finally, it integrates the scheduling result with a pre-implemented bitstream and accomplishes the accelerator design. 

With the SCGRA overlay, QuickDough has the lengthy hardware compilation reduced to an operation scheduling problem. Although the design and implementation of the SCRGA based accelerator must rely on the conventional hardware design flow, only one instance of the SCGRA implementation is required per application or application domain through the design iterations. In addition, the proposed SCGRA has quite regular hardware structure, and it scales well on both the implementation frequency and execution time in cycles, which are essential to the end-to-end run time. According to the experiments on Zedboard, FPGA accelerator using QuickDough achieves competitive performance compared to that using direct HLS, while the SCGRA does consume more hardware overhead especially the BRAM blocks. 

In \secref{sec:framework}, the proposed FPGA acceleration design methodology QuickDough will be elaborated. The SCGRA implementation and compilation will be presented in \secref{sec:scgraimplement} and \secref{sec:scgracompile} respectively. Experimental results are shown in \secref{sec:experiments}. Finally, we will discuss the limitations of current implementation in \secref{sec:discussion} and conclude in \secref{sec:conclusions}.

