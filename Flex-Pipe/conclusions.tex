\section{Limitations and Future Work}\label{sec:discussion}
While the current implementation of QuickDough has demonstrated promising initial results, there are a number of limitations that must be acknowledged and possibly addressed in future work.

First and foremost, the proposed method is designed to synthesize parallel compute kernels to execute on FPGAs only. As such, it is not a generic method to perform HLS on random logic. Moreover, the proposed method is intended to serve as part of a larger HW/SW synthesis framework that targets hybrid CPU-FPGA systems. Therefore, many high-level design decisions such as the identification of compute kernel to offload to FPGAs are not handled in this work. 

Secondly, the DFG is still manually generated, and a general front-end compilation that could transform high level language program kernel to DFG is still missing. Thridly, we just specify two SGCRA configurations for all the benchmark, while it is difficult for a high-level software designer to figure out an appropriate SCGRA configuration. An SCGRA optimizer will be developed to perform the SCGRA customization automatically in future. Finally, the capacity of the address buffers used in the accelerator limits the block size that can be adopted to the FPGA in a few cases. However, there are a large number of invalid address entries in it and this will be fixed in future. 

\section{Conclusions}\label{sec:conclusions}
In this paper, we have proposed QuickDough, an SCGRA overlay based FPGA accelerator design method, to compile compute intensive applications to a CPU+FPGA system. With the SCGRA overlay, the lengthy low-level implementation tool flow is reduced to a rapid operation scheduling problem. The compilation time from high level language application to the CPU+FPGA system is reduced by around two magnitudes, which contributes directly into higher application designers' productivity.

Despite the use of an additional layer of SCGRA on the target FPGA, the overall application performance is not necessarily compromised. Implementation with higher clock frequency resulting from the highly regular structure of the SCGRA, in combination with an in-house scheduler that can effectively schedule operations to overlap with pipeline latencies provides competitive performance compared to that using a commerical HLS based design method.
