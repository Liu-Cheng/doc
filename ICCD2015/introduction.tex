\section{Introduction}\label{sec:introduction}
Recent years have witnessed a tremendous growth in the use of accelerators in computer systems to improve the systems' performance and power-efficiency. Among these accelerators, GPUs and the Xeon Phi accelerators have stood out as two of the most popular choices in spite of their relatively short history as accelerators --- 5 of the top 10 systems on the top500 list take advantage of them. On the other hand, despite the long and successful track record of FPGA accelerators \cite{SAT04Survey, Biocomp10Survey, bing2014FPGA}, the use of FPGA accelerators in main-stream systems remains limited and has yet to receive widespread adoption beyond highly skilled hardware engineers. 

We argue that there are 3 major challenges faced by the software developers when using FPGAs as accelerators: (1) the unfamiliar hardware development methodology with low-level hardware description languages; (2) the lack of hardware-software support during run-time and compile-time; and (3) the lack of efficient implementation and debugging facilities.

To address challenge 1, recent advances in the high-level synthesis tools have significantly raised the abstraction level of the FPGA design, allowing users to effectively express hardware designs using familiar software languages such as C/C++, Java, Python and Scala \cite{VivadoHLS,legup}. At the same time, to address challenge 2, researchers have also explored various facilities to support mixed hardware-software designs in a unified language \cite{openCL,SDAccel} and run-time environment \cite{SoTECS:2008,Lubbers:2009,Ismail:2011}. Challenge 3, however, remains a major productivity hurdle to most software developers. Unlike compiling software programs, implementing a hardware design on to an FPGA using standard hardware design tools can take upward of days with some of the largest designs. This disproportionally long run time greatly limits the number of compile-debug-edit cycles per day and hinders the designer's productivity.

The focus of this work is therefore to address challenge 3. In particular, we are interested in significantly improving the speed of generating hardware accelerators on FPGAs for compute intensive loops expressed in high-level languages, while maintaining a competitive overall performance for the resulting system.

To that end, we have developed QuickDough, a design framework that rapidly generates loop accelerators and their associated software-hardware communication interfaces. By using a soft coarse-grain reconfigurable array (SCGRA) \emph{overlay} as an intermediate architecture implemented on top of the physical FPGAs, QuickDough partitions the complex accelerator development flow into two paths. Along the rapid and common path, QuickDough generates loop accelerators by selecting an overlay configuration from a library of partially implemented FPGA bitstream, schedules the compute operations from the user-provided loop onto the overlay, and finally updates the bitstream configuring the target FPGA. By employing different configuration selection algorithms, QuickDough allows users to perform tradeoff between performance and compilation time. At the end of the selection process, optimized communication interfaces will be produced accordingly. Meanwhile, QuickDough also includes a relatively slow yet less frequent path which pre-build an overlay based accelerator library targeting a group of applications. To expedite the library generation process, a small representative set of accelerator configurations are chosen as the library and generated automatically using a template based system.

% Experiments show that despite the use of an overlay, the accelerators generated by QuickDough 
% has promising performance speedup over the software executed on a hard ARM processor, yet 
% the accelerator generation completes in seconds achieving near-software compilation speed 
% and enhancing software designers' productivity in designing, developing and debugging 
% their applications with accelerators. 

Experiments show that QuickDough is able to implement the entire hardware-software accelerator system in the order of seconds, a speed compatible with the expectation of most software designers. Yet despite this fast compilation process, the resulting accelerators produced remain competitive in performance, offering up to $9\times$ speedup over the baseline software implementations in our benchmark applications.

In next section, we will elaborate on the FPGA loop accelerator design framework -- QuickDough. Then we will present the experimental results in \secref{sec:experiments}. Finally, we will compare QuickDough with related works in \secref{sec:relatedwork} and conclude in \secref{sec:conclusions}.


