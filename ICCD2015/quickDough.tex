\pdfminorversion=4
\documentclass[conference]{IEEEtran}
\usepackage[binary-units]{siunitx}
\usepackage{enumitem}
\usepackage{graphicx}
\usepackage{multirow}
\usepackage[caption=false,font=footnotesize]{subfig}
\usepackage{dblfloatfix}
\usepackage{xspace}
\usepackage{url}
%\usepackage[square, comma, sort&compress, numbers]{natbib}
\usepackage{algorithm}
\usepackage{algorithmic}
\usepackage[algo2e]{algorithm2e}
\usepackage{amsmath}
\usepackage[table]{xcolor}


% \usepackage{draftwatermark}%
% \SetWatermarkFontSize{20cm}%
% \SetWatermarkScale{2}%
% \SetWatermarkText{DRAFT Do not distribute}%

\setlength{\parskip}{0pt}
\renewcommand{\topfraction}{1}
\renewcommand{\textfraction}{0.15}
\renewcommand{\dbltopfraction}{1}

\renewcommand\floatpagefraction{.9}
\renewcommand\topfraction{.9}
\renewcommand\bottomfraction{.9}
\renewcommand\dbltopfraction{.9}
\renewcommand\textfraction{.1}   
\setcounter{totalnumber}{4}
\setcounter{topnumber}{8}
\setcounter{bottomnumber}{4}
\setcounter{dbltopnumber}{4}

\newcommand*{\Scale}[2][4]{\scalebox{#1}{$#2$}}
\newcommand{\eqnref}[1]{Equation~\ref{#1}}
%\newcommand{\eqnref}[1]{(\ref{#1})}
\newcommand{\figref}[1]{Figure~\ref{#1}}
\newcommand{\algref}[1]{Algorithm~\ref{#1}}
\newcommand{\secref}[1]{Section~\ref{#1}}
\newcommand{\tabref}[1]{Table~\ref{#1}}
\newcommand{\code}[1]{\texttt{#1}}
\newcommand{\tabincell}[2]{\begin{tabular}{@{}#1@{}}#2\end{tabular}}
\renewcommand{\algorithmcfname}{ALGORITHM}
\SetAlFnt{\small}
\SetAlCapFnt{\small}
\SetAlCapNameFnt{\small}
\SetAlCapHSkip{0pt}
\IncMargin{-\parindent}



\graphicspath{{./figures/}}


\begin{document}
\bstctlcite{IEEEexample:BSTcontrol}
%
% paper title
% can use linebreaks \\ within to get better formatting as desired
\title{QuickDough: A Rapid FPGA Loop Accelerator Design Framework Using Soft CGRA Overlay}


% author names and affiliations
% use a multiple column layout for up to three different
% affiliations

%% Original
% \author{\IEEEauthorblockN{Cheng Liu, Colin Yu Lin, Hayden Kwok-Hay So}
% \IEEEauthorblockA{Department of Electrical and Electronic Engineering\\
% The University of Hong Kong\\
% \{liucheng,linyu,hso\}@eee.hku.hk}}

%% Blind
\author{\IEEEauthorblockN{}
\IEEEauthorblockA{}}


% conference papers do not typically use \thanks and this command
% is locked out in conference mode. If really needed, such as for
% the acknowledgment of grants, issue a \IEEEoverridecommandlockouts
% after \documentclass

% for over three affiliations, or if they all won't fit within the width
% of the page, use this alternative format:
% 
%\author{\IEEEauthorblockN{Michael Shell\IEEEauthorrefmark{1},
%Homer Simpson\IEEEauthorrefmark{2},
%James Kirk\IEEEauthorrefmark{3}, 
%Montgomery Scott\IEEEauthorrefmark{3} and
%Eldon Tyrell\IEEEauthorrefmark{4}}
%\IEEEauthorblockA{\IEEEauthorrefmark{1}School of Electrical and Computer Engineering\\
%Georgia Institute of Technology,
%Atlanta, Georgia 30332--0250\\ Email: see http://www.michaelshell.org/contact.html}
%\IEEEauthorblockA{\IEEEauthorrefmark{2}Twentieth Century Fox, Springfield, USA\\
%Email: homer@thesimpsons.com}
%\IEEEauthorblockA{\IEEEauthorrefmark{3}Starfleet Academy, San Francisco, California 96678-2391\\
%Telephone: (800) 555--1212, Fax: (888) 555--1212}
%\IEEEauthorblockA{\IEEEauthorrefmark{4}Tyrell Inc., 123 Replicant Street, Los Angeles, California 90210--4321}}

% make the title area
\maketitle

\begin{abstract}
The QuickDough design framework is presented as a way to address productivity issues of developing
high-performance FPGA accelerators. QuickDough utilizes a soft coarse-grained reconfigurable array
(SCGRA) as an overlay on top of off-the-shelf FPGAs for rapid accelerator developments.
Instead of compiling high-level applications directly to HDL circuits, the compilation step is
reduced to a simpler operation scheduling task targeting the SCGRA overlay, significantly reducing
compilation time and increasing possible numbers of debug-cycle-per-day as a result.
The softness of the SCGRA allows highly customized application-specific design while the regular
structure of the SCGRA makes the customization much easier. When compared to the
execution on a general purposed processor, the accelerators generated using QuickDough achieves up
to 9X performance speedup.

\end{abstract}

\section{Introduction}
Offloading compute intensive nested loops to FPGA accelerators has 
been demonstrated as an effective way of performance 
acceleration across various application domains\cite{Chung2010}. 
However, the design productivity of developing such accelerators 
remains relatively low and it has become a major obstacle that 
hinders the wide adoption of FPGAs as compute engines. Although the use of 
high level synthesis (HLS) tools which allow the application designers to 
focus on high level functionality instead of low-level implementation details alleviates 
this shortcoming \cite{cong2011high}, the lengthy low-level FPGA implementation process 
greatly limits the number of compile-debug-edit cycles per day and dramatically 
affects the overall design productivity. 

To approach the above design productivity problem, 
researchers have recently turned to the use of virtual FPGA overlay 
architectures \cite{Grant2011Malibu,ZUMA2012,mesh-FUs,
ferreira2011fpga, kissler2006dynamically,scgra}. When combined properly to 
high level compilation tools, the overlay architecture based design methods 
are able to produce high-performance accelerators at near software 
compilation speed, but at the cost of hardware overhead, power and even performance.
By customizing the architectures of these \emph{virtual} 
overlays for a target user design, in theory, it is 
possible to significantly improve the performance-energy of the 
resulting accelerator. In practice, however, navigating through a 
labyrinth of architectural and compilation parameters to fine-tune 
an accelerator's performance-energy is a slow and non-trivial process. 
To require a user to manually explore such vast design space is going 
to counteract the productivity benefit of the utilizing overlay 
in the first place.

To obtain both high design productivity and advantages of 
application-specific customization, we have developed a 
soft CGRA (SCGRA) overlay based nested loop acceleration design 
framework. This framework targets a hybrid CPU-FPGA computing system where 
nested loop compute kernels expressed in high-level languages are compiled and 
executed on the SCGRA overlay built on top of FPGAs while the rest of the 
user application remains running on the host CPU. Given high-level design 
goals and design constraints, the framework automatically explores the 
design space and customizes architectural parameters specifically to the 
user application. In addition, the framework also exploits loop unrolling 
and hardware-software communication strategies in combination 
with buffer sizing and partition as performance enhancing techniques.
Once the design goals and constraints are fulfilled, the 
corresponding hardware accelerator and communication interface 
are generated and both the hardware accelerator and software 
are compiled to the hybrid CPU-FPGA system.  

As demonstrated in previous work, both the compilation from nested 
loops to the SCGRA overlay \cite{scgra} and the SCGRA overlay 
implementation \cite{ROB2014} are fast. Meanwhile, the SCGRA 
overlay is highly pipelined and has quite regular tiling structure, 
which makes the hardware overhead, power consumption and even implementation frequency 
highly predictable. Therefore, a multitude of design metrics such as performance 
and energy consumption can be accurately estimated using analytical models when the 
overlay scheduling result is available. And the nested loop specific acceleration problem can be 
reduced to a sub design space exploration centering an NP-complete SCGRA scheduling and 
a following customization with all the potential configurations well estimated. 
While the overlay scheduling depends on much less design parameters, the overall 
customization can be dramatically simplified. Accordingly, the overall design 
framework achieves both rapid compilation and fast application specific 
customization and ensures high design productivity and high performance of the resulting
accelerators at the same time.  

We performed a series of experiments to evaluate the efficiency 
and quality of the proposed design framework using a real-world 
benchmark. Compared to an exhaustive search, the proposed 
customization achieves similar results while reducing its 
runtime by 2 orders of magnitude on average. When compared to 
HLS implementations with moderate manual optimizations that can 
reasonably be expected from a novice user,  
the customized accelerators produced using the proposed framework 
has demonstrated competitive performance as well. 

With that, we consider the main contribution of this work is in the following areas:
\begin{itemize}[nosep]
\item We have developed a rapid customization framework that 
    performs automatic design parameter tuning for SCGRA overlay based 
    nested loop acceleration on a hybrid CPU-FPGA computing system. 
    The result is comparable to an exhaustive design space search while 
    it runs at a fraction of time.
\item We have developed a parametric regular SCGRA overlay template. It can be used 
    to generate FPGA accelerators with predictable implementation 
    frequency, hardware overhead and power consumption, which is essential to 
    both the rapid compilation and customization.
\item We have developed a hierarchy on-chip buffer. It allows flexible 
    buffer partition and makes good use of the efficient lock-step computation 
    of the SCGRA overlay.
\end{itemize}

In \secref{sec:relatedwork}, related work is briefly introduced. 
The overall automatic nested loop acceleration framework is illustrated 
in \secref{sec:acc-framework}. Then SCGRA overlay based FPGA accelerator 
is illustrated in \secref{sec:scgra} and the application-specific customization 
method is further detailed in \secref{sec:customization-method}. 
Experimental results are presented in \secref{sec:result} and limitations are 
discussed in \secref{sec:limitations}. Finally, the paper is 
concluded in \secref{sec:conclusion}.



\section{CNN accelerator overclocking} \label{sec:framework}
Overclocking can boot the clock frequency of CNN accelerators and is potentially beneficial to both 
the performance and energy efficiency. However, the timing violation may lead to distinct errors which 
may degrade the neural network prediction accuracy or even affect the functionality of the accelerators. 
To apply overclocking on CNN accelerators, we must handle all the possible 
problems incurred by the timing violation. 

\subsection{Overclocking overview}
Based on the intensity of overclocking and timing 
violation, we divide the overclocking incurred errors into different 
categories so that corresponding design methods can be used to 
alleviate the resulting problems efficiently. While it is difficult 
to precisely quantize the timing violation directly at runtime, we use 
the neural network prediction accuracy loss as the main 
classification metric. When there is only minor prediction accuracy 
loss which is less than 1\%, the penalty is usually acceptable 
and nothing needs to be done. When the prediction accuracy loss 
ranges from 1\% to 10\%, the moderate accuracy loss must be 
alleviated. When there is sever timing violation and 
accuracy loss, there is usually little chance to recover 
without improving the timing and the status of the accelerator 
may not be steady. It may result in considerable computing errors 
or even system stall due to critical control signal faults.  

As the critical paths may change at runtime due to the external 
environment variations such as temperature, the accuracy loss status may 
change as well. Typically, we expect that the lower overclocking setup 
may also lead to severe computing errors with relatively lower probability.
Fig \ref{fig:dynamic-loss} shows the possible accuracy loss status 
transition graph. It indicates that we should take the worst case 
into consideration even when the overclocking setup is relatively conservative.
\begin{figure}
	\center{\includegraphics[width=0.75\linewidth]{overview}}
    \caption{CNN accelerator status transition graph.}
\label{fig:loss-estimation}
\vspace{-1em}
\end{figure}

For the minor accuracy loss case, overclocking can be used directly. 
We will not dwell on it. For the moderate accuracy loss, the neural network 
can still be salvaged. As the neural network models are usually obtained 
from offline training on general purposed processors (GPPs) which assumes the 
neural networks to be executed on an equivalent computing device, 
the computing on overclocked CNN accelerator varies and does not match 
with the assumption. To address this problem, we have the neural network models 
retrained on the overclocked accelerator so that the computing variation can be 
tolerated by the retrained models. By getting rid of the computing difference between 
inference and training, the prediction accuracy can be improved.

For the severe accuracy loss situation, the design can be hardly salvaged 
due to the relatively large amount of timing violations. In this case, 
we take advantage of the FPGA reconfiguration 
capability to reload an implementation with lower clock frequency. 
In some extreme case, the accelerator may even hang up. This can also be 
resolved with FPGA reconfiguration. However, the challenge is to 
detect the situation timely and recover the system automatically.

In combination with the possible accelerator status transfer graph and 
the related optimization approach, we propose an runtime overclocking 
management mechanism that ensures both efficient and resilient 
neural network computing. The mechanism is shown in Fig \ref{fig:runtime-management}.
\begin{figure}
	\center{\includegraphics[width=0.45\linewidth]{manegment}}
    \caption{Runtime overclocking management.}
\label{fig:runtime-management}
\vspace{-1em}
\end{figure}

\subsection{Techniques for mitigating overclocking errors}
On top of the runtime overclocking management, we still need a series of techniques 
as mentioned in the above section to ensure resilient neural network processing 
on overclocked CNN accelerators. They will be detailed in this section.

\subsubsection{Accuracy loss estimation}
As illustrated in this section, typically a user may choose either the overclocking 
strategy with minor accuracy loss or moderate accuracy loss by setting up 
different clock frequency. Nevertheless, the critical paths 
may change at runtime with certain probability and the actual status of the CNN accelerator 
can be dynamic. Thus, it is of vital importance to detect the status of the CNN 
accelerator at runtime and activate the corresponding design strategy.

In order to determine the status of the overclocked CNN accelerators, 
we propose a runtime prediction accuracy loss estimation mechanism 
as shown in Fig \ref{fig:loss-estimation}. The basic idea is to insert a set 
of reference data to the input data stream. The actual prediction 
accuracy of the reference data can be computed in advance. When the 
reference data are processed on the overclocked CNN accelerator, the 
prediction accuracy of the refernce data can be obtained. By comparing the 
golden accuracy and the measured accuracy, we can calculate the accuracy loss and 
determine the status of the CNN accelerators accordingly. When the status of 
the accelerator is obtained, the corresponding optimization approach 
can be invoked. In addition, the amount of reference data inserted to 
the input data can be adjusted to ensure that the overhead is acceptable.

\begin{figure}
	\center{\includegraphics[width=0.85\linewidth]{loss_checkpoint}}
    \caption{Runtime neural network accuracy loss estimation.}
\label{fig:loss-estimation}
\vspace{-1em}
\end{figure}

On top of accuracy loss, the accelerator may hang up when the critical signals 
do not function as expected. Although a heart-beat mechansim may resolve this problem, 
it requires additional hardware design. To avoid introducing additional hardware, 
we use a simple timeout mechanism to determine whether the accelerator gets stuck. 

\subsubsection{On-accelerator retraining}
When there is moderate prediction accuracy loss due to the computing errors, 
we try to retrain the neural network model. The basic idea is to have both the 
applicaiton data and the computing variation patterns learned by the neural 
network model such that it can be less sensitive to the computing variations. 

Fig \ref{fig:retrain} shows the retraining framework built on top of Caffe.
The forward propagation that is usually performed on GPPs is transferred to 
the FPGA based CNN accelerator while the backward propagation remains the 
same. As the computing on the CNN accelerator is fixed point, the updated 
weight must be converted to fixed point before they are moved to FPGA 
for forward propagation. Accordingly, the result of the forward propagation 
needs to be converted to float point when it is sent to the host for backward 
propagation.

\begin{figure}
	\center{\includegraphics[width=0.85\linewidth]{retrain_framework}}
    \caption{On-accelerator neural network retraining framework.}
\label{fig:retrain}
\vspace{-1em}
\end{figure}


In this work, we utilized an OpenCL based CNN accelerator. The communication between 
host and the accelerator can be conveniently achieved with the OpenCL API. 
Nevertheless, many RTL based CNN accelerators can also be wrapped up with 
OpenCL API and reuse the retraining framework with minor modification.

\subsubsection{System reconfiguration and recovery}
When the accuracy loss gets too large, it indicates severe timing 
violations in the CNN accelerators which can be rather difficult to resolve with 
only fault-tolerant neural network models. Another extreme occasion is 
accelerator hangup when critical control signals are affected. In these cases, we 
opt to reconfigure the FPGA accelerator with a more conservative implementation.
However, when the errors are detected, previous computing may already suffer 
similar errors and many prediction can be wrong. To address this problem, we opt to 
utilize the checkpoint strategy to recover from previous checkpoint. 
Fig \ref{fig:checkpoint} shows the basic control flow of the checkpoint strategy.
Basically we divide the input data into blocks and some reference data 
will be added to the end of each block. When neural network accuracy measured with 
the reference data is abnormal, the processing will roll back to the last checkpoint 
which keeps the record of last input block index. Then we degrade the clock and 
new accelerator bitstream will be reloaded for the following inference.

%\begin{figure}
%	\center{\includegraphics[width=0.75\linewidth]{blank}}
%    \caption{Checkpoint strategy for neural network computing fault recovery.}
%\label{fig:checkpoint}
%\vspace{-1em}
%\end{figure}



\section{Experiments and results} \label{sec:result}

\subsection{Experiment Setup}
environment and benchmark applications

\subsection{Performance on Different Memory Models}

\subsection{Simulation Speed and Precision}


\section{Related Work}\label{sec:relatedwork}
Despite their promising performance advantage, the relatively low design productivity of developing
FPGA applications remains a major obstacle that hinders widespread adoption of FPGAs as commodity
computing devices. To address this problem, the design of QuickDough was inspired by the recent success in HLS tools.
It also took advantage of modern FPGAs' capabilities to allow for an additional overlay architecture
be employed for productivity sake.

\subsection{High-Level Synthesis}
To bridge the design productivity gap between software and hardware application development, many researchers have turned to the use of HLS techniques \cite{cong2011high}.
By raising the abstraction level of the physical hardware, HLS allows designers to express hardware designs using familiar high-level, software-like description languages such as C, Java, or Python \cite{cardoso2010compiling,Canis:2011:LHS:1950413.1950423}.
The low-level hardware implementations are then left to the tools to synthesize and optimize.
Indeed, with decades of research, some early results in HLS have already found their ways into FPGA vendors' commercial tools in recent years \cite{chen2005xpilot, zhang2008autopilot, VivadoHLS}.

Unfortunately, when considering the overall design productivity of developing hybrid software-gateware applications, the raised abstraction provided by HLS is only addressing part of the problem.
While the high-level abstraction makes expressing complex functionalities as FPGA gateware easier, the lengthy low-level compilation time spent in synthesis, mapping, placing and routing remains a bottleneck to the overall design productivity for an application designer.
Such long compilation time is particularly challenging for novice designers who are accustomed to the high speed of software compilation.
Most importantly, it is significantly impacting the possible compile-debug-edit cycle achievable per day by a designer, negatively impacting the productivity of the designer.

\subsection{Overlay Architectures}
To improve the speed of low-level implementation tools, researchers have explored various approaches over the past decades.
Inspired by application specific integrated circuit (ASIC) design flows, researchers and vendors have developed modular design flow and explored the use of pre-compiled hard macros \cite{lavin2010using,lavin2011} as implementation library.
In addition, researchers have also exploited the use of dynamic partial reconfiguration capabilities in FPGAs \cite{Frangieh2010} as a way to improve productivity.
In recent years, there has been an increased interest in applying the concept of \emph{overlay architectures} as a way to address this productivity challenge.  


An overlay architecture is a virtual intermediate architecture that is overlaid on top of the physical configurable fabric of an FPGA.  They are employed during the FPGA application implementation process for purposes such as to improve portability, security, and also productivity.
%Depending on the design goal, overlays have manifested in various forms, including HDL models, pre-synthesized or pre-implemented coarse-grained circuits, or even arrays of processing elements with various granularity. 

One of the most familiar categories of overlay consists of virtual FPGAs \cite{zuma2013carl,Grant2011Malibu,Coole2010Intermediate,Koch2013CI}. They are built either virtually or physically on top of off-the-shelf FPGA devices and typically feature coarser configuration granularity than the physical device.
Similar to virtual machines running on a typical computer, such virtual FPGA provides an additional layer that improves application portability and security.
Furthermore, because of the coarser-grained configurable fabric, implementing designs on such overlay is relatively easier than on a fine-grained device.
However, the additional layer imposes restrictions on the underlying fabrics' capability and usually results in moderate hardware overhead and timing degradation.

Another category of overlay architecture commonly employed is in the form of coarse-grained reconfigurable arrays (CGRAs).
The use of CGRAs provides unique advantages of performance especially for compute intensive applications as demonstrated by numerous ASIC CGRAs \cite{tessier2001reconfigurable} \cite{compton2002reconfigurable}.
Indeed, CGRAs on FPGA and ASIC have many similarities in terms of the scheduling algorithm and array structure.
However, they have quite different trade-offs in terms of configuration flexibility, overhead and performance.
In a nutshell, CGRAs on ASIC emphasize more on configuration capability to cover more applications, while FPGAs' inherent programmability greatly alleviates the concern.
Instead, CGRAs on FPGA may take advantage of the configurability of the underlying fabric to allow more intensive customization tailored to the target application.

The authors in \cite{kissler2006dynamically} developed WPPA (weakly programmable processor array), a VLIW architecture based parameterizable CGRA overlay. It featured an interconnection wrapper unit for each processing element (PE) that could be used for dynamic CGRAs topology customization. Unfortunately, programming and compilation on WPPA were not presented. The authors in \cite{ferreira2011fpga} proposed a heterogeneous CGRA overlay with a global multi-stage interconnection on FPGA. Compiling applications onto the overlay took only milliseconds for smaller DFGs. However, the global multi-stage interconnection required multiple stages for communication between each pair of PEs and resulted in either low implementation frequency or large communication latency in terms of cycles. In addition, there was no intermediate storage except the pipeline registers in the CGRA and it limited the performance of the operation scheduling.
In \cite{shukla2006quku}, a customized CGRA overlay called QUKU was developed for DSP algorithms. It had two-level configuration capability, while the low-speed configuration was used for operator reuse within an application and high-speed reconfiguration was used for optimization between different applications. Nevertheless, the hardware infrastructure was consist of simple operation elements which can only be adapted to a few specified DSP algorithms.
The authors in \cite{capalijia2013pipelined} built a more generic high speed mesh CGRA overlay using the elastic pipeline technique to achieve the maximum throughput. It adopted a data-driven execution flow and was suitable for smaller pipelined DFG execution, while it would be difficult to handle applications with random IO access. 

In general, previous CGRA overlays have demonstrated the promising performance acceleration capability for compute intensive applications. They typically take DFG as design entry and focus on hardware infrastructure design as well as corresponding mapping and scheduling. However, they are still lack of consideration on proper loop unrolling for DFG generation, on-chip buffering, the communication with host and even end-to-end performance which are essential for FPGA accelerator design especially from a HW/SW co-design engineer's perspective. 


Finally, a third category of overlay features soft-processor-like architectures with high degree of
control and data parallelism suitable for FPGA accelerations.  For example, in the work of MARC
\cite{Lebedev2010}, a many-core overlay with customizable data path was proposed.  Similarly, a
GPU-like overlay was proposed in \cite{Jeffrey2011potential}.


In this work, we opted to utilize a fully pipelined synchronous soft coarse-grained reconfigurable
array (SCGRA) as an overlay to facilitate rapid FPGA accelerator generation in a hybrid CPU-FPGA
system. Compared to previously proposed CGRAs, our overlay is designed to be \emph{soft} as the size,
processing element designs, as well as the interconnect topologies may all be customized as needed
providing just enough resource for an application specifically. Moreover, the design of our overlay
is regular and design parameters such as loop unrolling factor and overlay size have
relatively predictable influence on the overlay performance and overhead, which makes the
customization much easier and more efficient. Finally, it also takes advantage of the large number
of on-chip distributed memory on the FPGA for intermediate data storage and can handle large DFGs
with thousands of nodes. 

%On top of the above approaches, the use of \emph{overlays} in the form of HDL Model, pre-synthesized or pre-implemented coarse-grained reconfigurable circuits over the fine-grained FPGA devices, promises both to raise the abstraction level and reduce the compilation time.
%Recent years have seen a number of overlay designs being developed with granularities ranging from multi-processors to highly configurable logic arrays \cite{Lebedev2010,kissler2006dynamically,unnikrishnan2009application,Yiannacouras2009FPS,Guy2012VENICE,Jeffrey2011potential}. 

%% Not so much overlay, removed for clarity sake.
%Soft processors, which allow customization for target applications or application domains, have already been demonstrated to be efficient overlays on FPGA. A great number of work use embedded processors as FPGA overlays with micro-architecture parameters such as pipeline depth configurable \cite{Yiannacouras2007Exploration,microblaze,nios} and 


%instruction set architecture (ISA) customizable \cite{grad2009woolcano, }. 


%multi-processor overlay with both micro-architecture and interconnection customizable \cite{unnikrishnan2009application}, 

% vector processors overlay \cite{Guy2012VENICE,Yiannacouras2009FPS}



\section{Limitations and Future Work}\label{sec:discussion}
While the current implementation of QuickDough has demonstrated promising initial results, there are a number of limitations that must be acknowledged and possibly addressed in future work.

First and foremost, the proposed methodology is designed to synthesize parallel computing kernels to execute on FPGAs only. As such, it is not a generic methodology to perform HLS on random logic. Furthermore, the proposed method is intended to serve as part of a larger HW/SW synthesis framework that targets hybrid CPU-FPGA systems. Therefore, many high-level design decisions such as the identification of compute kernel to offload to FPGAs are not handled in this work. Also to guarantee the design productivity, a general front-end compilation that transforms high level language program kernel to DFG is still missing.

Currently, we just specify two SGCRA configurations for all the benchmark, while it is difficult for a high-level software designer to figure out an appropriate hardware configuration. An SCGRA optimizer will be developed to perform the SCGRA customization automatically in future.

Finally, the capacity of the address buffer used in QuickDough limits the block size that can be adapted to the FPGA in many cases. However, there are a large number of invalid address entries in it and this will be fixed in future. 

\section{Conclusions}\label{sec:conclusions}
In this paper, we have proposed QuickDough using a SCGRA overlay for compiling compute intensive applications to Zedboard. With the SCGRA overlay, the lengthy low-level implementation tool flow is reduced to a relatively rapid operation scheduling problem. The compilation time from an high level language application to the hybrid GPP+FPGA system is reduced by two magnitudes, which contributes directly into higher application designers' productivity.

Despite the use of an additional layer of SCGRA on the target FPGA, the overall application performance is not necessarily compromised. Implementation with higher clock frequency resulting from the highly regular structure of the SCGRA, in combination with an in-house scheduler that can effectively schedule operation to overlap with pipeline latencies provides competitive performance compared to a conventional HLS based design method.


\bibliographystyle{IEEEtran}
\bibliography{refs,ieeebstctl}

\end{document}

