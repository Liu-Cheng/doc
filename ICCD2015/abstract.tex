The use of CGRAs as compute accelerators has been 
demonstrated by numerous researchers as an effective solution 
to meet the performance requirement across many application domains. 
However, the design productivity of developing FPGA accelerators 
remains much lower compared to the use of a typical software 
development flow. Although the use of high-level synthesis (HLS) partly 
alleviates the shortcoming, the lengthy low-level FPGA implementation 
process including synthesis, placing and routing dramatically 
limits the number of compile-debug-edit cycles per day and 
hinders the widespread adoption of FPGAs. 

To address this design productivity problem, we have developed 
a rapid FPGA loop accelerator generation framework called 
QuickDough. By utilizing a soft coarse-grained reconfigurable 
array (SCGRA) overlay built on top of off-the-shelf FPGAs, it compiles 
an high-level loop to the overlay through a rapid operation 
scheduling first and then generates the FPGA accelerator bitstream through 
a rapid integration of the scheduling result and a pre-built 
overlay bitstream. According to the experiments, QuickDough is able to 
produce accelerators in the order of seconds while maintaining acceleration 
performance 1X-10X over the execution of the same software running 
on a hard ARM processor.  


