\section{Related Work}\label{sec:relatedwork}
Researchers have approached the challenge of lengthy hardware implementation tool run time from many angles.  Focusing on the low-level FPGA EDA tools, researchers have looked improving their run time by making quality-runtime trade-offs \cite{mulpuri2001runtime} and by parallelizing the tools themselves \cite{moctar2014parallel, goeders2011deterministic, altera-pc, 
xilinx-pc}.
Other researchers take advantages of the dynamic partial reconfiguration capabilities of modern FPGAs to shorten run time by effectively reducing the user design size \cite{Frangieh2010}.
Yet another group of researchers approach the problem from a higher level, innovating on how these tools are being used from a design methodology's point of view.  The use of modular design flow and by using pre-built hard macros \cite{lavin2013improving, korf2011automatic} have thus been explored.
While these approaches have significantly reduced the hardware implementation time, they remain at least 2 orders of magnitude slower when compared to a the software compilation experience. 

In recent years, there has been an increased interest in applying the concept 
of \emph{overlay architectures} as a way to address this productivity challenge. 
An overlay architecture is a virtual intermediate architecture that is overlaid 
on top of the physical configurable fabric of an FPGA. Overlays with different 
granularity ranging from virtual FPGAs \cite{zuma2013carl,Grant2011Malibu,
Coole2010Intermediate,Koch2013CI}, CGRA overlays \cite{kissler2006dynamically, 
ferreira2011fpga, shukla2006quku, capalijia2013pipelined, dsp2015cgra} to many-core processor arrays \cite{Lebedev2010, hannig2014invasive, boppu2014compact} and GPU-like overlays \cite{Jeffrey2011potential} 
have been developed. 

Among these overlays, CGRA overlays are particularly suitable for compute intensive loop acceleration as demonstrated by numerous prior works 
\cite{tessier2001reconfigurable,compton2002reconfigurable}.
A large number of CGRAs with different features have been developed and 
prototyped on FPGAs. A VLIW architecture based CGRA overlay was developed  
\cite{kissler2006dynamically}, which support dynamic topology customization. 
A heterogeneous CGRA overlay was proposed \cite{ferreira2011fpga} that utilized a global multi-stage interconnection to achieve topology customization 
to adapt to different applications. A customized CGRA overlay called 
QUKU \cite{shukla2006quku} was developed for DSP algorithms and it supported fast configuration for similarly to this work applications and slow configuration for distinct applications.
Finally High-speed CGRA overlays were built in \cite{capalijia2013pipelined} and \cite{dsp2015cgra} by using the elastic pipeline technique and smart DSP reuse respectively to achieve better performance and higher throughput.


Our proposed fully pipelined synchronous coarse-grained reconfigurable array overlay continues this
trend of exploiting coarse-grain reconfigurability to improve both design productivity and the
resulting accelerator performance. We particularly focus on using the overlay as the backbone of
FPGA loop accelerators targeting a hybrid CPU-FPGA computing system. With the pre-built library,
the loop accelerator can be generated in seconds. In addition, QuickDough pays particular attention
to the communication between the host processor and the accelerator and provides communication optimization
automatically, creating a seamless hardware-software co-design experience for the user.
Finally, our overlay was designed to be \emph{soft} from the beginning, featuring a template system
to allow for rapid overlay generation for either a single application or a domain of applications. 



