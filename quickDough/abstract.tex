FPGA accelerators used to offload compute intensive tasks of CPU offer both low power operation and great performance potential. However, they usually take much longer time to develop and are difficult to reuse, especially compared to the corresponding software design. Although the use of high-level synthesis (HLS) tools may partly alleviate this shortcoming, the lengthy low-level FPGA implementation process remains a major obstacle that limits the design productivity. To overcome this challenge, QuickDough, a rapid FPGA accelerator design method which utilizes soft coarse-grained reconfigurable arrays (SCGRAs) as an overlay on top of FPGA, is presented. Instead of compiling high-level applications directly as circuits implemented on the FPGA, the compilation process is reduced to an operation scheduling task targeting the SCGRA. Furthermore, the softness of the SCGRA allows domain-specific design of the processing elements, while allowing highly optimized SCGRA array be developed by a separate hardware design team. When compared to commercial high-level synthesis tools, QuickDough achieves competitive speedup in the end-to-end run time and reduces the compilation time by two orders of magnitude. 
