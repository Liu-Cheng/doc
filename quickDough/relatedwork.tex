\section{Related Work}\label{sec:relatedwork}
To improve the productivity of FPGA designers, researchers have approached the problem both by increasing the abstraction level and reducing the compilation time.

In the first case, decades of research in FPGA high-level synthesis have already demonstrated their indispensible role in promoting FPGA design productivity \cite{cong2011high}. Numerous design languages and environments \cite{cardoso2010compiling} have been developed to allow designers to focus on high-level functionality instead of low-level implementation details. While high-level abstraction may help a designers express the desired functionality, the low-level compilation time spent on synthesis, mapping, placing and routing for FPGAs remains a major hindrances to designs' productivity. Researchers have approached the problem from many angles, such as through the use of pre-compiled hard macros \cite{lavin2011} in the tool flow, the use of a partial reconfiguration, and modular design flow \cite{Frangieh2010}. 

On top of the above approaches, overlays, which can be parametric HDL Model, pre-synthesized or pre-implemented coarse-grained reconfigurable circuits over the fine-grained FPGA devices, promise both to raise the abstraction level and reduce the compilation time. Thus great research efforts have been attracted over the years and a number of overlays have been proposed \cite{Lebedev2010,kissler2006dynamically,unnikrishnan2009application,Yiannacouras2009FPS,Guy2012VENICE,Jeffrey2011potential}. The granularity of these overlays ranges from multi-processors to highly configurable logic arrays. 

Soft processors, which allow customization from various angles for target applications or application domains, have already been demonstrated to be efficient overlays for implementing an application on FPGA. \cite{Yiannacouras2007Exploration},\cite{microblaze}, and \cite{nios} employ general processors as the overlay and mainly have micro-architecture parameters such as pipeline depth configurable. \cite{grad2009woolcano} uses general processor with custom instruction extension as overlay, but hardware implementation is required whenever new custom instructions are added. \cite{Kock2013CI} develops a fine-grain virtual FPGA overlay specially for custom instruction extension, which makes the custom instruction implementation portable and fast. \cite{Lebedev2010} has customized data path on a many-core overlay, it could support both the coarse-grain multi-thread parallelism and data-flow style fine-grain threading parallelism. \cite{unnikrishnan2009application} adopts a multi-processor overlay with both micro-architecture and interconnection customizable. \cite{Guy2012VENICE} and \cite {Yiannacouras2009FPS} develop reconfigurable vector processors as the FPGA overlay to cover larger domains of applications. \cite{Jeffrey2011potential} presented a GPU-Like overlay for portability and it could explore both the data-level parallelism and thread level parallelism. These processor level overlays typically approach the customization through instruction set extension or micro-architecture parameters tuning, and the application developers don't need much interaction with the low level hardware customization. Thus an application can be implemented rapidly, while the penalty is the hardware overhead and implementation frequency.    

\cite{zuma2013carl} and \cite{Grant2011Malibu} build fine-grain components and mixed-grain components as a virtual FPGA overlay over the off-the-shelf FPGA devices. The virtual FPGAs allow the designers to reuse the virtual bitstream which is compatible across different FPGA vendors and parts. Particularly, the virtual FPGA with coarse granularity of components could also decrease the compilation time. \cite{Coole2010Intermediate} developed a family of intermediate fabrics which fits well for data parallel circuit implementation and the compilation time is almost comparable to software compilation. Apparently, the virtual FPGA overlays are beneficial to improving the design productivity and portability, though they do result in moderate hardware overhead and timing degradation.   

Between the processor level overlays and virtual FPGA level overlays, CGRA overlays on FPGA have unique advantages of compromising hardware implementation and performance especially for compute intensive applications as demonstrated by numerous ASIC CGRAs \ref{tessier2001reconfigurable} \ref{compton2002reconfigurable}. CGRAs on FPGA and ASIC have many similarities in terms of the scheduling algorithm and array structure, however, they have quite different trade-off on configuration flexibility, overhead and performance. Basically, CGRAs on ASIC need to emphasize more on configuration capability to cover more applications, while FPGAs' inherent programmability greatly alleviate this concern. Accordingly, CGRAs on FPGA could accept more intensive customization while design productivity comes up as a new challenge. 

\cite{ferreira2011fpga} proposed an heterogeneous CGRA overlay with multi-stage interconnection on FPGA, and the compilation can be done in milliseconds. While the CGRA size is quite limited and the implementation frequency is low due to the multi-stage interconnection. \cite{shukla2006quku} \cite{capalijia2013pipelined} employ coarse-grained reconfigurable array (cgra) as overlays, these overlays typically take data flow graph as input and the compilation doesn't involve any circuit optimization such as timing and piplining at all. particularly, \cite{capalijia2013pipelined} shows that the cgra overlay could run at high frequency, which is different from the virtual fpga overlay. these overlays inherit the advantages of traditional asic cgras \cite{tessier2001reconfigurable} in terms of exploring great parallelism as of the applications, preserve comparable implementation frequency and have more opportunities for cutomization thanks to the flexible fpga, therefore these cgra overlays present signigicant performance speedup, acceptable hardware overhead and extremely fast compilation. the benchmarks used are usually extracted from application kerenls and range from a few to dozens of nodes. and the evaluation is performed on a pure cgra overlay. nevertheless, using the cgra overlay as a fpga accelerator targeting a full application on a real system like a general processor (gpp) + fpga is still missing. 

Building on top of many the above ideas, we have opted to utilize a fully pipelined synchronous SCGRA as the overlay. Then we further implement it as an accelerator on Zedboard which is a hybrid ARM + FPGA system. The acceleratoin system is configurable and is capable to handle all of our four full compute intensive applications with diverse data sets. With this SCGRA overlay based acceleration system, an application can be implemented in a short time and the performance is competitive.

