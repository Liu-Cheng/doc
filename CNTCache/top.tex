\documentclass[conference]{IEEEtran}
\ifCLASSINFOpdf
  % \usepackage[pdftex]{graphicx}
  % declare the path(s) where your graphic files are
  % \graphicspath{{../pdf/}{../jpeg/}}
  % and their extensions so you won't have to specify these with
  % every instance of \includegraphics
  % \DeclareGraphicsExtensions{.pdf,.jpeg,.png}
\else
  % or other class option (dvipsone, dvipdf, if not using dvips). graphicx
  % will default to the driver specified in the system graphics.cfg if no
  % driver is specified.
  % \usepackage[dvips]{graphicx}
  % declare the path(s) where your graphic files are
  % \graphicspath{{../eps/}}
  % and their extensions so you won't have to specify these with
  % every instance of \includegraphics
  % \DeclareGraphicsExtensions{.eps}
\fi

\usepackage{booktabs} % For formal tables
\usepackage{multirow}
\usepackage{algorithm}
\usepackage[noend]{algorithmic}
%\usepackage[noend]{algpseudocode}
\usepackage[pdftex]{graphicx}
\usepackage[T1,hyphens]{url}
\usepackage[dvipsnames]{xcolor}
%\usepackage[colorlinks,urlcolor=blue]{hyperref}
\usepackage[]{hyperref}
\usepackage{subfigure}
\usepackage{amsmath}
\usepackage{siunitx}
% \usepackage{align}
% \usepackage{authblk}

% \usepackage[style=alphabetic,maxnames=1,minnames=1,maxbibnames=99]{biblatex}

\graphicspath{{./figures/}}
%\algrenewcommand\textproc{}

%\algsetup{linenosize=\small}

% correct bad hyphenation here
\hyphenation{op-tical net-works semi-conduc-tor}

\begin{document}
%
% paper title
% Titles are generally capitalized except for words such as a, an, and, as,
% at, but, by, for, in, nor, of, on, or, the, to and up, which are usually
% not capitalized unless they are the first or last word of the title.
% Linebreaks \\ can be used within to get better formatting as desired.
% Do not put math or special symbols in the title.
\title{CNT-Cache: an Energy-Efficient Carbon Nanotube Cache with Adaptive Encoding}
% Ironing out the irregularity in BFS for efficient OpenCL implementation on Xeon-FPGA
%\author[1]{Dawen Xu}
%\author[1]{Kexin Chu}
%\author[2]{Cheng Liu}
%\author[2]{Ying Wang}
%\author[2]{Lei Zhang}
%\author[2]{Huawei Li}
%\affil[1]{School of Electronic Science \& Applied Physics Hefei University of Technology Anhui,China}
%\affil[2]{Institute of Computing Technology, Chinese Academy of Sciences,Beijing,China}
%\affil[]{\{xudawen@hfut,chukexin2017@mail.hfut\}.edu.cn, \{liucheng,wangying2009,zlei,lihuawei\}@ict.ac.cn}

% make the title area
\maketitle

% As a general rule, do not put math, special symbols or citations
% in the abstract
\begin{abstract}
    Carbon Nanotubu field-effect transistor (CNFET) that 
    promises both higher clock speed and energy efficiency 
    becomes an attractive alternative to the conventional 
    power-hungry CMOS cache. We observe that the CNFET-based cache 
    constructed with typical SRAM cells has distinct energy consumption 
    when reading/writing 0 and 1 from/to it. For instance, the energy consumption 
    of writing 1 to an SRAM cell is almost 10X higher than writing 0. 
    With this observation, we propose an energy-efficient cache design 
    called CNT-Cache to take advantage of this feature. It predicts the cache 
    line access pattern based on the latest cache line access history.  
    On top of the prediction, it decides the optimal cache line encoding 
    to match the cache operation preferences at runtime. According to 
    our experiments on a set of benchmark programs, the optimized CNFET-based 
    D-Cache reduces the dynamic power consumption by 22.2\% on average 
    compared to the baseline CNFET cache.
\end{abstract}

% For peer review papers, you can put extra information on the cover
% page as needed:
% \ifCLASSOPTIONpeerreview
% \begin{center} \bfseries EDICS Category: 3-BBND \end{center}
% \fi
%
% For peerreview papers, this IEEEtran command inserts a page break and
% creates the second title. It will be ignored for other modes.
\IEEEpeerreviewmaketitle

\section{Introduction} \label{sec:intro}
Breadth-first search (BFS) is the basic building component of many graph algorithms 
and is thus of vital importance to high-performance graph processing. Nevertheless, 
it is notoriously difficult for accelerating on FPGAs because of the 
irregular memory access and the low computation-to-memory ratio. 
At the same time, BFS on large graphs also involves tremendous 
parallelisms which indicate great potential for acceleration. 
With both the challenge and the parallelization potential, 
BFS has attracted a number of researchers exploring its acceleration on FPGAs 
\cite{attia2014cygraph, betkaoui2012reconfigurable, Dai2017foregraph, Ma2017fpga,
umuroglu2015hybrid, oguntebi2016graphops, engelhardt2016gravf, zhou2016high}. 

Previous work have shown that BFS accelerators on FPGAs can provide competitive  
performance and superior energy efficiency when given comparable memory bandwidth. 
However, these work typically optimize BFS or relatively general graph processing 
with dedicated circuit design using hardware description language (HDL). The HDL 
based designs with customized circuits are beneficial to the resulting performance 
and save resource consumption, but it usually takes long time for development, 
upgrade, maintenance and porting to a different FPGA device, which are all 
important concerns from the perspective of the accelerator developers. Another 
engineering yet non-trivial problem is the high barrier to use the FPGA powered graph 
processing accelerators in high-level applications such as 
big data analytics, which is mostly caused by the lack of 
well-defined high level interface and user-friendly SDK supporting 
various hardware systems. Improving the ease of using the 
HDL based accelerators requires a lot of design efforts 
such as driver and runtime environment to support newer 
devices and diverse computing systems. This is also one of 
the key obstacles hindering the widespread adoption of 
the FPGA accelerators despite the great performance-energy 
efficiency advantages.

The limitation of the conventional HDL design method in combination with the 
rapid advancements of the HLS techniques makes the HLS tools attractive. 
HLS tools are increasingly adopted in both industry and academia for rapid FPGA prototyping and 
application acceleration. Software programmable FPGAs \cite{koch2016fpgas, xilinx-sdaccel} 
gets widespread acceptance. Nevertheless, the current HLS based design tools are mostly used for 
applications with relatively regular memory access patterns and data paths. 
It remains challenging for the HLS tools to accelerate BFS with irregular 
memory access patterns and complex data paths. In general, the main reasons lies in the 
following aspects. First of all, the HLS tools nowadays can support only very limited 
on-chip buffer optimizations, so it is rather difficult to handle irregular memory accesses 
especially random memory accesses. Secondly, hardware pipelining strategy in HLS tools 
is usually conservative to ensure the functional correctness, while 
this also leads to inefficient hardware implementations when the data path is data 
dependent or dynamic. Under such a context, we explore the use of Intel OpenCL for 
efficient BFS acceleration on software programmable FPGAs.

To cope with the irregular memory accesses and dynamic data paths in BFS, we proposed 
a series of optimization methods to regularize 
both the data path and memory accesses for efficient HLS implementation. 
We start with graph edge reordering. Basically, the edges 
are shuffled based on its destination vertices and divided into batches. In each batch, 
the edges point to different segments of the vertices. Then we have the vertex visiting 
status stored into multiple on-chip buffer banks while each bank stores the visiting status 
of vertices in different vertex segments. In combination with the edge batching, we can read 
and process the edges in the granularity of batches without any stall. In addition, we also have 
the coupled CPU to gather the scattered frontier vertices' edge location and combine them as 
as an sequential array such that the data path on the FPGA can be pipelined smoothly. 
According to the experiments on a set of big graphs, the optimized high level BFS 
accelerator achieves up to 70X performance speedup when compared to the design 
in Spector benchmark. It achieves around 80\% of the handcrafted design on 
similar FPGA boards. 

The major contributions of this work are summarized as follows.
\begin{itemize}
    \item As far as we know, this is the first highly optimized and open-sourced 
		HLS based BFS accelerator on FPGAs targeting portability and ease of 
		use on top of performance. 
    \item We proposed a set of combined methods to regularize the irregular
		memory accesses and dynamic data paths of BFS. This may shed light on similar 
		irregular application acceleration on FPGAs using HLS tools.
    \item The resulting accelerator shows significant performance speedup 
        over a baseline HLS design and gets close to state-of-art handcrafted 
		design on a set of representative graphs.
\end{itemize}

The rest part of the paper is organized as follows. In Section \ref{sec:relatedwork}, 
we brief the background of software programmable FPGAs and related work of 
BFS acceleration especially on FPGAs. In Section \ref{sec:motivation},  
we analyze the performance of basic BFS implementations with best-effort HLS 
optimizations and demonstrate the challenge of BFS acceleration using OpenCL. 
In Section \ref{sec:bfs-opt}, we present the overview of the BFS accelerator 
design using OpenCL and detail the major optimization methods.
In Section \ref{sec:experiment}, we present comprehensive experiments of the 
BFS accelerator. Finally, we conclude this work in Section \ref{sec:conclusion}.



\section{Related Work}\label{sec:relatedwork}
Despite their promising performance advantage, the relatively low design productivity of developing
FPGA applications remains a major obstacle that hinders widespread adoption of FPGAs as commodity
computing devices. To address this problem, the design of QuickDough was inspired by the recent success in HLS tools.
It also took advantage of modern FPGAs' capabilities to allow for an additional overlay architecture
be employed for productivity sake.

\subsection{High-Level Synthesis}
To bridge the design productivity gap between software and hardware application development, many researchers have turned to the use of HLS techniques \cite{cong2011high}.
By raising the abstraction level of the physical hardware, HLS allows designers to express hardware designs using familiar high-level, software-like description languages such as C, Java, or Python \cite{cardoso2010compiling,Canis:2011:LHS:1950413.1950423}.
The low-level hardware implementations are then left to the tools to synthesize and optimize.
Indeed, with decades of research, some early results in HLS have already found their ways into FPGA vendors' commercial tools in recent years \cite{chen2005xpilot, zhang2008autopilot, VivadoHLS}.

Unfortunately, when considering the overall design productivity of developing hybrid software-gateware applications, the raised abstraction provided by HLS is only addressing part of the problem.
While the high-level abstraction makes expressing complex functionalities as FPGA gateware easier, the lengthy low-level compilation time spent in synthesis, mapping, placing and routing remains a bottleneck to the overall design productivity for an application designer.
Such long compilation time is particularly challenging for novice designers who are accustomed to the high speed of software compilation.
Most importantly, it is significantly impacting the possible compile-debug-edit cycle achievable per day by a designer, negatively impacting the productivity of the designer.

\subsection{Overlay Architectures}
To improve the speed of low-level implementation tools, researchers have explored various approaches over the past decades.
Inspired by application specific integrated circuit (ASIC) design flows, researchers and vendors have developed modular design flow and explored the use of pre-compiled hard macros \cite{lavin2010using,lavin2011} as implementation library.
In addition, researchers have also exploited the use of dynamic partial reconfiguration capabilities in FPGAs \cite{Frangieh2010} as a way to improve productivity.
In recent years, there has been an increased interest in applying the concept of \emph{overlay architectures} as a way to address this productivity challenge.  


An overlay architecture is a virtual intermediate architecture that is overlaid on top of the physical configurable fabric of an FPGA.  They are employed during the FPGA application implementation process for purposes such as to improve portability, security, and also productivity.
%Depending on the design goal, overlays have manifested in various forms, including HDL models, pre-synthesized or pre-implemented coarse-grained circuits, or even arrays of processing elements with various granularity. 

One of the most familiar categories of overlay consists of virtual FPGAs \cite{zuma2013carl,Grant2011Malibu,Coole2010Intermediate,Koch2013CI}. They are built either virtually or physically on top of off-the-shelf FPGA devices and typically feature coarser configuration granularity than the physical device.
Similar to virtual machines running on a typical computer, such virtual FPGA provides an additional layer that improves application portability and security.
Furthermore, because of the coarser-grained configurable fabric, implementing designs on such overlay is relatively easier than on a fine-grained device.
However, the additional layer imposes restrictions on the underlying fabrics' capability and usually results in moderate hardware overhead and timing degradation.

Another category of overlay architecture commonly employed is in the form of coarse-grained reconfigurable arrays (CGRAs).
The use of CGRAs provides unique advantages of performance especially for compute intensive applications as demonstrated by numerous ASIC CGRAs \cite{tessier2001reconfigurable} \cite{compton2002reconfigurable}.
Indeed, CGRAs on FPGA and ASIC have many similarities in terms of the scheduling algorithm and array structure.
However, they have quite different trade-offs in terms of configuration flexibility, overhead and performance.
In a nutshell, CGRAs on ASIC emphasize more on configuration capability to cover more applications, while FPGAs' inherent programmability greatly alleviates the concern.
Instead, CGRAs on FPGA may take advantage of the configurability of the underlying fabric to allow more intensive customization tailored to the target application.

The authors in \cite{kissler2006dynamically} developed WPPA (weakly programmable processor array), a VLIW architecture based parameterizable CGRA overlay. It featured an interconnection wrapper unit for each processing element (PE) that could be used for dynamic CGRAs topology customization. Unfortunately, programming and compilation on WPPA were not presented. The authors in \cite{ferreira2011fpga} proposed a heterogeneous CGRA overlay with a global multi-stage interconnection on FPGA. Compiling applications onto the overlay took only milliseconds for smaller DFGs. However, the global multi-stage interconnection required multiple stages for communication between each pair of PEs and resulted in either low implementation frequency or large communication latency in terms of cycles. In addition, there was no intermediate storage except the pipeline registers in the CGRA and it limited the performance of the operation scheduling.
In \cite{shukla2006quku}, a customized CGRA overlay called QUKU was developed for DSP algorithms. It had two-level configuration capability, while the low-speed configuration was used for operator reuse within an application and high-speed reconfiguration was used for optimization between different applications. Nevertheless, the hardware infrastructure was consist of simple operation elements which can only be adapted to a few specified DSP algorithms.
The authors in \cite{capalijia2013pipelined} built a more generic high speed mesh CGRA overlay using the elastic pipeline technique to achieve the maximum throughput. It adopted a data-driven execution flow and was suitable for smaller pipelined DFG execution, while it would be difficult to handle applications with random IO access. 

In general, previous CGRA overlays have demonstrated the promising performance acceleration capability for compute intensive applications. They typically take DFG as design entry and focus on hardware infrastructure design as well as corresponding mapping and scheduling. However, they are still lack of consideration on proper loop unrolling for DFG generation, on-chip buffering, the communication with host and even end-to-end performance which are essential for FPGA accelerator design especially from a HW/SW co-design engineer's perspective. 


Finally, a third category of overlay features soft-processor-like architectures with high degree of
control and data parallelism suitable for FPGA accelerations.  For example, in the work of MARC
\cite{Lebedev2010}, a many-core overlay with customizable data path was proposed.  Similarly, a
GPU-like overlay was proposed in \cite{Jeffrey2011potential}.


In this work, we opted to utilize a fully pipelined synchronous soft coarse-grained reconfigurable
array (SCGRA) as an overlay to facilitate rapid FPGA accelerator generation in a hybrid CPU-FPGA
system. Compared to previously proposed CGRAs, our overlay is designed to be \emph{soft} as the size,
processing element designs, as well as the interconnect topologies may all be customized as needed
providing just enough resource for an application specifically. Moreover, the design of our overlay
is regular and design parameters such as loop unrolling factor and overlay size have
relatively predictable influence on the overlay performance and overhead, which makes the
customization much easier and more efficient. Finally, it also takes advantage of the large number
of on-chip distributed memory on the FPGA for intermediate data storage and can handle large DFGs
with thousands of nodes. 

%On top of the above approaches, the use of \emph{overlays} in the form of HDL Model, pre-synthesized or pre-implemented coarse-grained reconfigurable circuits over the fine-grained FPGA devices, promises both to raise the abstraction level and reduce the compilation time.
%Recent years have seen a number of overlay designs being developed with granularities ranging from multi-processors to highly configurable logic arrays \cite{Lebedev2010,kissler2006dynamically,unnikrishnan2009application,Yiannacouras2009FPS,Guy2012VENICE,Jeffrey2011potential}. 

%% Not so much overlay, removed for clarity sake.
%Soft processors, which allow customization for target applications or application domains, have already been demonstrated to be efficient overlays on FPGA. A great number of work use embedded processors as FPGA overlays with micro-architecture parameters such as pipeline depth configurable \cite{Yiannacouras2007Exploration,microblaze,nios} and 


%instruction set architecture (ISA) customizable \cite{grad2009woolcano, }. 


%multi-processor overlay with both micro-architecture and interconnection customizable \cite{unnikrishnan2009application}, 

% vector processors overlay \cite{Guy2012VENICE,Yiannacouras2009FPS}



\section{Energy Efficient CNFET-based Cache Design} \label{sec:warp}
The energy consumption of accessing '0' and '1' in CNFET-based SRAM cell 
differs dramatically. By taking advantage of the feature, we propose 
an adaptive run-time encoding approach and build an energy efficient 
CNFET-based cache named CNT-Cache. The architecture as well as the 
encoding approach will be detailed in this section.

\begin{figure}
    \center{\includegraphics[width=0.88\linewidth]{cache_archi}}
    \caption{Architecture of CNT-Cache}
\label{fig:architecture}
\vspace{-1em}
\end{figure}

\subsection{CNT-Cache Overview}
Figure \ref{fig:architecture} illustrates the proposed CNT-Cache 
architecture with adaptive encoding support.
Compared to a typical cache architecture, it requires a few additional 
components as highlighted in orange. The most critical part is the 
adaptive encoding module. It can encode each cache line data with 
either more '1' bits or '0' bits depending on the cache line operation 
preferences. For example, suppose the original data has more '1' bits and 
the operation is more efficient in handling '0' according 
to Table \ref{tab:rw-analysis}. Then we can encode the 
data by inverting each bits and have an additional bit to record the encoding 
direction. Note that the operation can either be read or write. Meanwhile, 
the adaptive encoder is essentially a series of inverters
with 2-to-1 multiplexers and operates based on the encoding direction bit read from 
each cache line. The simple structure has negligible influence 
on the timing of the critical data path. 

Another key component is the encoding direction predictor. 
It determines the encoding pattern or preference of each cache line based on 
a window of historical accesses. With the access pattern (i.e. the number of read 
and the number of write in a window), it checks whether the current cache 
line data matches the cache line encoding direction.
If the current encoding does not fit the cache line data, we
change the encoding direction and update the cache line data accordingly.
To avoid affecting the cache write data path, a data FIFO is used to delay the update
until there is an idle time slot. Meanwhile, an index FIFO is also needed to 
decide the update cache line address synchronously. Also the cache line history 
will be reset. If there is no need to change the encoding,
we just update the cache access counter in history region in the cache line.
To enable encoding direction prediction of each cache line, we have each 
cache line access history stored along with the cache line data. Therefore, we need to 
widen the cache line to store the access history as well 
as the encoding direction. The added bits are marked as 'H\&D' (history and direction) in Figure \ref{fig:architecture}.

In general, the proposed CNT-Cache adjusts the cache line encoding 
based on the cache line data access pattern to minimize the overall 
cache access energy consumption. The encoding is in the cache 
operation data path, so it must be simple for efficient 
hardware implementation. The predictor determines the cache line 
encoding direction based on the access history in a window as 
well as the encoding switch overhead. It requires more computing 
but will not affect the critical data paths of the cache operations. 

\subsection{Cache line data encoder}
The cache line data encoder operates based on the encoding direction 
read from the corresponding cache line. When there are less 
preferred bits in the cache line data, typically we need to invert 
the whole cache line. Nevertheless,
this approach may fail to explore the continuous preferred bits 
in the data and convert them to bits that are inefficient for 
the cache operations. To address this problem, we adopt a fine-grained 
encoding approach to further reduce the inefficient bits in the data. 
Basically the input data is divided into multiple partitions and 
each partition is encoded independently to maximize the preferred bits.
In this case, % each partition needs one bit to record the encoding direction and 
we need to add more direction bits to each cache line.

To help illustrate the partitioned encoding, an example is presented in 
Figure \ref{fig:encode}. The raw data has much more '0' bits than '1' bits.
When the cache line is read intensive, %according to the encoding direction, 
the data needs to be inverted using the baseline encoding approach. 
In this case, the $(K-1)$th partition of the cache line data with more '1' bits 
are inverted though they are preferred for cache read. 
When the partitioned encoding approach is applied, the $(K-1)$th 
partition remains unchanged. Similar optimization opportunities can also be explored 
when the cache line is write intensive. Compared to the baseline encoding,
the partitioned encoding needs more direction bits.

\begin{figure}
    \center {\includegraphics[width=0.85\linewidth]{encode}}
    \caption{An example of partitioned cache line encoding}
\label{fig:encode}
\vspace{-1em}
\end{figure}

The adaptive encoding module is essentially a series of inverters and 2 to 1 multiplexers
as described in Figure \ref{fig:architecture}.
When the partitioned encoding approach is utilized, the select signal of the multiplexers 
that are originally shared by the whole cache line data are now replaced with the partition 
direction bits instead. It has little influence on timing of the encoder module design. 

\subsection{Encoding direction predictor}
Encoding direction predictor determines the encoding of each cache line data.
It is of vital importance to the energy efficiency of the CNT-Cache.
While achieving optimal encoding for each cache access may cause frequent encoding 
direction switch and the switch overhead is non-trivial, thus the predictor 
performs prediction in the granularity of a window of accesses and decides if the encoding 
should be updated.

The proposed prediction strategy can be roughly divided into two steps as 
shown in Algorithm \ref{alg:prediction}. Firstly, we mainly analyze the operation 
preference of the cache line based on the access history. 
Basically, we keep a window $W$ of the latest accesses. 
$W$ also represents the prediction cycle. Within $W$, 
if the number of read is larger than a threshold $Th_{rd}$, 
it indicates that the cache line is read intensive. 
Otherwise, it is considered to be write intensive. 
The threshold is used to determine if it saves energy 
when we encode the stored data for reading. 
Suppose there are $x$ '0' bits and $y$ '1' bits on 
average in a window of accesses. Without loss of generality, assume $x<y$. 
The energy consumption using different encoding can be calculated with 
Equation \ref{eq:read_encoding_energy} and Equation \ref{eq:write_encoding_energy} 
respectively. Here $E_{rd0}$, $E_{rd1}$, $E_{wr0}$ and $E_{wr1}$ refer to 
the energy consumption of reading bit'0'/'1' and writing bit'0'/'1'. When $E_{rd\_enc}$ and $E_{wr\_enc}$ is equal, we can obtain $Th_{rd}$ using Equation \ref{eq:threshold}. 
Since $E_{rd0} - E_{rd1}$ is quite close to $E_{wr1} - E_{wr0}$ 
according to Table \ref{tab:rw-analysis}, $Th_{rd}$ is roughly half of $W$.

\begin{algorithm}
    \renewcommand{\algorithmicrequire}{\textbf{Input:}}
	\renewcommand{\algorithmicensure}{\textbf{Output:}}
    \caption{Encoding direction prediction algorithm} 
    \label{alg:prediction}
    \footnotesize
	\begin{algorithmic}[1] 
	    \REQUIRE Cache line access history including access 
	    number $A_{num}$, the number of write accesses $Wr_{num}$, and 
	    maximum access per prediction $W$, Cache line encoding direction $D$, 
	    Read intensive threshold $Th_{rd}$, Cache line data $Data$, 
	    Bit '1' number threshold array for encoding switch $Th_{bit1num}[W]$.
	    
	    \ENSURE Cache line access pattern ${Pattern}$, new encoding direction $D_{new}$ and 
	    new encoded data $Data_{new}$.
	    
	    \IF{($A_{num} = W$)}
	        \STATE
	        \STATE // Step 1: access pattern prediction
	        \IF{($Wr_{num} > Th_{rd}$)}
	            \STATE $Pattern \gets 1$ // write intensive
	        \ELSE
	            \STATE $Pattern \gets 0$ // read intensive
	        \ENDIF

            \STATE
            \STATE // Step 2: check if the cache line encoding will be changed.
	        \STATE $bit1num \gets getNumOfBit1(Data)$ // count '1' in $Data$
	        \IF {($Pattern = 1$)} %// write intensive
	            \IF{($bit1num > Th_{bit1num}[Wr_{num}]$)}
	                \STATE $D_{new} \gets \lnot D$
	                \STATE $Data_{new} \gets \lnot Data$
	            \ENDIF
	       \ELSE
	            \IF{($bit1num < Th_{bit1num}[Wr_{num}]$)}
	                \STATE $D_{new} \gets \lnot D$
	                \STATE $Data_{new} \gets \lnot Data$
	            \ENDIF
	       \ENDIF
	       
	       \STATE $A_{num} \gets 0$
	       \STATE $Wr_{num} \gets 0$
	       
	    \ELSE
	    	\STATE $Wr_{num} \gets Wr_{num} + 1$ when there is a cache write.
	        \STATE $A_{num} \gets A_{num} + 1$ when there is a cache access.
	    \ENDIF
	 
	%\EndProcedure
	\end{algorithmic}
\end{algorithm}

\footnotesize
\begin{equation}
    \label{eq:read_encoding_energy}
    \begin{split}
    E_{rd\_enc}=Th_{rd} \times (x \times E_{rd0} + y \times E_{rd1}) + \\ (W - Th_{rd}) \times (x \times E_{wr0} + y \times E_{wr1}) \\
    \end{split}
\end{equation}

\begin{equation}
    \label{eq:write_encoding_energy}
    \begin{split}
    E_{wr\_enc}=Th_{rd} \times (y \times E_{rd0} + x \times E_{rd1}) + \\ (W - Th_{rd}) \times (y \times E_{wr0} + x \times E_{wr1}) \\
    \end{split}
\end{equation}

\begin{equation}
\label{eq:threshold}
    \begin{split}
    % E_{rd\_enc} - E_{wr\_enc} = (y-x)(Th_{R/W} \times (E_{rd0} - E_{rd1}) + \\  (W-Th_{R/W}) \times (E_{wr0} - E_{wr1})) = 0 \\
    Th_{rd} = W \times \frac{1}{1 + \frac{E_{rd0} - E{rd1}}{E_{wr1}-E_{wr0}}}
    \end{split}
\end{equation}
\normalsize

As we predict the cache line read/write preferences using the two access 
counters i.e. $A_{num}$ and $Wr_{num}$, we need $2\times log{2}{W}$ 
bits to store them in each cache line. While it is usually expensive 
to add bits to the cache line, $W$ must be set properly to 
reduce the overhead of history bits.

In the second step, we mainly evaluate the current cache line data
and see if it fits the encoding and the access preferences in history, 
which is already determined in the first step. An intuitive approach 
is to calculate the energy efficiency of operating on the cache line 
data and compare with the cache access energy efficiency using a 
different encoding. If it consumes less energy, it means that the cache 
line data encoding is optimal. Otherwise, we need to reset the encoding 
and update the data stored in the cache line. 

The energy consumption of current cache line data access $E$ can be 
calculated using Equation \ref{eq:energy-efficiency}.
When a different encoding approach is used, the energy consumption 
becomes $\bar{E}$ as given in Equation \ref{eq:_energy-efficiency}.
Suppose $E_{encode}=N_{1} \times E_{wr0} + (L - N_{1}) \times E_{wr1}$ stands for the energy consumption of a cache 
line encoding switch which essentially updates the re-encoded data to cache.
When $E = \bar{E} + E_{encode}$, we can obtain the bit number ('1') threshold 
$Th_{bit1num}$ as presented in Equation \ref{eq:bit1num} 
where $E_{save} = (W-Wr_{num}) \times (E_{rd0}-E_{rd1}) - Wr_{num}\times(E_{wr1}-E_{wr0})$. 
Note that $bit1num$ represents the number of '1' bits in the data. 
It can be calculated using a bit counting function $getNumOfBit1()$ as described in 
Algorithm \ref{alg:prediction}. To make the representation short, 
we set $N_{1} = bit1num$. $L$ is the cache line length.

According to Equation \ref{eq:bit1num}, the threshold also depends on 
the cache line access history i.e. $Wr_{num}$ and it needs to be calculated at run-time.
Fortunately, we can obtain all the possible bit number threshold in advance 
and construct an array $Th_{bit1num}$ as presented in Algorithm \ref{alg:prediction}. 
The array can be implemented with a table that has $W$ entries. 
It returns the exact bit number threshold given $Wr_{num}$. In this case, the predictor 
can compare the number of bit '1' in the cache line with the threshold read from the 
table, and decides if it is necessary to inverse the encoding rapidly, which greatly 
simplifies the predictor design. 

\footnotesize
\begin{equation}
    \label{eq:energy-efficiency}
    \begin{split}
        %E=(W-Wr_{num}) \times (bit1num \times E_{rd1} + (L - bit1num) \times E_{rd0}) + \\
        %Wr_{num} \times (bit1num \times E_{wt1} + (L - bit1num) \times E_{wt0})
        E=(W-Wr_{num}) \times (N_{1} \times E_{rd1} + (L - N1) \times E_{rd0})\\
        + Wr_{num} \times (N_{1} \times E_{wr1} + (L - N1) \times E_{wr0})
    \end{split}
\end{equation}

\begin{equation}
    \label{eq:_energy-efficiency}
    \begin{split}
        %\bar{E}=(W-Wr_{num}) \times (bit1num \times E_{rd0} + (L - bit1num) \times E_{rd1}) + \\
        %Wr_{num} \times (bit1num \times E_{wt0} + (L - bit1num) \times E_{wt1})
        \bar{E}=(W-Wr_{num}) \times (N_{1} \times E_{rd0} + (L - N_{1}) \times E_{rd1}) \\ 
        + Wr_{num} \times (N_{1} \times E_{wr0} + (L - N_{1}) \times E_{wr1})
    \end{split}
\end{equation}

%\begin{equation}
%    \label{eq:energyencoding}
%    \begin{split}
%        E_{encode}=N_{1} \times E_{wr0} + (L - N_{1}) \times E_{wr1}
%    \end{split}
%\end{equation}

%\begin{equation}
%    \label{eq:save}
%        E_{save} = (W-Wr_{num}) \times (E_{rd0}-E_{rd1}) - Wr_{num}\times(E_{wr1}-E_{wr0})
%\end{equation}

\begin{equation}
    \label{eq:bit1num}
    \begin{split}
        N_{1} = \frac{L \times (E_{save} - E_{wr1})}{2\times E_{save} - (E_{wr1}-E_{wr0})}
    \end{split}
\end{equation}
\normalsize

%In addition, how to determine the data encoding pattern of the current cache block is the key to the effect of the above run-time encoding strategy. Because the misjudgment of each data encoding pattern will cause incorrect data encoding and sub-optimal performance.The sub-optimal performance can offset the overall optimization effect of the entire CNT-cache evenly. There are two things need to be considered when developing a effective judgment model: Firstly, the data encoding pattern of cache block should be determined by its previous access mode rather than recent access mode. Simply encoding data with its most recent access mode may result in frequent switching between different optimization patterns, which will cause energy waste. Secondly, the encoding phase should be removed from the critical path. For example, if every normal access to the cache needs to wait for the results of the prediction process, the delay it generates will significantly affect the performance of the cache.

%In general, the dynamic power consumption of cache in a period of time can be calculated by using the following formula:
%\begin{align}
%E = N_{allr} \times E_{avgr} + N_{allw} \times E_{avgw}
%\end{align}
%where, $N_{allr}$ and $N_{allw}$indicates the overall number of read- /write- accesses to the cache during run-time.$E_{avgr}$ and $E_{avg_w}$ represent the average read or write power consumption for a cache block. 
%However, in this article, not only how many cache blocks are read/written, but also the 0/1 number of stored data should be considered. So the above formula can be written as:
%\begin{align}
%E = \sum_{C=0}^N(E_{cache block})
%\end{align}
%\begin{align}
%E_{cache block} = E_{read} \times N_{read} + E_{write} \times N_{write}
%\end{align}
%\begin{align}
%E_{read} = N_{read0} \times E_{read0} + N_{read1} \times E_{read1}
%\end{align}
%\begin{align}
%E_{write} = N_{write0} \times E_{write0} + N_{write1} \times E_{write1}
%\end{align}
%Here $E_{cache block}$, $E_{read}$ and $E_{write}$ indicates the power consumption of one single cache block or one single read-/write to a cache block, $N_{read}$ and $N_{write}$ represents the number of read/write accesses to the cache block in a period if time. C and N are used to summarize the dynamic power consumption of all cache blocks during the period of time.

%Then, during the run-time of the system, we divide the run-time into small segments for each cache block(checkpoint), And record the number of accesses and write accesses to the current cache line during the checkpoint($AC$ indicates whether a checkpoint is completed and $WC$ indicates the number of write accesses). After that, the number of bit 0/1 in stored data will be recorded by Bit Counter, and the dynamic power consumption of each optimization direction will be calculated and compared. Finally, if the new direction performs better, the data encoding and write back processes will be started, updating the flag bits and real data while clearing other flag information. one thing to note here is that if the encoding direction changes, it will cause an additional writeback operation, so here a write operation is added when calculating the dynamic power consumption of the new direction, which means $E_{new} = E_{read} \times N_{read} + E_{write} \times (N_{write} + 1)$.

%Of course, it is unrealistic to calculated the above formula for each pattern prediction, because the formula above is too complicated. It is found that the TB(defined as the threshold for the number of Bits 1 when the mode changes) is related to the number of write accesses and k via calculation Within a limited number of accesses. Once these three values are determined, the TB is uniquely determined. Since the size of checkpoint and K are pre-set, only a small hash table(equal to the size of checkpoint) is needed to avoid all calculations.
	

%\subsubsection{Threshold}
%As mentioned above, firstly, the access power consumption can be reduced only by accurately predicting the access pattern of each cache block, because inaccurate access pattern prediction will only lead to higher waste of access energy. Secondly, the optimization type of the cache block should be determined by its previous access pattern, since simply encoding the data with the latest access mode may result in frequent switching between different optimization methods, which will cause additional energy wastage. In order to avoid inaccurate data encoding as much as possible, we introduce a threshold $\Delta$T to determine whether the optimization pattern of current cache block needs to be changed. The $\Delta$T indicates that the new encoding direction saves more than $\Delta$T of the original one during the checkpoint. In other words, the new pattern becomes the stable optimization pattern only when $E_{original} - E_{new} > \Delta T \times E_{original}$ is satisfied. In summary, the algorithm of the pattern predictor is shown in Algorithm.\ref{alg:prediction}. By default, we set checkpoint as 15 accesses operation to each cache block. And in order to save the maximum power overhead, we will explore the relationship between $\Delta$T and dynamic energy saving through a series of experiments. The results are shown in section\ref{sec:result}.

\section{Experiments} \label{sec:experiment}
We measure the performance of the HLS based BFS 
accelerator on Alpha Data ADM-7v3 and KU115 using a set of 
representative graphs and compare them to both a 
baseline design and previous handcrafted BFS accelerators. 
The baseline design refers to a design with HLS pragmas 
added to the native C based level synchronous BFS implementation. 
Then we briefly evaluate the design optimization 
methods including pipelining, redundancy removal, 
caching and data path duplication 
proposed in this work. Based on the design on ADM-7v3, 
we further ported the design 
to KU115 on Nimbix Cloud and explored the portability of the 
BFS accelerator. 

\subsection{Experiment Setup}
The graph benchmark used in this work includes three real-world graphs and 
two synthetic graphs generated using R-MAT model \cite{chakrabarti2004rmat} 
as listed in Table \ref{tab:graph}. The real-world graphs are from social network 
\cite{snapnets} while the R-MAT graphs are generated 
using the Graph 500 benchmark parameters ($A=0.59, B=0.19, C=0.19$). To make the 
presentation easier, the five benchmark graphs are shorted as Youtube, 
LJ, Pokec, R-MAT\uppercase\expandafter{\romannumeral1}, 
R-MAT\uppercase\expandafter{\romannumeral2} respectively. We refer 
to an R-MAT graph with scale $S$ ($2^{S}$ nodes) and edge factor $E$ ($E\times 2^{S}$). 
In order to avoid trivial search, we only choose vertices from the largest 
connected component as the BFS starting point.

\begin{table}
    \centering
  \vspace{-0.3em}
  \caption{Graph Benchmark}
  \label{tab:graph}
  \vspace{-0.3em}
  \begin{tabular}{cccc}
    \toprule
      Name & \# of vertex & \# of edge & Type \\
    \midrule
      Youtube & 1157828 & 2987624 & Undirectional \\
      LJ & 4847571 & 68993773 & Directional \\
      Pokec & 1632804 & 30622564 & Directional \\
      R-MAT\uppercase\expandafter{\romannumeral1} & 524288 & 16777216 & Directional \\
      R-MAT\uppercase\expandafter{\romannumeral2} & 2097152 & 67108864 & Directional \\
  \bottomrule
\end{tabular}
\vspace{-1em}
\end{table}

\subsection{Performance comparison}
We use the million traverse per second (MTEPS) as 
the performance metric. The performance of the proposed BFS 
accelerator on the graph benchmark is 
presented in Table \ref{tab:performance-summary}. 
The implementation on ADM-7v3 achieves up to 
82.16 MTEPS on the R-MAT\uppercase\expandafter{\romannumeral1} graph and 
38.83 MTEPS on average. When compared to a baseline HLS based 
BFS accelerator, the proposed design shows 24.7X to 77.5X performance 
speedup on the benchmark. 
With the comparison, it is clear that straightforward HLS optimizations 
are far from sufficient and dedicated high level optimizations are critical to 
the performance of the resulting BFS accelerator.
\begin{table}
  \vspace{-0.3em}
    \centering
  \caption{Performance summary}
  \vspace{-0.3em}
  \label{tab:performance-summary}
  \begin{tabular}{cccccc}
    \toprule
      Benchmark & Youtube & LJ & Pokec & RMAT\uppercase\expandafter{\romannumeral1} & RMAT\uppercase\expandafter{\romannumeral2} \\
    \midrule
      MTEPS & 14.35 & 28.05 & 36.94 & 82.16 & 32.67 \\
      Speedup & 77.50 & 36.82 & 38.83 & 62.18 & 24.70 \\
  \bottomrule
\end{tabular}
\vspace{-1em}
\end{table}

We also compare this work to a set of existing BFS accelerators on FPGAs. 
As the platforms and graph benchmarks used in these work are mostly different and it is 
difficult to make a complete fair end-to-end comparison. Here we provide two implementations 
on Alpha-Data ADM-7v3 and KU115 respectively. A rough comparison result is listed 
in Table \ref{tab:compare}. The best HLS based BFS implementation on KU115 is 
getting close to that in \cite{zhang2017boosting} 
and \cite{nurvitadhi2014graphgen}, though the peak memory bandwidth 
is relatively higher. When compared to design on high-end 
FPGA computing system such as Convey HC-2 with highly optimized memory sub systems, 
the performance is still much lower. 

Since different FPGAs may have diverse memory bandwidth, we also measure the 
per bandwidth BFS performance i.e. MTEPS/GB. According to the experiments, we 
can see that the HLS based BFS accelerator on Alpha-Data achieves higher 
MTEPS/GB. The comparison shows that the memory bandwidth on KU115 is not 
fully explored. This is mainly caused by the fact that only 16 global memory 
ports are allowed to be implemented in the SDAccel design and 
limited parallel data paths can be instantiated on the FPGAs as 
mentioned in previous section. We believe the performance of the 
proposed design can be further improved given more parallel data paths.

\begin{table}
  \vspace{-0.3em}
  \caption{FPGA based BFS accelerator comparison}
  \label{tab:compare}
    \setlength{\tabcolsep}{4pt} % Default value: 6pt
    %\renewcommand{\arraystretch}{1.5} % Default value: 1
  \vspace{-0.3em}
  \begin{tabular}{cccccc}
    \toprule
      Work & Platform & Graph & MTEPS & BW(GB/s) & MTEPS/GB \\
    \midrule
      \cite{betkaoui2012reconfigurable} & Convey HC-2 & R-MAT & 1600 & 80  & 20 \\
      \cite{attia2014cygraph} & Convey HC-2 & R-MAT    & 1900 & 80  & 23.8 \\
      \cite{zhang2017boosting} & Micro-AC510       & R-MAT  & 166.2  & 60  & 2.8 \\
      \cite{nurvitadhi2014graphgen} & VC707 Kit & Twitter & 148.6 & 12.8 & 11.6 \\
      \cite{dai2016fpgp}  & VC707 Kit & Twitter & 12  & 12.8 & 0.95 \\
      this work & ADM-7v3 & R-MAT & 57.41 & 10.8 & 5.3 \\
      this work & ADM-7v3 & Table \ref{tab:graph} & 38.8 & 10.8 & 3.6 \\
	  this work & KU115 & R-MAT & 120.84 & 76.8 & 1.57\\
	  this work & KU115 & Table\ref{tab:graph} & 77.98 & 76.8 & 1.02\\
  \bottomrule
\end{tabular}
\vspace{-1em}
\end{table}

\subsection{Design Configuration and Resource Overhead}
With the software emulation based tuning, 
we can decide the design configurations rapidly. The graph specific configuration 
of the BFS accelerator targeting ADM-7v3 is summarized in 
Table \ref{tab:parameter-setup}. The hash tables for LJ and R-MATII 
as highlighted in the table are shrunk to fit for the on-chip memory constraints. 
Note that the \textit{depth} read and write cache 
are set to be the same and the cache size in the table refers to the capacity of 
one cache size.

\begin{table}
  \vspace{-0.3em}
  \caption{Memory optimization parameter setup on ADM-7v3}
  \label{tab:parameter-setup}
  %\setlength{\tabcolsep}{4pt} % Default value: 6pt
  %\renewcommand{\arraystretch}{1.5} % Default value: 1
    \centering
  \vspace{-0.3em}
  \begin{tabular}{ccccccc}
    \toprule
      Benchmark & Hash Table & Cache Size & Prefetch Buffer \\
    \midrule
      Youtube  & 256K  & 16K $\times$ 64B & 64B \\
      LJ       & \textbf{512K} & 32K $\times$ 64B & 64B \\
      Pokec    & 1024K & 16K $\times$ 64B & 64B \\
      R-MATI   & 512K  & 8K $\times$  64B & 64B \\
      R-MATII  & \textbf{512K} & 32K $\times$ 64B & 64B \\
  \bottomrule
\end{tabular}
\vspace{-1em}
\end{table}

The corresponding FPGA resource consumption is 
presented in Table \ref{tab:mem-resource}. 
FF and LUT consumption don't change much with the different 
design configurations and they take up only a small portion 
of the total FPGA resources. Block RAMs turns out to be the major 
resource bottleneck, and it leads to the adoption of sub optimal 
design configurations.

\begin{table}
  \vspace{-0.5em}
  \caption{FPGA resource consumption on ADM-7v3}
  \label{tab:mem-resource}
  \vspace{-0.3em}
  %\setlength{\tabcolsep}{4pt} % Default value: 6pt
  %\renewcommand{\arraystretch}{1.5} % Default value: 1
    \centering
  \begin{tabular}{ccccccc}
    \toprule
      Config. & FF & \% & LUT & \% & RAMB18K & \% \\
    \midrule
      Youtube  & 65244 & 7 & 108810 & 25 & 1515  & 51 \\
      LJ       & 65266 & 7 & 108829 & 25 & 2784  & 94 \\
      Pokec    & 65262 & 7 & 108812 & 25 & 2155 & 73 \\
      R-MATI   & 65244 & 7 & 108808 & 25 & 1217 & 41 \\
      R-MATII  & 65266 & 7 & 108829 & 25 & 2784 & 94 \\
  \bottomrule
\end{tabular}
\vspace{-1em}
\end{table}

\subsection{Optimization evaluation}
In this section, we evaluate the 
performance of the BFS accelerators with the different optimizations. 
Basically we start from the baseline design and 
add the optimizations including pipelining, hash redundancy removal, 
prefetching, caching and data path duplication in order. The performance improvement 
can be found in Figure \ref{fig:opt-performance}. 
In general, the performance of the BFS accelerator improves 
significantly when more optimization techniques are applied. Particularly,
pipelining and data path duplication enhance the performance most. 
The performance improvement brought by the hash table based filtering 
seems to be trivial, but it actually boosts the performance by over 20\% on average. 
In addition, it also affects the cache efficiency as observed in Section \ref{sec:observation}
and is thus critical to the overall accelerator performance.

\begin{figure}
\center{\includegraphics[width=0.85\linewidth]{opt-performance}}
    \caption{BFS accelerator optimization technique evaluation. The performance on 
    all the graphs improves when more optimizations including pipelining, 
    redundancy removal, prefetching, caching, and data path duplication are 
    gradually applied to the design.}
\label{fig:opt-performance}
\vspace{-1em}
\end{figure}

There is only one memory bank available in ADM-7v3, so we 
evaluate the memory bank-aware data layout strategy by porting the design to KU115.
Without the bank-aware layout optimization, porting the design from ADM-7v3 to 
KU115 achieves less performance improvement despite the much larger memory 
bandwidth on KU115. When the optimization is applied, the multiple-bank memory 
on KU115 can be utilized. The performance of the BFS accelerator on KU115 improves 
significantly as shown in Table \ref{tab:porting-summary} especially for the 
graphs with more edges. This experiment also 
demonstrated the portability of the proposed HLS based BFS accelerator.
\begin{table}
	\vspace{-0.5em}
    \centering
	\caption{Memory-bank aware data layout optimization influence on the BFS accelerator performance (MTEPS)}
  \label{tab:porting-summary}
  \vspace{-0.3em}
  \begin{tabular}{cccccc}
    \toprule
	Benchmark & Youtube & LJ & Pokec & RMAT\uppercase\expandafter{\romannumeral1} & RMAT\uppercase\expandafter{\romannumeral2} \\
    \midrule
	ADM-7v3 & 14.35 & 28.05 & 36.94 & 82.16 & 32.67 \\
	KU115 with bank opt. & 18.69 & 61.49 & 68.04 & 122.48 & 119.2 \\
	KU115 no bank opt. & 14.15 & 41.74 & 40.82 & 85.18 & 63.31\\
  \bottomrule
\end{tabular}
\vspace{-1em}
\end{table}

\section{Discussion}
According to our experience using SDAccel for BFS accelerator design, 
we have come up with a few insights on accelerating complex applications 
with HLS design tools. 1) The high level design tools are able to produce competitive 
hardware implementations for not only regular applications but 
also irregular applications. 2) Optimizing the HLS design for higher performance is non-trivial. It is not 
so much friendly to software programmers as expected. Hardware design 
experiences are important for optimizing the HLS based design. 
3) The HLS tools are still not as mature as the software compilers. 
It is not rare to come across a design which is functionally correct 
in software emulation but wrong in hardware implementation. The bugs are 
even more difficult to address than the HDL design bugs. It may be caused by 
many different factors such as concurrent memory access conflict or deadlock. 
4) Applications with dynamic memory accesses can be found in many applications. 
And customized hardware architectures such as hash table, and cache, channels etc are 
classical wisdom that can be applied to optimize these applications. However, 
these building blocks in existing HLS design tools are still quite limited. More 
support on these features will be beneficial to complex application 
acceleration with HLS tools.

\section{Conclusions} \label{sec:conclusion}
Handcrafted HDL based BFS accelerators usually suffer 
high portability and maintenance cost 
as well as ease of use problem despite the relatively 
good performance. HLS based BFS accelerator can greatly 
alleviate these problems, but it is 
difficult to achieve satisfactory performance due to 
the inherent irregular memory access and 
complex nested loop structure. In this work, we developed 
a series of HLS based optimizations such as redundancy removal, 
prefetching, caching and data path duplication. 
With the optimizations, BFS performance can be greatly improved. 
According to the experiments on 
a representative graph benchmark, the resulting HLS based BFS accelerator 
achieves up to 70X speedup compared to a baseline HLS design. 
When compared to the existing HDL based BFS accelerators on 
similar FPGA cards, the proposed HLS based BFS accelerator on KU115 
gets over 70\% of the MTEPS while it still preserves the 
software-like features including portability and ease of use 
and maintenance.  


\section{Conclusion} \label{sec:Conclusion}
In this paper, we propose to take the CNN accelerator’s ‘undeterministic’ behaviors into consideration 
at training and have the CNN model to learn the accelerator’s behaviors. To that end, we further build 
an open-sourced training system based on Caffe on a hybrid CPU-FPGA architecture. Then use the training 
system to deal with an overclocked CNN accelerator and an accelerator with soft errors. According to our 
experiments, the proposed training can improve the prediction accuracy of four CNN models up to 3.4\% when 
the CNN accelerator is overclocked on the extreme situation. This method is also beneficial to the CNN 
accelerators with soft errors. In the case with most soft errors, it improves the prediction accuracy up 
to 6.8\% and by 3.58\% on average. The disadvantage is the much longer training time due to the frequent 
data transfer between host memory and device memory. This problem can be resolved when porting the system 
to closely coupled CPU-FPGA architectures with shared memory.

%\appendix
%\section{Acknowledgement}

%\begin{acks}
%  The authors would like to thank Sam Ho for providing the suggestions on
%  HLS design debugging and optimization as well as the SDAccel usage. 

%\end{acks}


\bibliographystyle{IEEEtran}
\bibliography{refs} 


% trigger a \newpage just before the given reference
% number - used to balance the columns on the last page
% adjust value as needed - may need to be readjusted if
% the document is modified later
%\IEEEtriggeratref{8}
% The "triggered" command can be changed if desired:
%\IEEEtriggercmd{\enlargethispage{-5in}}

% references section

% can use a bibliography generated by BibTeX as a .bbl file
% BibTeX documentation can be easily obtained at:
% http://mirror.ctan.org/biblio/bibtex/contrib/doc/
% The IEEEtran BibTeX style support page is at:
% http://www.michaelshell.org/tex/ieeetran/bibtex/
%\bibliographystyle{IEEEtran}
% argument is your BibTeX string definitions and bibliography database(s)
%\bibliography{IEEEabrv,../bib/paper}
%
% <OR> manually copy in the resultant .bbl file
% set second argument of \begin to the number of references
% (used to reserve space for the reference number labels box)
%\begin{thebibliography}{1}

%\bibitem{IEEEhowto:kopka}
%H.~Kopka and P.~W. Daly, \emph{A Guide to \LaTeX}, 3rd~ed.\hskip 1em plus
%  0.5em minus 0.4em\relax Harlow, England: Addison-Wesley, 1999.

%\end{thebibliography}




% that's all folks
\end{document}


