\section{Related Work} \label{sec:relatedwork}
CNFET with high energy efficiency becomes an attractive 
alternative to power-hungry CMOS based cache. It attracts 
a lot of attentions of researchers and a large number of 
CNFET-based SRAM cell designs and cache designs have been 
proposed.

The authors in \cite{wang2011design,wang2011high} developed 
a CNFET-based 6T SRAM cell and demonstrated the great 
performance and energy efficiency advantages over a conventional 
CMOS-based SRAM cell. However, the CNFET-based 6T SRAM cell suffers 
high leakage current and standby power consumption. To address the problem,
\cite{zhang2012sram,kim2008low} proposed CNFET-based 8T SRAM cell designs, 
Along with this route, \cite{sun2014novel, sun2017high} proposed 
CNFET-based 9T SRAM cell designs to further reduce the leakage power.
In addition, the CNFET-based 9T SRAM cells also enhance the electrical 
quality and promise better read/write stability, which are also of 
vital importance to SRAM cell design. 

On top of the CNFET-based SRAM cell design, there are still many 
fabrication challenges that need to be remedied with appropriate 
architectural design. For instance, metallic-CNT variation may cause 
invalid SRAM cells and CNT density variation may lead to asymmetric 
electrical features. To detect the invalid cells efficiently, 
the authors in \cite{xie2015jump, Li2016A} proposed to jump over 
multiple consecutive cells rather than marching through every cell 
during testing. The authors in \cite{jiang2017cnfet} proposed a flexible 
row/column SRAM cell disabling strategy to minimize the negative impact of 
invalid cells. Unlike the disabling scheme, the authors in \cite{li2016defect} 
proposed to add redundant SRAM array blocks that can be shared to fix 
various faulty blocks. To address the asymmetric electrical problems, 
\cite{jiang2017cnfet} proposed a CNFET-based register file to take advantage of the 
feature and proposed a dynamic register assignment scheme such that registers with 
shorter access delay can be used for frequent data. Similarly, the authors in 
\cite{xu2019exploring} explored the feature and proposed a new last level cache design 
supporting variable cache access delay.

Unlike prior work that mainly focus on the performance improvement, we aim to further 
improve the energy efficiency of CNFET-based Cache. We observed that the operation 
power of a CNFET-based SRAM cell \cite{sun2014novel, sun2017high} varies 
dramatically depending on the value (0 or 1) stored in the cell. According to 
Table \ref{tab:rw-analysis}, read operation prefers '0' while write prefers '1'. 
For a cache line data with both '0' and '1', we may encode the data to adapt to 
the cache line operation pattern for higher energy efficiency.

There are also works proposed to balance memory read and write with 
data encoding from different angles \cite{cho2009flip} 
\cite{jacobvitz2013coset} \cite{seyedzadeh2015pres}. They typically  
target at STT-RAM or PCM for various purposes such as improving 
memory lifetime. Although their solutions cannot be employed to address the problem of CNT-cache due to their distinct memory architecture and memory operation characteristics, 
these encoding algorithms inspire this work.
