\section{Conclusion} \label{sec:conclusion}
CNTFET has become a promising technology for cache design due to the 
potential high operation speed and energy efficiency. Nevertheless, we 
observe that typical CNFET-based SRAM cells have distinct energy consumption 
for accessing 0 and 1. By taking advantage of the CNTFET feature, we propose 
CNT-Cache design that allows runtime encoding of each cache line data based on 
its latest access history. Basically, it predicts the cache line access 
preferences which can either be read intensive or write intensive. Then it 
decides the optimal cache line data encoding and updates the encoding at 
runtime to adapt to the variation of the applications. According to our 
experiments on a set of benchmark program on gem5, the proposed CNT-Cache 
reduces the dynamic energy consumption of Dcache by 22.2\% on average with 
2.21\% chip area overhead.
