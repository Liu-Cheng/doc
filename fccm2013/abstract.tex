Compared to the use of a typical software development flow, the productivity of developing FPGA-based compute applications remains much lower. Although the use of high-level synthesis (HLS) tools may partly alleviate this shortcoming, the lengthy low-level FPGA implementation process remains a major obstacle to high productivity computing, limiting the number of compile-debug-edit cycles per day. Furthermore, high-level application developers often lack the intimate hardware engineering experience that is needed to achieve high performance on FPGAs, therefore undermining their usefulness as accelerators.  To address these productivity and performance problems, a high-level synthesis methodology that utilizes soft coarse-grained reconfigurable arrays (SCGRAs) as an intermediate compilation step is presented. Instead of compiling high-level applications directly as circuits implemented on the FPGA, the compilation process is reduced to an operation scheduling task targeting the SCGRA. Furthermore, the softness of the SCGRA allows domain-specific design of the processing elements, while allowing highly optimized SCGRA array be developed by a separate hardware design team. An SCGRA operating at over 400MHz on a commercial FPGA is presented here. When compared to commercial high-level synthesis tools, the proposed design methodology achieved 0.8-21x times speedup in the application run time while application compilation time is reduced by 10-100x. 
