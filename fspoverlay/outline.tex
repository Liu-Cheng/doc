\begin{center}
\Large Outline
\end{center}

\begin{raggedleft}
\footnotesize

\begin{itemize}[nosep]
\item Motivation
\begin{itemize}[nosep]
\item Difference in hardware and software design experience
\item Many features trivially expected by modern software designers are not available in hardware -- high design productivity, fast debug cycles, relocatable code, portability, virtualization, etc.
\item  Overlay provides many of such features
\end{itemize} % 1st level
\item FPGA Overlays

\begin{itemize}[nosep] % 2nd level

\item
Definition of overlay
\begin{itemize}[nosep]
\item Definition: A virtual reconfigurable architecture that overlays on top of the physical FPGA configurable fabric.
\end{itemize}

\item
Overlay vs Custom Design vs CGRA
\begin{itemize}[nosep]
\item  Although very similar, overlay $\ne$ custom design
\item Overlays are designed to be somewhat general purpose (or at least compatible with a domain of applications)
\item  Overlay $\ne$ plain, hard, CGRA
\item The use of CGRA is a mean to achieve other goals in the overlay.
\item  The CGRA is ``soft'' by design
\end{itemize}

\item
Benefits of Overlays
\begin{itemize}[nosep]
\item Virtualization
\begin{itemize}
\item Portability
\item Security
\item Compatibility
\end{itemize}
\item Design productivity
\begin{itemize}
\item Compilation time
\item Debugging facilities
\end{itemize}
\item HW/SW Separation of Concerns 
\begin{itemize}
\item SW: map application to overlay, which is more software friendly
\item HW: map overlay to physical hardware
\end{itemize}

\end{itemize}

\item
Short Survey of existing Overlay Designs
\begin{itemize}[nosep]
\item Virtual FPGAs
%\item Hard macro/modular design methodologies
\item CGRAs
\item Multi-core/MPPA/Other processor arrays
\item GPU / OpenCL overlay
\item etc
\end{itemize}
(Questions: maybe we want to categorize overlays according to the intended uses (e.g. security, portability, etc)
(Another way to categorize them will be by their different architecture)
(The work presented at OLAF can be a good starting point on a survey of existing overlay projects.)
\end{itemize} %end 2nd level

\item
Case Studies
Here we show various examples of overlays as “case studies”.

\begin{itemize}[nosep]
\item QuickDough
\item Malibu? Zuma?
\end{itemize}

\item
State-of-the-art and availability/usability for software engineers
\begin{itemize}[nosep]
\item (I don’t think there is any commercially available overlay offer.  Software compatibility layers may be available in some way.
\item e.g. Think of VPR and the “virtual” target FPGA. It can be a compatibility target for different FPGA vendors if needed.)
\item Tools for Overlays?
\end{itemize}
\end{itemize}

\end{raggedleft}
