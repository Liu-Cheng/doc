As a case study to illustrate how an FPGA overlay works in practice, the design and implementation of the research project QuickDough will be examined in this section.

\input{quickdough}


\iffalse
In this section, the designs of two research overlay will be examined in detail as case studies.  The first overlay, called QuickDough, was developed as part of a larger automatic hardware-software accelerator generator framework.  It demonstrates how coarse-grained reconfigurable array can serve as an efficient overlay and provide reasonable performance at the end.  When combined with 

The second overlay that we will examine is Zuma, one of the first virtual FPGA overlay.  It demonstrates how an FPGA can be built on top of a native FPGA efficiently for sake of providing compatible and portable designs.

\input{quickdough}
\input{zuma}

\fi


