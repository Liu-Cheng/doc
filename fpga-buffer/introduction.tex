\section{Introduction}
Offloading compute intensive nested loops to FPGA accelerators has 
been demonstrated by many researchers to be an effective solution 
for performance enhancement across many application domains \cite{Chung2010}. 
However, the relatively low productivity in developing FPGA-based 
compute applications remains a major obstacle that hinders 
their widespread employment \cite{cong2011high}.

To address this productivity challenge, researchers have recently 
turned to the use of virtual FPGA overlay architectures and 
high-level compilation tools that, when combined properly, 
are able to produce high-performance accelerators at near
software compilation speed \cite{Grant2011Malibu, ZUMA2012, 
mesh-FUs, ferreira2011fpga, kissler2006dynamically, scgra}.
Not only have these early works illustrated the promise of using overlays 
to improve productivity, they have also highlighted a unique potential 
of overlays --- By customizing the architectures of these \emph{virtual} 
yet often regular overlays for a target user design, in theory, it is 
possible to significantly improve the power-performance of the 
resulting accelerator. In practice, however, navigating through a 
labyrinth of architectural and compilation parameters to fine-tune 
an accelerator's power-performance is a slow and non-trivial process. 
To require a user to manually explore such vast design space is going 
to counteract the productivity benefit of the utilizing overlay 
in the first place.

To address both the design productivity and performance challenge, 
we have developed a soft coarse-grained reconfigurable array (SCGRA) overlay based 
nested loop acceleration framework targeting a hybrid CPU-FPGA system.
In this framework, given high-level design constraints, the framework automatically  
customizes the overlay architectural parameters, exploits loop unrolling 
and hardware-software communication in combination 
with buffer sizing specifically to an application.  
In addition, by taking advantage of the regularity of the SCGRA overlay, 
a multitude of design metrics such as performance and hardware consumption can 
be accurately estimated using analytical models once the overlay scheduling 
result is available. While the overlay scheduling depends on much less design 
parameters, the overall customization framework can be dramatically simplified.
When the optimized design parameters are decided, the 
corresponding hardware accelerator and communication interface 
are generated and both the hardware accelerator and software 
are compiled to the hybrid CPU-FPGA system.
Moreover, the SCGRA overlay allows rapid bitstream reuse \cite{scgra} during 
design iterations of an application. With an initial overlay implementation, 
nested loops expressed in high-level languages can be compiled to the SCGRA overlay in seconds. 
With both the efficient application-specific customization and rapid bitstream reuse,   
the proposed design framework ensures both high design productivity and 
high performance of FPGA loop acceleration.

According to the experiments, it takes the proposed design framework around 10 minutes
to 20 minutes to complete the loop accelerator customization. When compared to the performance of
the benchmark executed on a hard ARM processor, the resulting FPGA accelerators achieve up to 10X speedup.  

With that, we consider the main contribution of this work is in the following areas:
\begin{itemize}[nosep]
\item We have developed an SCGRA overlay based FPGA loop accelerator design framework that 
    automatically performs FPGA accelerator architectural design parameters, compilation options and
    communication tuning. The resulting accelerators have shown promising performance speedup over
    an ARM processor.
\item By taking advantage of the regularity of the SCGRA overlay based FPGA accelerator, we have
    built a series of performance and overhead models and further simplified the customization
    problem using these models.
\end{itemize}


