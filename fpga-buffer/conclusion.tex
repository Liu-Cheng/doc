%\section{Limitations and Future Work} \label{sec:limitations}
%Despite the promising advantages of the proposed customization framework, there 
%are a few limitations that must be acknowledged and hopefully addressed 
%in future work.
%
%First of all, the proposed design framework mainly targets compute intensive 
%nested loops. As such, it is not a generic method to perform HLS on random logic.
%
%Secondly, the grouping strategy used in the framework makes the IO buffer partitioning 
%difficult, because the input/output data of the DFGs in the same group may be located in
%diverse partitioned banks and the DFG scheduling may not be reused among these DFGs. 
%This limitation will be fixed by introducing an additional buffering stage in future.
%Currently, input/output buffer partitioning is not supported in this framework and the 
%accelerators developed only have a single input buffer and a single output buffer. 
%
%Thirdly, data transmission and the computation are sequentially repeated to complete 
%the nested loop on the accelerator. However, it is possible to pipeline the two 
%processing stages for better performance while performance models need to be 
%adjusted accordingly.
%
\section{conclusion} \label{sec:conclusion}
In this work, we have presented an automatic nested loop acceleration 
framework based on a homogeneous SCGRA overlay built on top of 
off-the-shelf FPGA devices. Given high-level user constraints, 
the framework performs an intensive system customization specifically 
to a nested loop and provides a complete loop acceleration on a hybrid CPU-FPGA system.
Particularly, by taking advantage of the regularity of 
the SCGRA overlay, the complex customization problem is simplified and typically completes in 10
minutes to 20 minutes. According to the experiments, the resulting accelerators achieve up to 10X
speedup over a hard ARM processor on Zedboard. 

