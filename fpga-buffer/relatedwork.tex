\section{Related Work} \label{sec:relatedwork}
Overlay architecture which is a virtual intermediate architecture overlaid on 
top of off-the-shelf FPGA is increasingly applied as a way to address the 
productivity challenge. 

Various overlays with diverse configuration granularities and flexibility 
ranging from virtual FPGAs \cite{Grant2011Malibu, ZUMA2012}, 
array-of-FUs \cite{mesh-FUs,ferreira2011fpga}, soft 
CGRA \cite{kissler2006dynamically, scgra}, soft GPU \cite{Guppy2012GPU-Like}, 
vector processors\cite{Yiannacouras2009FPS, MXP2013} to 
configurable processors or multi-core processors \cite{unnikrishnan2009application, 
MARC2010, Yiannacouras2007Exploration, Capalija2009coarse-grain, OCTAVO2012, iDEA2012} 
have been developed over the years. SCGRA overlay provides unique 
advantages on compromising hardware implementation 
and performance for compute intensive nested loops as demonstrated 
by numerous ASIC CGRAs \cite{tessier2001reconfigurable, compton2002reconfigurable}.
Most importantly, it allows both rapid compilation by taking advantage of 
the overlays' tiling structure \cite{ROB2014} and efficient bitstream 
reuse within the design iterations of an application \cite{scgra}, 
thus it is particularly promising for high productivity nested loop acceleration.

Indeed, SCGRA overlays have many similarities in terms of array structure 
and scheduling algorithm with ASIC CGRAs. ASIC CGRAs emphasize 
more on configuration capability and limited customization is allowed due 
to the overhead constraints \cite{zhou2014application, miniskar2014retargetable} 
while SCGRA overlays allow more intensive architectural customization providing just enough hardware
to the target application or application domains 
because of the FPGA's inherent programmability. Moreover, hardware resources such as 
DSP blocks and RAM blocks available on FPGAs are discrete, which results in different 
design constraints for SCGRA overlay customization. 

The authors in \cite{colinheart} developed an SCGRA topology customization method using 
genetic algorithm and showed the potential benefits of the SCGRA overlay customization, 
but the rest of the system design parameters were not covered.
In order to achieve both high design productivity and high performance with low overhead,  
a complete nested loop acceleration framework targeting CPU-FPGA system 
is developed in this work. It supports intensive application-specific
customization including the overlay architectural customization, 
the compilation customization and communication interface customization 
for optimized performance. When the customized design parameters are determined, 
corresponding hardware accelerator and software can be compiled to the target 
CPU-FPGA system rapidly eventually providing a push-button solution for a nested loop 
acceleration. 


