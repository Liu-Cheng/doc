%%%%%%%%%%%%%%%%%%%%%%%%%%%%%%%%%%%%%%%%%
% Medium Length Professional CV
% LaTeX Template
% Version 2.0 (8/5/13)
%
% This template has been downloaded from:
% http://www.LaTeXTemplates.com
%
% Original author:
% Trey Hunner (http://www.treyhunner.com/)
%
% Important note:
% This template requires the resume.cls file to be in the same directory as the
% .tex file. The resume.cls file provides the resume style used for structuring the
% document.
%
%%%%%%%%%%%%%%%%%%%%%%%%%%%%%%%%%%%%%%%%%

%----------------------------------------------------------------------------------------
%	PACKAGES AND OTHER DOCUMENT CONFIGURATIONS
%----------------------------------------------------------------------------------------

\documentclass{resume} % Use the custom resume.cls style
\hfuzz=100pt
\usepackage{natbib, bibentry}
\usepackage[left=0.75in,top=0.6in,right=0.75in,bottom=0.6in]{geometry} % Document margins

\name{Cheng Liu} % Your name
\address{+65-8421-1303}
\address{Shenzhen, Guangdong, China} % Your secondary addess (optional)
\address{st.liucheng@gmail.com} % Your phone number and email

%----------------------------------------------------------------------------------------
%	EDUCATION SECTION
%----------------------------------------------------------------------------------------

\begin{document}

\begin{rSection}{Education}

{\bf The University of Hong Kong} \hfill {\em Sep 2011 - Apr 2016} \\ 
Ph.D in Computer Engineering \\
{\bf Harbin Institute of Technology} \hfill {\em Sep 2007 - Jul 2009} \\
M.E. in Electronic Engineering \\
{\bf Harbin Institute of Technology} \hfill {\em Sep 2003 - Jul 2007} \\
B.E. in Electronic Engineering \\

\end{rSection}

\begin{rSection}{Research Interest}
    \vspace{-1em}
\item FPGA Acceleration, Reconfigurable Computing, Hardware/Software Co-design, Network-on-Chip
    \vspace{-0.5em}
\end{rSection}
%----------------------------------------------------------------------------------------
%	WORK EXPERIENCE SECTION
%----------------------------------------------------------------------------------------
\begin{rSection}{Experience}
\begin{rSubsection}{School of Computing, National University of Singapore}{Dec
    2016 - present}{Project: Graph acceleration on heterogeneous CPU-FPGAs}{}
\item In this project, we aim to develop a high-performance graph accelerator framework for 
    large scale-free graph analytics on a heterogeneous CPU-FPGA computing platform which has 
    become important computing infrastructures in the cloud. Here are a few aspects that we 
    are focusing on at the moment. 1) Explore graph specific optimizations especially for the 
    memory access efficiency. 2) Automatically tune the graph accelerators for higher performance 
    or throughput given the memory bandwidth and other resource constraints. 
    3) Develop a flexible co-design of the graph accelerator that can dynamically 
    offload sub tasks from CPUs to FPGAs for higher memory bandwidth and computing resource utilization.  
\end{rSubsection}

\begin{rSubsection}{SenseTime Group Limited}{Apr 2016 - Oct. 2016}{Project: Convolutional neural network
    (CNN) acceleration on Zynq}{}
\item A flexible hardware accelerator for general convolution layer of CNN was developed. It allows
arbitrary tiling for various CNN configurations such as kernel size and stride size. In this project, 
I was also responsible for the CNN ip driver development on an embedded Linux system on Zc706.
\end{rSubsection}

\begin{rSubsection}{The University of Hong Kong}{Sep 2011 - Dec 2015}{Research Project: Promoting FPGA design productivity using soft CGRA overlay}{}
\item In this project, a rapid FPGA loop accelerator design framework was developed. By using a
    regular soft coarse-grained reconfigurable array (SCGRA) overlay as an intermediate layer, it is
    able to compile a high-level (C/C++) nested loop to an FPGA loop accelerator bitstream in
    seconds with pre-built empty SCGRA bitstream. In addition, the overlay allows intensive customization 
    specifically to an application or a domain of applications for the sake of performance with 
    a few manual optimization routines. To facilitate applying this framework on a CPU-FPGA system,
    we also provide a template for generating the communication interface between the CPU and 
    the resulting FPGA accelerator. The framework is demonstrated on Zedboard. The resulting 
    accelerators exhibit competitive performance over the software counterparts on the ARM processor. 
\end{rSubsection}

%------------------------------------------------

\begin{rSubsection}{Institute of Computing Technology, Chinese Academy of Sciences}{Sep 2009 - Mar 2011}{Research Project: Highly reliable network-on-chip design}{}
\item In this project, we aimed to enhance NoC reliability by innovating on NoC micro-architecture. With
    the observation that Vertical Through Silicon Vias (TSV) in a classical 3D NoC is
    under-utilized, a TSV sharing method was developed for homogeneous 3D NoC with negligible 2D area
    overhead. This can either be used to improve the reliability of 3D NoC or shrink the TSV consumption.
    For regular 2D NoC, a flexible data path salvaging scheme was proposed to build highly reliable NoC.

\end{rSubsection}

%------------------------------------------------

\begin{rSubsection}{Harbin Institute of Technology}{Oct 2006 - Jul 2009}{Research Project: High performance network-on-chip design}{}
\item In this project, we aimed to improve NoC performance as well as power efficiency. A number of design
optimization methods were implemented for this purpose: 1) A delicate virtual channel assignment was
used to improve NoC performance and remove the deadlock in adaptive routing. 2) A lightweight
dynamic virtual channel sharing architecture was implemented to reduce the chip area as well as the
power consumption. 3) With the observation that NoC components have quite low utilization, a dynamic
power gating scheme was developed to enhance NoC power efficiency.
\end{rSubsection}

\end{rSection}

%----------------------------------------------------------------------------------------
% Publications
%----------------------------------------------------------------------------------------
\begin{rSection}{Publications}
    \begin{rSubsection} {Book Chapter}{}{}{}
    \item Hayden Kwok-Hay So and \textbf{Cheng Liu}, "FPGA overlays", in press FPGAs for Software
        Engineers, Dirk Koch, Frank Hannig and Daniel Ziener, Ed., 2016.  
    \end{rSubsection}

    \begin{rSubsection} {Journal}{}{}{}
    \item Ying Wang, Yinhe Han, Lei Zhang, Binzhang Fu, \textbf{Cheng Liu}, Huawei Li, and Xiaowei Li.
        "Economizing TSV Resources in 3-D Network-on-Chip Design." Very Large Scale Integration
        (VLSI) Systems, IEEE Transactions on 23, no. 3 (2015): 493-506. (cited 12)  
    \item Yinhe Han, \textbf{Cheng Liu}, Hang Lu, Wenbo Li, Lei Zhang, and Xiaowei Li. "RevivePath:
        Resilient network-on-chip design through data path salvaging of router." Journal of Computer
        Science and Technology 28, no. 6 (2013): 1045-1053. (cited 3)
    \item Qingli Zhang, \textbf{Cheng Liu}, Liyi Xiao, Fangfa Fu.” Low Latency Router Design Supporting both
        Deterministic Routing and Adaptive Routing.” Journal of Computer-Aided Design \& Computer
        Graphics, 21(12), 2009 (in Chinese)
    \end{rSubsection}

    \begin{rSubsection} {Conference \& Workshop}{}{}{}
    \item Ho-Cheung Ng, \textbf{Cheng Liu} and Hayden Kwok-Hay So. "A Soft Processor Overlay with
        Tightly-coupled FPGA Accelerator," Overlay Architectures for FPGAs (OLAF), Second
        International Workshop on, pp. 1-6, Feb, 2016. 
    \item \textbf{Cheng Liu}, Ho-Cheung Ng, and Hayden Kwok-Hay So. "Automatic Nested Loop Acceleration on
        FPGAs Using Soft CGRA Overlay", FPGAs for Software Programmers (FSP), 2nd International
        Workshop on, Sep. 2015. (cited 11)
    \item \textbf{Cheng Liu}, Ho-Cheung Ng, and Hayden Kwok-Hay So. "QuickDough: A Rapid FPGA Loop
        Accelerator Design Framework Using Soft CGRA Overlay", Field Programmable Technology,
        International Conference on (FPT), Dec. 2015 (cited 5) 
    \item \textbf{Cheng Liu}, Lei Zhang, Yinhe Han, and Xiaowei Li. "Vertical interconnects squeezing in
        symmetric 3D mesh Network-on-Chip." In Proceedings of the 16th Asia and South Pacific Design
        Automation Conference, pp. 357-362. IEEE Press, 2011. (Cited 54)
    \item \textbf{Cheng Liu}, Lei Zhang, Yinhe Han, and Xiaowei Li. "A resilient on-chip router design
        through data path salvaging." In Proceedings of the 16th Asia and South Pacific Design
        Automation Conference, pp. 437-442. IEEE Press, 2011. (Cited 21)
    \item Yang Xu, Qing-li Zhang, Fang-fa Fu, Ming-yan Yu, and \textbf{Cheng Liu}. "NISAR: An AXI compliant
        on-chip NI architecture offering transaction reordering processing." In Proceedings of the
        7th International Conference on ASIC, pp. 890-893. IEEE, 2007. (Cited 23)
%    \item \textbf{Cheng Liu}, Lei Zhang, Yinhe Han, Xiaowei Li. "Dynamic Buffer Regulator for 3D Mesh
%        Network-on-Chip." Workshop of 3D Integration - Applications, Technology, Architechture, Design 
%        Automation and Test in Europe (DATE), Mar 2011.
    \end{rSubsection}

    \begin{rSubsection} {Poster}{}{}{}
    \item \textbf{Cheng Liu}, and Hayden Kwok-Hay So. "Automatic Soft CGRA Overlay Customization for
        High-Productivity Nested Loop Acceleration on FPGAs." In Field-Programmable Custom Computing
        Machines (FCCM), 2015 IEEE 23rd Annual International Symposium on, pp. 101-101. IEEE, 2015.
    \item \textbf{Cheng Liu}, Colin Lin Yu, and Hayden Kwok-Hay So. "A soft coarse-grained reconfigurable
        array based high-level synthesis methodology: Promoting design productivity and exploring
        extreme FPGA frequency." In Field-Programmable Custom Computing Machines (FCCM), 2013 IEEE
        21st Annual International Symposium on, pp. 228-228. IEEE, 2013. (cited 3)
    \end{rSubsection}

\end{rSection}

\begin{rSection}{Teaching Experience}
    \vspace{-1em}
\item Teaching Assistant – to Dr. V. Tam \\In ELEC1503, Object Oriented Programming 
and Data Structures, 2012 \\Performed the lab demonstrations and Q\&A upon request

\item Teaching Assistant – to Dr. Hayden So \\In ELEC3441 Computer Architecture, 2014-2015
\\Developed part of the homework, prepared the lab environment, Q\&A upon request, and graded homework as well as the exam papers.

\item Teaching Assistant – to Dr. Hayden So, Dr. Edmund Lam, and Dr. Kenneth K. Y. Wong 
\\In ENGG1203, Introduction to Electrical and Electronic Engineering, 2013-2014
\\Performed the lab demonstrations and Q\&A upon request. 

\end{rSection}

\begin{rSection}{Awards}
    \vspace{-1em}
\item Postgraduate Studentship, The University of Hong Kong, Sep 2011 - Aug 2015 
    \vspace{-0.5em}
\item First Class Scholarship, Harbin Institute of Technology, Sep 2007- Jul 2009
    \vspace{-0.5em}
\item Outstanding Postgraduate Student, Key Lab of Computing System and Architecture, ICT Dec 2010
    \vspace{-0.5em}
\item Second Award of National Postgraduate Mathematical Modeling Contest, May 2008
    \vspace{-0.5em}
\item Second Award of Electronic Design Contest, Heilong Jiang Province, Mar 2007
    \vspace{-0.5em}
\end{rSection}

\begin{rSection}{Skills}
    \begin{rSubsection}{Programming Languages}{}{}{}
    \item Proficient: SystemC, C/C++, Matlab, Verilog, VHDL
    \item Competent: CUDA, JAVA, Python
    \end{rSubsection}
\end{rSection}

\begin{rSection}{Reference}
\item Dr. Hayden Kwok-Hay So                        (hso@eee.hku.hk)
\item Dr. Jinxiang Wang                             (jxwang@hit.edu.cn)
\item Dr. Liyi Xiao                                 (xiaoly@hit.edu.cn)
\item Dr. Ying Wang                                 (wangying2009@ict.ac.cn)
\item Dr. Fangfa Fu                                 (fff1984292@hit.edu.cn)
\item Dr. Lei Zhang                                 (zlei@ict.ac.cn)
\item Dr. Xiaowei Li                                (lxw@ict.ac.cn)
\item Dr. Yinhe Han                                 (yinhes@ict.ac.cn)
\end{rSection}
\end{document}
