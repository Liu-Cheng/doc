The use of FPGAs as accelerators for compute-intensive loops has been demonstrated by numerous researchers as an effective solution to meet both the performance and energy requirements across many application domains. However, the design productivity of developing FPGA accelerators remains much lower compared to the use of a typical software development flow. Although the use of high-level design tools may partly alleviate this shortcoming, the lengthy low-level FPGA implementation process including synthesis, placing and routing dramatically as well as the complex design space exploration limits the number of compile-debug-edit cycles per day and hinders the widespread adoption of FPGAs. 

To address this design productivity problem, we have developed a rapid FPGA loop accelerator design framework called QuickDough. By utilizing a soft coarse-grained reconfigurable array (SCGRA) overlay built on top of off-the-shelf FPGAs, it compiles a high-level loop to the overlay through a rapid operation scheduling first and then generates the FPGA accelerator bitstream through a rapid integration of the scheduling result and a pre-built overlay bitstream. In addition, it also supports intensive application-specifc customization for the sake of energy efficiency of the resulting accelerators with moderate of design efforts, which makes the whole design framework accessible to users witout any hardware design experience. 

According to the experiments, QuickDough is able to produce accelerators in the order of seconds with pre-built library while achieving up to 9X performance speedup over the execution of the same software running on a hard ARM processor. With additional application-specific customization, energy-delay product can be reduced by xxx and performance can be further improved slightly.
