%%%%%%%%%%%%%%%%%%%%%%%%%%%%%%%%%%%%%%%%%
% Medium Length Professional CV
% LaTeX Template
% Version 2.0 (8/5/13)
%
% This template has been downloaded from:
% http://www.LaTeXTemplates.com
%
% Original author:
% Trey Hunner (http://www.treyhunner.com/)
%
% Important note:
% This template requires the resume.cls file to be in the same directory as the
% .tex file. The resume.cls file provides the resume style used for structuring the
% document.
%
%%%%%%%%%%%%%%%%%%%%%%%%%%%%%%%%%%%%%%%%%

%----------------------------------------------------------------------------------------
%	PACKAGES AND OTHER DOCUMENT CONFIGURATIONS
%----------------------------------------------------------------------------------------

\documentclass{resume} % Use the custom resume.cls style
\hfuzz=100pt
\usepackage[UTF8]{ctex}
\usepackage{fontspec}
\usepackage{natbib, bibentry}
\usepackage[left=0.75in,top=0.6in,right=0.75in,bottom=0.6in]{geometry} % Document margins

\name{刘成} % Your name
\address{+86-13713529934}
\address{北京市海淀区中关村南路6号,中科院计算技术研究所} % Your secondary addess (optional)
\address{liucheng@ict.ac.cn} % Your phone number and email

%----------------------------------------------------------------------------------------
%	EDUCATION SECTION
%----------------------------------------------------------------------------------------

\begin{document}

\begin{rSection}{教育}

{\bf 香港大学} \hfill {\em 2011-09 - 2016-04} \\ 
计算机工程专业,电机与电子工程系\\
{\bf 哈尔滨工业大学} \hfill {\em 2007-09 - 2009-07} \\
微电子学与固体电子学专业, 航天学院\\
{\bf 哈尔滨工业大学} \hfill {\em 2003-09 - 2007-07} \\
电子信息科学与技术专业,航天学院 \\

\end{rSection}

\begin{rSection}{研究方向}
    \vspace{-1em}
\item FPGA 计算加速, FPGA 虚拟化, 深度学习计算加速,软硬件协同设计, 片上网络
    \vspace{-0.5em}
\end{rSection}
%----------------------------------------------------------------------------------------
%	WORK EXPERIENCE SECTION
%----------------------------------------------------------------------------------------
\begin{rSection}{研究经历}
\begin{rSubsection}{中科院计算所}{2018-06 - 现在}{
        项目: 低功耗深度学习硬件加速器}{}
\item 主要进行低功耗的深度学习加速器体系结构、电路设计优化以及自动化定制等研究,
	开发适用于物端智能处理的超低功耗AI加速芯片,满足物端识别、分析等智能应用需求。
\end{rSubsection}


\begin{rSubsection}{计算机学院, 新加坡国立大学}{2016-12 - 2018-06}{
        项目: 异构CPU-FPGA系统上的图计算加速}{}
\item 本项目目标是开发一个面向大型图的软件可编程高性能图计算库,
    可以用于在弱耦合与紧耦合的CPU-FPGA计算系统上的图计算加速。
    目前已经在Alpha-Data加速卡上实现了一个BFS加速器的设计,并且
    可以根据FPGA片上资源,调整设计参数,实现性能的优化。
    同时也正在进行针对紧耦合的CPU-FPGA系统的一般性图计算优化研究。
	针对Intel的Harp系统也即Xeon-FPGA,提出一种以数据
	流处理为基础的软硬件协同设计方案,用于自动化的生成软硬件
	协同的图计算设计。
\end{rSubsection}

\begin{rSubsection}{商汤科技}{2016-04 - 2016-10}{
        项目: Zynq系统上CNN卷积加速器设计}{}
\item 设计了一个灵活可配置的卷积加速器,能够支持任意的分块以及参数的配置,
    目标是能够快速的配置,以满足不同场景下不同CNN模型加速的需求。在ZC706上实现,
    并用于大场景人脸识别的展示系统。
\end{rSubsection}

\begin{rSubsection}{香港大学}{2011-09 - 2015-12}{
        项目: 基于软CGRA的FPGA虚拟化研究}{}
\item 本项目的目标是提供一个FPGA的虚拟层,用于多种相似的计算核心加速逻辑的快速实现,
    相比通常的高层综合或者HDL的硬件加速方式,可以极大的提高设计效率。为此,我们在FPGA上构建了
    一个规则的CGRA虚拟层,然后将高层语言的计算核心,编译成数据流图,并进一步调度到
    指定的CGRA虚拟层上。这样不同但相似的计算核心,在整个设计过程中,可以复用CGRA虚拟层的
    硬件综合实现。此外,我们进一步针对Xilinx Zynq系统,开发了相应的软硬件接口模板,使得计算核心
    的加速对用户近乎透明,达到接近软件编译的效率,获得显著的硬件加速。
\\
\item 将CGRA虚拟层移植到一个简单的RISC-V处理器上,添加加速逻辑调用指令,
    分析了在通用处理器上增加可重构加速逻辑的优化潜力。 
\end{rSubsection}

%------------------------------------------------

\begin{rSubsection}{中科院计算所}{2009-09 - 2011-03}{
        项目: 高可靠的片上网络微结构研究}{}
\item 针对3D片上网络的TSV的可靠性问题,设计了TSV共享的片上路由器结构,
    增加了少量的逻辑,显著的提高了设计的可靠性。
\end{rSubsection}

%------------------------------------------------

\begin{rSubsection}{哈尔滨工业大学}{2006-10 - 2009-07}{
       项目: 高性能片上路由器微结构设计}{}
\item 优化片上网络路由器的流水线,使用tsmc 130 nm工艺,实现了800MHz片上路由器的设计.
    针对路由器中虚拟通道的开销大的问题,实现了动态虚拟通道共享的设计,减小芯片面积开销。
\end{rSubsection}

\end{rSection}

%----------------------------------------------------------------------------------------
% Publications
%----------------------------------------------------------------------------------------
\begin{rSection}{论文}
    \begin{rSubsection} {Book Chapter}{}{}{}
	\item Xuntao Cheng, \textbf{Cheng Liu}, Bingsheng He, "Emerging Hardware Technologies", In Book 
		Encyclopedia of Big Data Technologies, pp.1-5, Jan 2018
    \item Hayden Kwok-Hay So and \textbf{Cheng Liu}, "FPGA overlays", in press FPGAs for Software
        Engineers, Dirk Koch, Frank Hannig and Daniel Ziener, Ed., 2016.  
    \end{rSubsection}

    \begin{rSubsection} {期刊}{}{}{}
    \item Ying Wang, Yinhe Han, Lei Zhang, Binzhang Fu, \textbf{Cheng Liu}, Huawei Li, and Xiaowei Li.
        "Economizing TSV Resources in 3-D Network-on-Chip Design." Very Large Scale Integration
        (VLSI) Systems, IEEE Transactions on 23, no. 3 (2015): 493-506. (cited 12)  
    \item Yinhe Han, \textbf{Cheng Liu}, Hang Lu, Wenbo Li, Lei Zhang, and Xiaowei Li. "RevivePath:
        Resilient network-on-chip design through data path salvaging of router." Journal of Computer
        Science and Technology 28, no. 6 (2013): 1045-1053. (cited 3)
    \item Qingli Zhang, \textbf{Cheng Liu}, Liyi Xiao, Fangfa Fu.” Low Latency Router Design Supporting both
        Deterministic Routing and Adaptive Routing.” Journal of Computer-Aided Design \& Computer
        Graphics, 21(12), 2009 (in Chinese)
    \end{rSubsection}

    \begin{rSubsection} {会议}{}{}{}
	\item Dawen Xu, Kaijie Tu, Ying Wang, \textbf{Cheng Liu}, Bingsheng He, Huawei Li, FCN-Engine: 
		Accelerating Deconvolutional Layers in Classic CNN Processors, International Conference On 
		Computer Aided Design (ICCAD), 2018 (to appear)
    \item Ho-Cheung Ng, \textbf{Cheng Liu} and Hayden Kwok-Hay So. "A Soft Processor Overlay with
        Tightly-coupled FPGA Accelerator," Overlay Architectures for FPGAs (OLAF), Second
        International Workshop on, pp. 1-6, Feb, 2016. 
    \item \textbf{Cheng Liu}, Ho-Cheung Ng, and Hayden Kwok-Hay So. "Automatic Nested Loop Acceleration on
        FPGAs Using Soft CGRA Overlay", FPGAs for Software Programmers (FSP), 2nd International
        Workshop on, Sep. 2015. (cited 12)
    \item \textbf{Cheng Liu}, Ho-Cheung Ng, and Hayden Kwok-Hay So. "QuickDough: A Rapid FPGA Loop
        Accelerator Design Framework Using Soft CGRA Overlay", Field Programmable Technology,
        International Conference on (FPT), Dec. 2015 (cited 11) 
    \item \textbf{Cheng Liu}, Lei Zhang, Yinhe Han, and Xiaowei Li. "Vertical interconnects squeezing in
        symmetric 3D mesh Network-on-Chip." In Proceedings of the 16th Asia and South Pacific Design
        Automation Conference, pp. 357-362. IEEE Press, 2011. (Cited 55)
    \item \textbf{Cheng Liu}, Lei Zhang, Yinhe Han, and Xiaowei Li. "A resilient on-chip router design
        through data path salvaging." In Proceedings of the 16th Asia and South Pacific Design
        Automation Conference, pp. 437-442. IEEE Press, 2011. (Cited 23)
    \end{rSubsection}

    \begin{rSubsection} {Poster}{}{}{}
    \item \textbf{Cheng Liu}, and Hayden Kwok-Hay So. "Automatic Soft CGRA Overlay Customization for
        High-Productivity Nested Loop Acceleration on FPGAs." In Field-Programmable Custom Computing
        Machines (FCCM), 2015 IEEE 23rd Annual International Symposium on, pp. 101-101. IEEE, 2015.
    \item \textbf{Cheng Liu}, Colin Lin Yu, and Hayden Kwok-Hay So. "A soft coarse-grained reconfigurable
        array based high-level synthesis methodology: Promoting design productivity and exploring
        extreme FPGA frequency." In Field-Programmable Custom Computing Machines (FCCM), 2013 IEEE
        21st Annual International Symposium on, pp. 228-228. IEEE, 2013.
    \end{rSubsection}

\end{rSection}

%\begin{rSection}{教学}
%    \vspace{-1em}
%\item 助教 – to Dr. V. Tam \\In ELEC1503, Object Oriented Programming 
%and Data Structures, 2012 \\实验上机指导和答疑
%
%\item 助教 – to Dr. Hayden So \\In ELEC3441 Computer Architecture, 2014-2015
%\\协助设计基于RISC-V的课程作业和实验以及答疑.
%
%\item 助教 – to Dr. Hayden So, Dr. Edmund Lam, and Dr. Kenneth K. Y. Wong 
%\\In ENGG1203, Introduction to Electrical and Electronic Engineering, 2013-2014
%\\实验上机指导和答疑. 
%
%\item 助教 – to Dr. Bingsheng He 
%\\新加坡国立大学冬季课程 "Big Data Systems on Future Hardware" 负责 "System Optimizations and Performance Tuning for New Generation FPGAs"  部分, 2018 
%
%\end{rSection}
%
%\begin{rSection}{奖励}
%    \vspace{-1em}
%\item 博士奖学金, 香港大学, 2011-09 - 2015-08 
%    \vspace{-0.5em}
%\item 一等奖学金, 哈尔滨工业大学, 2007-09 - 2009-07
%    \vspace{-0.5em}
%\item 优秀研究生奖学金, 计算所系统结构重点实验室,2010-12
%    \vspace{-0.5em}
%\item 国家数学建模竞赛二等奖, 2008-05
%    \vspace{-0.5em}
%\item 黑龙江省电子竞赛二等奖, 2007-03
%    \vspace{-0.5em}
%\end{rSection}
%
%\begin{rSection}{技能}
%    \begin{rSubsection}{编程语言}{}{}{}
%    \item 熟悉: SystemC, C/C++, Matlab, Verilog, VHDL, HLS, OpenCL for FPGA
%    \item 了解: JAVA, Python
%    \end{rSubsection}
%\end{rSection}
%
\end{document}
