\section{Introduction}\label{sec:introduction}
Recent years have witnessed a tremendous growth in the use of FPGA 
accelerators in computer systems to improve the systems' performance 
and energy efficiency \cite{kim2014compression, majumder2014hardware, 
iouliia2004reconfigurable, souradip2010hardware, putnam2014reconfigurable}. 
Despite the great advantages, the use of FPGA accelerators in main-stream 
systems remains limited and has yet to receive widespread adoption beyond 
highly skilled hardware engineers \cite{cong2011high}. When compared to the 
wide adoption of GPUs and Xeon Phi accelerators, it is believed that the much 
lower design productivity in developing FPGA applications, which is generally 
caused by both the lengthy FPGA implementation and notorious inaccessibility 
to software programmers, has become one of the major obstacles that hinder the 
software developers using FPGA as compute accelerators. 

In order to address the FPGA accessibility problem, recent advances 
in high level synthesis tools have significantly raised the abstraction level 
of FPGA design, allowing users to effectively express hardware designs using 
familiar software programming languages such as C/C++, Java, Python and Scala. 
However, despite the continuously evolved FPGA compilation algorithms and 
design methods in the past decades, the lengthy FPGA implementation process 
including synthesis, placing and routing which typically takes dozens of minutes 
for a moderate design and upwards to hours or even days for some of the 
largest design remains a major productivity hurdle to most software developers. 

To address this compilation challenge, researchers opt to build FPGA overlays on top of 
the off-the-shelf FPGAs. Instead of compiling the user design directly to the FPGAs, 
the user design is compiled to the overlay while the overlay is pre-built on the FPGA. 
Since the overlays usually have higher hardware abstraction, the compilation step 
from user design to the overlay is usually very fast. The overlay implementation 
remains slow, but it can be reused through the design iterations or among different 
applications and thus it is amortized. Accordingly, the overall compilation time is 
reduced. Following this route, many FPGA overlays 
have been built and explored in the past few years. In this work, 
we are interested in significantly improving the speed of generating 
FPGA accelerators for DFGs extracted from compute intensive loops   
while maintaining a competitive overall performance of the resulting 
FPGA accelerated computing system.

To that end, we have developed an open sourced SCGRA overlay. 
 
that rapidly generates loop accelerators and their associated software-hardware communication interfaces. By utilizing a soft coarse-grained reconfigurable array (SCGRA) overlay as an intermediate architecture implemented on top of the physical FPGA, we pre-build an accelerator library with a group of various configurations on top of the SCGRA overlay. With the pre-built library, QuickDough generates the loop accelerator by selecting an accelerator configuration from it, schedules the compute operations from the user-provided loop onto the accelerator, and finally updates the pre-built accelerator configuring the target FPGA, which essentially avoids the FPGA implementation process during the accelerator design iterations. By employing different configuration selection algorithms, QuickDough allows users to perform trade-off between performance and compilation time. At the end of the selection process, optimized communication interfaces will be produced accordingly. Meanwhile, QuickDough also helps the users to automatically pre-build an overlay based accelerator library targeting a single application or a group of similar applications. To expedite the library generation process, a small representative set of accelerator configurations are chosen as the library and generated automatically using a template based system.

The contributions of this work is summaried as follows:
\begin{itemize}
    \item We present an open sourced CGRA overlay library. The library is compatitable 
        to the SDAccel environment. It is portable and can be accessible to software designers.
    \item We present a noval CGRA overlay library management framework such that 
        the CGRA overlay implementations can be shared and repeated hardware implementations can be avoided.
    \item We present a highly pipelined CGRA overlay template and optimized CGRA overlay instances can 
        be generated rapidly.
\end{itemize}

In \secref{sec:framework}, the proposed FPGA acceleration design framework QuickDough including both the SCGRA overlay implementation and compilation will be elaborated. Then the automatic SCGRA overlay based accelerator library pre-building will be detailed in \secref{lib-build}. Experimental results are shown in \secref{sec:experiments}. Finally, we will discuss the limitations of current implementation in \secref{sec:discussion} and conclude in \secref{sec:conclusions}.
