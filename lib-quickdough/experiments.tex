\section{Experiments}\label{sec:experiments}
With an objective to improve designers' productivity in developing FPGA accelerators, the key goal
of QuickDough is to reduce FPGA loop accelerator development time for a hybrid CPU-FPGA system.
By using 32 typical loop kernels as the benchmark, we have evaluated the FPGA accelerator generation 
time with QuickDough. Meanwhile, to warrant the merit of such framework, the performance of the 

generated acceleration system should remain competitive. For that purpose, the performance is then 
compared against to that of software executed on an ARM processor. Finally, the pre-built accelerator 
library that affects both the design productivity and overhead of the resulting accelerators is also
discussed.

%The experiment section is organized as follows. We will first briefly introduce the benchmark programs in the following subsection and explain the basic experiment setup in \secref{subsec:setup}. Then we will discuss the accelerator library update in \secref{subsec:lib-update}. Finally, we will elaborate the loop accelerator generation time, performance and implementation overhead in \secref{subsec:acc-gen}, \secref{subsec:acc-perf} and \secref{subsec:acc-impl} respectively. 



