The design productivity of FPGA development which remains magnitudes lower compared to typical software development severely hinders the widespread adoption of FPGAs. Particularly, the lengthy low-level FPGA implementation process including synthesis, placing and routing dramatically limits the number of compile-debug-edit cycles per day and lowers the FPGA design productivity. To address this design productivity problem, we have developed a rapid FPGA loop accelerator generation framework called QuickDough. Instead of trying to reduce the implementation time, it reuses a pre-built accelerator library to avoid the lengthy implementation process during design iterations. By utilizing a soft coarse-grained reconfigurable array (SCGRA) overlay built on top of off-the-shelf FPGAs as the backbone of the accelerators in the library, it compiles a high-level loop to the FPGA through a rapid operation scheduling first and then generates the FPGA accelerator bitstream through a rapid integration of the scheduling result and a pre-built accelerator bitstream selected from the library. According to the experiments, QuickDough is able to produce accelerators in the order of seconds while achieving up to 9X performance speedup over the execution of the same software running on a hard ARM processor.  
