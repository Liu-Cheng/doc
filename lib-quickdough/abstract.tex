Despite the great advantages of FPGAs on performance and energy efficiency,
the use of FPGAs remains limited compared to that of CPUs and GPUs. 
It is believed that the lengthy compilation, which is typically orders of 
magnitude slower than the software compilation on CPUs and GPUs, has become 
one of the major obstacles that hinder the widespread adoption of FPGAs.   
To address the this problem, FPGA overlays built on top of the off-the-shelf 
FPGAs are proposed recently. By using the overlay as an intermediate layer, 
applications can be compiled to the overlay rapidly while the lengthy overlay 
implementation can be performed in advance and reused during the design 
iterations or even among domains of applications. 

Along this route, we create an open sourced soft coarse-grained reconfigurable array (SCGRA) 
overlay targeting large data flow graphs (DFGs) extracted 
from the compute-intensive loop kernels. To make best use of the overlay for rapid 
accelerator design, we further propose a library based accelerator design 
framework named QuickDough. With comprehensive experiments, we analyze the 
compilation time as well as the run-time reconfiguration time of the 
SCGRA overlay based FPGA accelerator design. Finally,  
we wrap up the SCGRA overlay based accelerator as an OpenCL kernel 
and demonstrate a complete SCGRA overlay based acceleration solution 
on a typical CPU + FPGA card computing system, which exhibits the potential 
of the SCGRA overlay based accelerator design comprehensively. 

