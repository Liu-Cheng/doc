%%%%%%%%%%%%%%%%%%%%%%preface.tex%%%%%%%%%%%%%%%%%%%%%%%%%%%%%%%%%%%%%%%%%
% sample preface
%
% Use this file as a template for your own input.
%
%%%%%%%%%%%%%%%%%%%%%%%% Springer %%%%%%%%%%%%%%%%%%%%%%%%%%

\preface

%% Please write your preface here
If your computer crashes, you can revive it by a reboot, an empirical solution that usually turns out to be effective. The rationale behind this solution is that transient faults, either in hardware or software, can be fixed by refreshing the machine state. Such a “silver bullet”, however, could be futile in the future because the faults, especially those existing in the hardware such as Integrated Circuit (IC) chips, cannot be eliminated by refreshing. What we need is a more sophisticated mechanism to steer the system back to the right track. The “magic cure” is the on-chip fault-tolerant mechanism, which relies on a suite of built-in design-for-reliability logic, including fault detection, fault diagnosis, and fault recovery, working in a unified manner. 

In the past decade, we have successfully applied the proposed fault-tolerant computing mechanism onto various chip designs including generic circuits, general purposed processors, network-on-chips, and deep learning processors and gradually formulate an systematic built-in on-chip fault-tolerant computing paradigm, which can be utilized to guide IC designs against faults caused by manufacture defects, radiation particles, or progressively aging. In addition to the basic fault detection, fault diagnosis, and fault recovery, the proposed on-chip fault-tolerant computing paradigm also provides attractive benefits, such as facilitating graceful performance degradation, mitigating the impact of verification blind spots, and improving the chip yield.

In this book, we mainly illustrate the built-in on-chip fault-tolerant computing paradigm with demonstrations on genetic circuits, general purposed processors, network-on-chips, and deep learning processors. The entire book consists of 6 chapters. Chapter 1 presents the background of fault-tolerant chip designs and overview of the fault-tolerant computing paradigm. Chapter 2, Chapter3, Chapter 4, and Chapter 5 demonstrate the use of the proposed fault-tolerant computing paradigm on specific areas of chips including generic circuits, general purposed processors, network-on-chips, and deep learning processors respectively. Chapter 6 concludes this book with a brief summary of this book and discussion of future directions.

The techniques involved in this book are collected from thesis of Guihai Yan, Lei Zhang, Songjun Pan, Cheng Liu, Wen Li supervised by Prof. Xiaowei Li. All the relevant materials are already published in the leading conferences and journals in VLSI design and EDA. Prof. Xiaowei Li organized this book in general, Prof. Cheng Liu finished Chapter 1, Chapter 4, Chapter 5, and Chapter 6, Prof. Guihai Yan mainly finished Chapter 2 and Chapter 3. Prof. XX and Prof. XXX reviewed this book and Prof. Tim Cheng wrote Foreword for this book. All the efforts are greatly appreciated.

The techniques presented in this book are partly selected from research founded by XXX, XXX, XXX, and XXX. 

\vspace{\baselineskip}
\begin{flushright}\noindent
\leftline{Institute of Computing Technology, Haidian, Beijing} \\
May 2022\hfill {\it Xiaowei Li}\\
\end{flushright}


