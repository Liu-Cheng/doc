\section{Related Work} \label{sec:relatedwork}
In this section, we will review the neural network training for CNN accelerators first.
Then we will discuss the various approaches that design constraints can be relaxed for
performance or energy efficiency. 

Numerous hardware accelerators especially the CNN accelerators have
been developed in the past few years. \cite{Cnvlutin_25} 
\cite{deepburing_12}, \cite{Aydonat_27}, 
\cite{Caffeine_6}, \cite{Wei_29}. To 
deploy the neural network models on the accelerators efficiently, 
the models need to be trained for the target accelerator. 
The neural network training is mostly performed on GPPs with floating point.
When the floating point models are quantized and deployed directly 
on the CNN accelerators with low computing precision, the prediction 
accuracy of the models may drop\cite{Matthieu2014_8, Hwang2014_17}. 
Many offline converting or online tuning approaches are proposed to 
ensure the neural network models executed on the CNN accelerator 
with little accuracy loss\cite{BinaryConnect_14,courbariaux2014}. 

There are also efforts spent to use FPGAs for training. 
Caffeine\cite{Caffeine_6} provided a general CNN accelerator design and integrated it with 
the training framework Caffe. FCNN\cite{fcnn_5} implemented FPGA kernels 
for both forward propagation and backward propagation. It can also be used in Caffe 
and allows both training and inference. These work focus on the performance optimization 
and fine-tuning for quantized neural network models. 
They also support training with CNN accelerators, but they do not take the general 
computing errors of the CNN accelerators into consideration and compromise this 
feature for computing benefits through relaxed design constraints.

Hardware design typically requires strict design constraints to ensure 
reliable behavior. However, many applications such as image processing 
can tolerate computing errors. In this case, design constraints can be 
relaxed and great performance or energy efficiency benefits can be achieved 
with little negative influence. Near-threshold circuit\cite{M2017NT,BH2005} and SRAM\cite{G2010SRAM,SA2008SRAM}, 
and overclocking \cite{overclock_3} have been explored and demonstrated 
to be promising approaches. Neural networks are also resilient and 
offers tremendous design optimizations by relaxing the design constraints. 
Nevertheless, the computing errors incurred by the relaxed design constraints 
may cause considerable accuracy loss when the offline 
trained neural network models are deployed on the accelerators directly. 
The benefits are relatively limited. Therefore, exploring the resilience 
of neural network models is highly demanded, which can provide more 
advantageous design trade-offs. 
