\section{Related Work} \label{sec:relatedwork}
\textbf{Approximate computing:}Approximate computing is a promising technique for efficiency optimization\cite{Miao_40,han_41}.
Diverse techniques have been explored in prior work that apply approximate computing approaches to improve NN energy efficiiency.
Minerva \cite{Reagen_42}is an example that uses circuitlevel techniques to handle memory error in NN accelerators,
employing Razor sampling circuits for fault detection and equipping the weight fetch stage with bit masking and word
masking for flipped weights to mitigate bit-flip errors caused by NTV-based weightstorage.
Recent research proposes more explicit techniques to exploit NN’s intrinsic error tolerance and flexibility
during training to improve efficiency. For example, both AxNN and ApproxAnn take neuron criticality into
consideration and perform periodical retraining for self-healing \cite{AxNN_43,approxANN_44}. AxNN proposes the characterization
of neuron criticality first, then the replacement of non-critical neurons with their approximate versions.
To ensure targeted accuracy, iterative retraining is used for error recovery. Inspired by AxNN,
ApproxAnn proposes a more reliable way to quantify neuron criticality and adopts iterative heuristics to gain maximum efficiency.
However,the approximation is performed on a pre-trained network with hardware-agnostic training, which does not optimize
for error tolerance, so the target accuracies in their designs are met with relatively conservative approximations.

\textbf{CNN accelerator:} There have been notable efforts made to create hardware accelerators of 
machine learning algorithms for the sake of higher performance and energy-efficiency \cite{Cnvlutin_25} 
in the past few years. Among the accelerators, the regular 2D array architecture has become 
a mainstream solution because of the relatively higher PE and bandwidth utility. Runtime reconfigurable 
PE arrays are applied to provide customized solutions for efficient CNN inference on FPGAs \cite{Caffeine_6,deepburing_12}. 
In \cite{Aydonat_27}, an array of processing elements (PEs) with novel architecture was developed. With intensive 
data reuse, it reduces the external memory bandwidth requirements dramatically and outperforms 
the systolic-like structure proposed in \cite{Caffeine_6}. Compared to the compact hardware design in \cite{Caffeine_6,Aydonat_27}, 
Wei X et al. in \cite{Wei_29} implemented a high-throughput CNN design and did comprehensive design space exploration 
on top of accurate models to determine the optimal design configuration.

\textbf{Training of accelerators:} Training approaches for CNN accelerator can be classified into three 
categories: (1) convert a pretrained floating point CNN model into a fixed point model without 
training, (2) train a CNN model with fixed point constraint, and (3) FPGA-implemented forward \& backward propagation 
training tools. For first category, \cite{Yunchao_19} applied codebook based on scalar and vector quantization methods 
in order to reduce the model size. \cite{Cnvlutin_25} analyzed the quantization sensitivity of the network for each layer 
and then manually decide the quantization bit-widths. \cite{Hwang2014_17} find direct quantization for fixed-point 
network design does not yield good results and optimized the fixed-point design by employing back propagation 
based retraining. \cite{Matthieu2014_8} adapted a higher precision for the parameters during the updates than during 
the forward and backward propagations for accumulating small changes in the parameters. \cite{Hwang2014_17} used only binary 
weights to train deep neural networks. 

  However, these approaches of the former two categories are not suitable for ‘unstable’ circuit. 
For the third category, FCNN\cite{fcnn_5}  reconfigured a streaming data path at runtime to cover the training cycle 
for the various layers in a CNN. Caffeine\cite{Caffeine_6}  provides tunable parameters, including the number and 
size of input/output feature maps, shape and strides of weight kernels, pooling size and stride, 
ReLU kernels, and the total number of CNN/DNN layers. Caffeinated FPGA\cite{DiCecco_4} implemented FPGA kernels 
for forward and backward for Caffe and these kernels target the Xilinx SDAccel OpenCL environment 
for training and inference with CNNs. However, these approaches did not consider the unstable 
hardware behavior into their framework or either gave a way to train CNN under the un-deterministic situation.

\textbf{Unstable Hardware Behavior:} Overclocking, soft Error, circuit defect induced by process 
variation etc. result in the un-determined behavior of the circuits.  Overclocking, a technique to gain 
the additional performance from a given component by increasing its operating speed, may cause timing error. \cite{overclock_3} gave 
the strands of research of arithmetic precision determination and overclocking. Razor\cite{Razor_15}  projected scaled 
the supply voltage and clock frequency beyond the most conservative value. Soft errors are unintended 
transitions of logic state in a circuit typically caused external source of ionizing radiations. 
The shrinking transistor sizes increased the soft-errors. \cite{Mansour_20} proposed An Automated SEU 
Fault-Injection Method and Tool for HDL-Based Design. \cite{Subasi_30} inject single-bit flips into the register-transfer 
level descriptions of floating-point ALUs.


