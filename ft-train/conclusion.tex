\section{Conclusion} \label{sec:Conclusion}
In this paper, we propose to take the CNN accelerator’s dynamic behaviors into consideration 
at training and have the CNN model to learn the accelerator’s behaviors. To that end, we further build 
an training system based on Caffe on a hybrid CPU-FPGA architecture. Then use the training 
system to deal with an approximate CNN accelerator and an overclocked CNN accelerator 
and an accelerator with soft errors. According to our 
experiments, the top1 and top5 prediction accuracy can be inproved to 14.4\% and 16.1\% respectively 
on the worst case when use in approximate CNN accelerator.
The proposed training can alse improve the top1 and top 5 prediction accuracy 
of four CNN models up to 13.7\% and and 11.6\% respectively when 
the CNN accelerator is overclocked on the extreme situation. What's more, this method is beneficial to the CNN 
accelerators with soft errors. In the case with most soft errors, it improves the prediction accuracy up 
to 11.1\% and by 3.4\% . The disadvantage is the much longer training time due to the frequent 
data transfer between host memory and device memory. This problem can be resolved when porting the system 
to closely coupled CPU-FPGA architectures with shared memory.

%\appendix
%\section{Acknowledgement}

%\begin{acks}
%  The authors would like to thank Sam Ho for providing the suggestions on
%  HLS design debugging and optimization as well as the SDAccel usage. 

%\end{acks}
