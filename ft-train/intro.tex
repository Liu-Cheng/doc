\section{Introduction} \label{sec:intro}
Inspired by the widespread adoption of neural networks in massive fields such as image classification, 
video surveillance, speech recognition, and robot vision, neural network accelerators 
\cite{pipecnn_2,Zhang2015_9,Qiu2016_10,deepburing_12,Farabet2010_13,Zeng2018_18} 
are increasingly explored and deployed to improve the computing performance and energy efficiency.
Unlike generic applications, neural networks usually involve redundancy and are known to be 
fault tolerant\cite{Reagen2016}. By taking advantage of this feature, many neural network accelerator optimizations 
such as neural network pruning and low-precision quantization can be utilized to improve 
performance and energy efficiency notably with minor inference accuracy penalty\cite{Han2016DeepCC}. 

In line with these optimizations, we opt to relax the design constraints of 
the neural network accelerators, which provides a unique way to achieve notable 
improvements on performance or energy efficiency with small inference accuracy loss. 
For generic hardware design, strict design constraint is typically required to 
guarantee correct computing under even the most severe environments. On the contrast, 
neural network accelerators are more resilient and less sensitive to computing 
errors incurred by timing violations. Given relaxed design constraints, many 
aggressive hardware optimization techniques can be applied with computing errors. 
For instance, emerging techniques such as  near-threshold voltage regime\cite{RG2010NT} 
and subthreshold digital logic design\cite{BH2005,B2006} promise high energy efficiency 
but suffer instability and poor yield. Conventional neural network accelerator 
can be pushed to operate at higher clock frequency with timing violations
\cite{overclock_3,Paceline_15}.
Lowering vdd to have the accelerator operated at near-threshold voltage 
also improves the energy efficiency. At the fab level, design rules may 
be aggressively pushed to reduce the expense. 

Motivated by the great advantages of relaxed design constraints,
we further explore the use of neural network resilience for more 
effective design trade-offs. When there are timing violations
and computing errors in the accelerator, the unmodified neural networks 
executed on the accelerators during inference is different 
from that computed on GPPs during the training, which may cause clear 
prediction accuracy loss. Instead of deploying the unmodified neural 
network models on the accelerators directly, we borrow the retraining 
idea from prior neural network quantization work \cite{Hwang2014_17,Matthieu2014_8} 
and have the deep neural network models to learn and tolerate the computing errors.  
Basically, we have the forward computing performed on the accelerator and 
then transfer the computing results to the host processor for 
backward propagation. With this approach, both application data and computing 
errors are learned and incorporated in the neural network models.  
Meanwhile, we define a set of standard interfaces to make it convenient 
to integrate general CNN accelerators into the retraining framework. 

In addition, we notice that some of the neural network layers are more sensitive to the 
computing errors and the sensitive layers dramatically limit the usefulness of the retraining. 
Thus, we schedule the most sensitive layer which is usually the last 
layer of the neural networks to host processors to reduce the negative influence 
of the computing errors. With both the retraining and sensitive layer protection, 
the neural networks become more resilient to the computing errors caused by 
the aggressive design options such as near-threshold logic or overclocking.
Compared to the original neural networks, the prediction accuracy of top-1 and top-5 
improves by xxx and xxx on average. The contributions of this work are 
summarized as follows.

\begin{itemize}
	\item We propose to improve the fault-tolerance of neural networks and make use of it to relax 
		the accelerator design constraints for higher performance or energy efficiency.

	\item We propose a neural network training framework to obtain resilient neural network models. 
		By integrating accelerator into conventional training, we have the computing errors 
		learned with the application data. By protecting the layer with most large errors, 
		we can further improve the resilience of the neural networks and pose more 
		opportunities to hardware optimizations.

	\item With comprehensive experiments, we show that the proposed training framework 
		could enhance the prediction accuracy of neural networks significantly 
		when the accelerators run with computing errors incurred by either 
		overclocking or lower voltage.
\end{itemize}
The paper is organized as follows. Section II analyzes the influence of 
the CNN accelerator computing errors on the neural network prediction accuracy. 
Section III presents the proposed training for neural networks executed on accelerators with computing errors.
Section IV demonstrate the use of the training on accelerators with overclocking and generic computing errors. 
Section V briefs the related work and Section VI draws the conclusion. 


