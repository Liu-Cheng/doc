\section{Introduction} \label{sec:intro}
In recent years, deep neural networks especially the convolutional neural networks (CNN) 
have shown great performance in massive fields such as image classification, 
video surveillance, speech recognition, and robot vision. To ensure both 
higher energy-efficiency and lower processing latency, various CNN accelerators 
\cite{pipecnn_2,Zhang2015_9,Qiu2016_10,deepburing_12,Farabet2010_13,Zeng2018_18} 
have been developed and are increasingly deployed in the diverse applications. 
With the CNN accelerators, a neural network is usually trained 
using frameworks like Caffe, Pytorch and Tensorflow on general purposed 
processors (GPPs) first. After the training, the resulting neural network 
model is further quantized and compiled to the target CNN accelerator. 

The implicit assumption of the offline training is that the neural 
network model executed on GPPs can produce equivalent result to 
that on the CNN accelerators. Nevertheless, there are 
many scenarios where the computing on CNN accelerators 
is different from that is calculated with GPPs. For instance, the 
CNN accelerator may be implemented with approximate arithmetic logic 
such as \cite{Approximate_Multiplier_31,Approximate_32,approximate_33} for 
the purpose of higher performance or energy efficiency. 
Overclocking  \cite{overclock_3,Paceline_15,overclock_35} which allows certain 
timing errors to enable higher clock frequency
is another occasion, and is already supported in commercial chips 
or FPGA designs \cite{overclock_retain_34,overclock_Algorithm_36}. Some of the CNN accelerators may be 
exposed in error-prone environment where soft errors 
may affect the circuit behavior randomly. 
In these scenarios, it is almost impossible to have the offline training 
framework to aware the exact accelerator behavior. 
If the training framework ignores the accelerators’ dynamic 
behavior, the prediction accuracy of the resulting neural 
network model may degrade dramatically, which is expected according to 
the neural network training theory.

To address the above problems, an intuitive approach is to 
constrain the accelerator design to ensure the neural 
network computing result to be close to that obtained from GPPs. 
In this case, the approximate strategy, overclocking scheme and 
fault tolerant design method must be conservative. 
While deep neural networks typically have the potential to 
learn and tolerate errors or variation in computing as demonstrated in 
numerous CNN quantization work \textbf{xxx}, we opt to 
integrate the CNN accelerator to the training framework 
and retrain the neural network models specifically for the 
targeting CNN accelerator or scenario such that the undeterministic 
accelerator behavior can be learned and tolerated without 
modifying the accelerator design or allowing more aggressive 
hardware design at least. 

Along this idea, we develop an on-accelerator training framework 
on top of Caffe to retrain the neural network models 
running on CNN accelerators with undeterministic behavior. 
In particular, we adopt the knowledge 
distilling approach, which essentially has a general and bigger
neural network to teach a smaller or similar neural network 
targeting at a specific scenario, for the fine tuning. 
Basically, we have the golden neural network model 
to guide the fine tuning on the CNN accelerator with undeterministic 
computing. This ensures more reliable fine tuning convergence.  
In addition, we explore the fine tuning strategy and find 
that the batch size of the training can be a critical hyper parameter 
especially for neural networks on CNN accelerators with dynamic variation. 

The framework is implemented on a heterogeneous CPU-FPGA platform where 
CNN accelerator stays on FPGA. Runtime computing result of the 
CNN accelerator with undeterministic variation is sent back to 
host CPU for the fine-tuning. Meanwhile, we define a set of standard 
interfaces to make it convenient to integrate general CNN accelerators 
into the fine tuning framework. Finally, we take an approximate CNN accelerator,
an overclocked CNN accelerator and faulty CNN accelerator as examples 
and demonstrate the usefulness of the on-accelerator training framework on 
these representative scenarios. The contributions of this work are 
summarized as follows.

\begin{itemize}
	\item We propose an on-accelerator neural network training framework to 
		have the dynamic computing variation of the CNN accelerators to be 
		learned with the application data. With this method, neural 
		network models can be executed on these CNN accelerators with 
		undeterministic behavior without hardware modification.

	\item We build an end-to-end on-accelerator training framework on top of Caffe
		and apply the knowledge distilling method for fine tuning neural networks 
		executed on CNN accelerators with undeterministic behavior. Meanwhile, we 
		define a set of interfaces to allow general CNN accelerators 
		to make use of the training framework.

	\item We explore the use of the on-accelerator training on CNN accelerators 
		with approximate arithmetic logic, overclocking and soft errors and 
		demonstrate the usefulness of the proposed system on a set of 
		representative neural networks.
\end{itemize}
The paper is organized as follows. Section II analyzes the influence of 
the CNN accelerator's undeterministic behavior when using the offline 
training and motivates this work. Section III presents the 
proposed on-accelerator training framework and 
details the interface of integrating general CNN accelerators within the 
training framework. Section IV performs three typical case studies using the 
proposed training framework and demonstrates the use of the on-accelerator 
training. Section V briefs the related work and Section VI draws the conclusion. 


