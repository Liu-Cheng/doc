\section{Introduction}
Improving general-purpose processing system is getting extremely 
difficult. More and more computer architects believe that the major 
improvements in cost-energy-performance will come from domain-specific 
hardware accelerators. Recent years have already seen a number of successful 
demonstrations utilizing domain specific hardware accelerators for critical 
domains of applications such as deep neural network \cite{Jouppi2017tpu, Li2017survey} 
database operations \cite{Wu2014q100} and graph processing \cite{Jun2016graphicionado, Ozdal2016energy}. 
In order to explore the hardware accelerator design, a hardware accelerator simulator 
is usually required. Indeed there are already many exisitng tools \cite{systemc, chisel} and 
models \cite{dramsim2, ramulator} that can be used to help with the hardware accelerator 
design, it is non-trivial to develop a hardware accelerator on top of these work. For instance, there 
is a lack of general public cycle-accurate memory models available in \cite{systemc, chisel} while 
\cite{dramsim2, ramulator} expose only primitive memory access interface and need to be further 
wrapped for an accelerator simulator. And a general accelerator simulator 
framework is highly desired for the hardware accelerator simulator development.

Despite the difference of the accelerator simulators, we argue that a general 
accelerator simulator design framework should have three common yet important 
features. First of all, it should provide memory models of various memory 
architectures. Basically memory is usually critical to the hardware accelerator 
and greatly affects the accelerator design. At the same time, memory techniques evolve rapidly 
over the years and novel memory architectures with distinct features emerge. In order to explore 
hardware accelerator design, various memory architectures needs to be evaluated. 
Secondly, it should provide abstract user-frinedly memory interfaces. Hardware accelerators 
usually have complex memory access patterns such as stream access, burst access as well as random access. 
Thus higher abstract memory access interface instead of primitive memory access interface should be provided. 
Thirdly, it should provide trade-off between simulation speed and precision. Hardware accelerators 
may have distinct simulation speed and precision requirements while exploring the hardware accelerator. 
For instance, some of the applications such as graph accelerators may process on a big data set. 
Low-level accurate memory model may result in extremely long simulation. Thus a simplified memory model 
should be used to obtain the general performance of the accelerators. For applications that are sensitive 
to the memory access latency, more accurate memory models are preferred.

There is still a lack of general accelerator simulator framework that fullfills 
all the three features mentioned above. To that end, we proposed a flexible hardware accelerator 
simulation framework to be reused for general hardware accelerator simulator development. Basically, it 
integrates ramulator supporting various memory architectures as the underlying memory model and thus allows 
hardware accelerator exploration over a broad range of memory architectures. In addition, abstract memory 
interfaces as well as memory content management are provided to faciliate the accelerator accessing 
the memory model. Finally, it also provides a mix of cycle-accurate memory model and simiplified 
analytical memory model obtained though sampling to compromise on simulation speed and accuracy.

The rest of the paper is organized as follows. Section 2 briefly reviews the related work. 
Section 3 presents the proposed accelerator simulation framework. Section 4 provides the 
experimental results and Section 5 concludes this paper.





