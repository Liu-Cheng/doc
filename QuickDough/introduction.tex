\section{Introduction}\label{sec:introduction}

By raising the abstraction level of the physical FPGA hardware, numerous researchers have
demonstrated the potential of utilizing FPGA overlay architectures as a way to improve designers'
productivity \cite{Lebedev2010,kissler2006dynamically,unnikrishnan2009application,Yiannacouras2009FPS,Guy2012VENICE,Jeffrey2011potential}.
Overlay architectures, similar to overlay networks in conventional network designs, are intermediate
virtual architectures that are overlaid on top of the physical FPGA configurable fabric with
purposes such as to improve design portability, security, and designer's productivity.
Overlay architectures may be utilized at different stages of application implementation and have thus manifested in a number of different forms.
Examples of overlays developed for such purposes range from parametric HDL models, pre-synthesized or pre-implemented circuits, to multicore processors or even arrays of coarse-grained processing units connected by an on-chip network.
In practice, the additional overlay layer may in some cases result in implementations with less than optimal performance when compared to their hand-optimized alternatives.
Nevertheless, the use of overlays remains highly anticipated as many early works have already demonstrated their potential as techniques to improve portability, security, and most importantly, productivity of application designers using FPGA-based reconfigurable computers.

In this work, the QuickDough rapid compilation and synthesis framework for applications targeting hybrid CPU-FPGA systems is presented.  The goal of QuickDough is to provide a high-productivity compilation framework for applications that execute on both CPU and FPGA during run time.
In particular, it automates the process of generating hardware accelerators and their associated CPU-FPGA communication infrastructure for loop kernels that are designated for FPGA acceleration by the user.
By utilizing a soft coarse-grained reconfigurable array (SCGRA) as an overlay, QuickDough is able to generate the configuration bitstream for the targeted platform rapidly in the order of seconds.

QuickDough achieves this speed by first translating compute kernels into data flow graphs (DFGs),
which are then statically scheduled onto the target overlay.
It then integrates the scheduling result with a pre-implemented bitstream of the overlay to produce
the final configuration bitstream with both the accelerator and its communication infrastructure.
When compared to a typical FPGA implementation flow that includes lengthy steps such as synthesis
and place-and-route, the run time of QuickDough is almost 2 orders of magnitude shorter when
compiling our benchmark programs.

Moreover, the softness of the overlay allows application-specific customization for better
performance-overhead trade-off and the regularity of the overlay makes the customization much
easier. According to the experiments, it takes around 10 minutes to 20 minutes to complete the
customization and the customized accelerators achieve up to 9X speedup over the execution on a hard
ARM processor. Although the overlay pre-implementation which must eventually rely
on conventional hardware design tools and the customization over a large design space are
relatively slow, they are required only once per application throughout the design iterations and
can therefore be amortized as the number of compile-debug-edit cycle increases.

After exploring related work in next section, the QuickDough design methodology will be elaborated
in \secref{sec:framework}. The design and implementation of the overlay will then be presented in
\secref{sec:scgraimplement}, the compilation framework will be illustrated in
\secref{sec:scgracompile} and the customization will be detailed in \secref{sec:customization}. In
\secref{sec:experiments}, experimental results will be shown.
Finally, we will discuss the benefit and limitations of current implementation in
\secref{sec:discussion} and conclude in \secref{sec:conclusions}.
