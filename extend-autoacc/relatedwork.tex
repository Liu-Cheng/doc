\section{Related Work} \label{sec:relatedwork}
Despite the performance and power advantages, the design 
productivity of developing FPGA applications remains low 
due to the lengthy compilation and complex application-specific 
customization. And it has become the major obstacle 
that hinders the wide adoption of FPGAs as commodity computing devices. 
The community from both the industry and academia have developed 
many different methods from diverse angles to tackle the problem. 
These methods can be roughly classified into three categories. 
The first category mainly focuses on improving the low-level 
implementation tools. A number of approaches such as making 
quality/runtime trade-offs \cite{mulpuri2001runtime}, parallel 
compilation \cite{moctar2014parallel,goeders2011deterministic,altera-pc,xilinx-pc} 
and using hard-macro techniques \cite{lavin2013improving,korf2011automatic} have been explored from this angle. The second 
category mainly centers the HLS design flow while the third one 
primarily relies on the overlay concept. They later two categories 
will be detailed in the following sections.

\subsection{High-Level Synthesis} 
With many years of continuous endeavor, a number of tools have emerged as 
mature solutions for HLS \cite{VivadoHLS,Legup,zhang2008autopilot}. They typically 
allow designers to express hardware designs using high-level  
description languages such as C, C++ etc. and also enable evaluation of different 
design choices using pragmas or directives. Indeed, they significantly improve 
the design productivity compared to the conventional hardware design flow using 
hardware description languages. However, when considering the overall design 
productivity of developing hybrid software-gateware applications, HLS is 
only addressing part of the problem, as the lengthy low-level compilation 
including synthesis, mapping, placing and routing remains a bottleneck for 
an application designer \cite{ROB2014,capalija2014tile}.

Customizing the generated hardware specifically to an user 
application is also time-consuming for designers and thus critical to the design 
productivity. A number of algorithms such as generic algorithms 
relied on local-search techniques \cite{schafer2009adaptive,sengupta1997genetic}, 
learning-based methods \cite{onlinecustomization,carrion2012machine}, 
divide and conquer algorithm \cite{DCcustomization} 
and a calibration free algorithm \cite{RCcustomization} etc. have been developed 
to perform the DSE on top of HLS tools. The algorithms can efficiently help automate the 
customization or DSE process. However, the algorithms must rely on HLS tools 
to estimate the implementation information such as implementation frequency, 
overhead or power for the corresponding customization. While the hardware generated 
can be irregular and may vary dramatically, thus the accuracy of the estimation 
especially on implementation frequency and power can be rather limited, which may
fail to optimize an HW/SW co-design problem.  

\subsection{Overlay Architectures}
Overlay architecture which is a virtual intermediate architecture overlaid on 
top of off-the-shelf FPGA is increasingly applied as a way to address the 
productivity challenge. 

Various overlays with diverse configuration granularities and flexibility 
ranging from virtual FPGAs \cite{Grant2011Malibu,ZUMA2012}, 
array-of-FUs \cite{mesh-FUs,ferreira2011fpga}, soft 
CGRA \cite{kissler2006dynamically,scgra}, soft GPU \cite{Guppy2012GPU-Like}, 
vector processors\cite{Yiannacouras2009FPS,MXP2013} to 
configurable processors or multi-core processors 
\cite{unnikrishnan2009application,MARC2010,Yiannacouras2007Exploration,Capalija2009coarse-grain,OCTAVO2012,iDEA2012} 
have been developed over the years. SCGRA overlay provides unique 
advantages on compromising hardware implementation 
and performance for compute intensive nested loops as demonstrated 
by numerous ASIC CGRAs \cite{tessier2001reconfigurable,compton2002reconfigurable}.
Most importantly, it allows both rapid compilation by taking advantage of 
the overlays' tiling structure \cite{ROB2014} and efficient bitstream 
reuse within the design iterations of an application \cite{scgra}, 
thus it is particularly promising for high productivity nested loop acceleration.

Indeed, SCGRA overlays have many similarities in terms of array structure 
and scheduling algorithm with ASIC CGRAs. ASIC CGRAs emphasize 
more on configuration capability and limited customization is allowed due 
to the overhead constraints \cite{zhou2014application,miniskar2014retargetable} 
while SCGRA overlays allow more intensive architectural customization 
because of the FPGA's inherent programmability. Moreover, hardware resources such as 
DSP blocks and RAM blocks available on FPGAs are discrete, which results in different 
design constraints for SCGRA overlay customization. 

The authors in \cite{colinheart} developed an SCGRA topology customization method using 
genetic algorithm and showed the potential benefits of the SCGRA overlay customization, 
but the rest of the system design parameters were not covered.
In order to achieve both high design productivity and high performance with low overhead,  
a complete nested loop acceleration framework targeting CPU-FPGA system 
is developed in this work. It supports intensive application-specific
customization including the overlay architectural customization, 
the compilation customization and communication interface customization 
for various design goals. When the customized design parameters are determined, 
corresponding hardware accelerator and software can be compiled to the target 
CPU-FPGA system rapidly eventually providing a push-button solution for a nested loop 
acceleration. 
