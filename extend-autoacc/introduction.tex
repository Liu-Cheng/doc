\section{Introduction}
Offloading compute intensive nested loops to FPGA accelerators has 
been demonstrated as an effective way of performance 
acceleration across various application domains\cite{Chung2010}. 
However, the design productivity of developing such accelerators 
remains relatively low and it has become a major obstacle that 
hinders the wide adoption of FPGAs as compute engines. Although the use of 
high level synthesis (HLS) tools which allow the application designers to 
focus on high level functionality instead of low-level implementation details alleviates 
this shortcoming \cite{cong2011high}, the lengthy low-level FPGA implementation process 
greatly limits the number of compile-debug-edit cycles per day and dramatically 
affects the overall design productivity. 

To approach the above design productivity problem, 
researchers have recently turned to the use of virtual FPGA overlay 
architectures \cite{Grant2011Malibu,ZUMA2012,mesh-FUs,
ferreira2011fpga, kissler2006dynamically,scgra}. When combined properly to 
high level compilation tools, the overlay architecture based design methods 
are able to produce high-performance accelerators at near software 
compilation speed, but at the cost of hardware overhead, power and even performance.
By customizing the architectures of these \emph{virtual} 
overlays for a target user design, in theory, it is 
possible to significantly improve the performance-energy of the 
resulting accelerator. In practice, however, navigating through a 
labyrinth of architectural and compilation parameters to fine-tune 
an accelerator's performance-energy is a slow and non-trivial process. 
To require a user to manually explore such vast design space is going 
to counteract the productivity benefit of the utilizing overlay 
in the first place.

To obtain both high design productivity and advantages of 
application-specific customization, we have developed a 
soft CGRA (SCGRA) overlay based nested loop acceleration design 
framework. This framework targets a hybrid CPU-FPGA computing system where 
nested loop compute kernels expressed in high-level languages are compiled and 
executed on the SCGRA overlay built on top of FPGAs while the rest of the 
user application remains running on the host CPU. Given high-level design 
goals and design constraints, the framework automatically explores the 
design space and customizes architectural parameters specifically to the 
user application. In addition, the framework also exploits loop unrolling 
and hardware-software communication strategies in combination 
with buffer sizing and partition as performance enhancing techniques.
Once the design goals and constraints are fulfilled, the 
corresponding hardware accelerator and communication interface 
are generated and both the hardware accelerator and software 
are compiled to the hybrid CPU-FPGA system.  

As demonstrated in previous work, both the compilation from nested 
loops to the SCGRA overlay \cite{scgra} and the SCGRA overlay 
implementation \cite{ROB2014} are fast. Meanwhile, the SCGRA 
overlay is highly pipelined and has quite regular tiling structure, 
which makes the hardware overhead, power consumption and even implementation frequency 
highly predictable. Therefore, a multitude of design metrics such as performance 
and energy consumption can be accurately estimated using analytical models when the 
overlay scheduling result is available. And the nested loop specific acceleration problem can be 
reduced to a sub design space exploration centering an NP-complete SCGRA scheduling and 
a following customization with all the potential configurations well estimated. 
While the overlay scheduling depends on much less design parameters, the overall 
customization can be dramatically simplified. Accordingly, the overall design 
framework achieves both rapid compilation and fast application specific 
customization and ensures high design productivity and high performance of the resulting
accelerators at the same time.  

We performed a series of experiments to evaluate the efficiency 
and quality of the proposed design framework using a real-world 
benchmark. Compared to an exhaustive search, the proposed 
customization achieves similar results while reducing its 
runtime by 2 orders of magnitude on average. When compared to 
HLS implementations with moderate manual optimizations that can 
reasonably be expected from a novice user,  
the customized accelerators produced using the proposed framework 
has demonstrated competitive performance as well. 

With that, we consider the main contribution of this work is in the following areas:
\begin{itemize}[nosep]
\item We have developed a rapid customization framework that 
    performs automatic design parameter tuning for SCGRA overlay based 
    nested loop acceleration on a hybrid CPU-FPGA computing system. 
    The result is comparable to an exhaustive design space search while 
    it runs at a fraction of time.
\item We have developed a parametric regular SCGRA overlay template. It can be used 
    to generate FPGA accelerators with predictable implementation 
    frequency, hardware overhead and power consumption, which is essential to 
    both the rapid compilation and customization.
\item We have developed a hierarchy on-chip buffer. It allows flexible 
    buffer partition and makes good use of the efficient lock-step computation 
    of the SCGRA overlay.
\end{itemize}

In \secref{sec:relatedwork}, related work is briefly introduced. 
The overall automatic nested loop acceleration framework is illustrated 
in \secref{sec:acc-framework}. Then SCGRA overlay based FPGA accelerator 
is illustrated in \secref{sec:scgra} and the application-specific customization 
method is further detailed in \secref{sec:customization-method}. 
Experimental results are presented in \secref{sec:result} and limitations are 
discussed in \secref{sec:limitations}. Finally, the paper is 
concluded in \secref{sec:conclusion}.


