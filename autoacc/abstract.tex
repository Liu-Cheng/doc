Offloading compute intensive nested loops to an FPGA accelerator 
has been demonstrated by numerous researchers as an effective solution 
to meet the performance requirement across many application domains. 
However, the relatively low design productivity caused by compiling 
high level loops to FPGA accelerators and customizing 
the resulting FPGA accelerators to specific applications has become the major 
obstacle to the wide adoption of FPGAs as computing devices. 

To address the productivity and performance problems, an automatic 
nested loop acceleration framework on a hybrid CPU-FPGA system is 
presented. It adopts a regular soft coarse-grained 
reconfigurable array (SCGRA) overlay as the backbone of the FPGA 
accelerator and can be implemented much faster than a conventional 
high level synthesis (HLS) design flow. Most importantly, it 
automatically customizes not only the overlay architectural 
design parameters but also high level compilation options and 
communication interfaces between the accelerator and host CPU, 
which completely shields the hardware design details  
and makes it accessible to high level application designers. 
In addition, by taking advantage of the regularity of the 
SCGRA overlay, a straightforward yet effective customization 
method is developed. According to the experiments, the proposed 
customization method can achieve similar Pareto-optimal 
curve to that acquired by using an exhaustive search while 
its runtime is around two orders of magnitude shorter.
Experiments also show that the customized accelerators 
achieve competitive performance compared to the 
implementations using off-the-shelf HLS design tools. 

