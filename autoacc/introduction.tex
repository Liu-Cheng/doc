\section{Introduction}
Nested loops with abundant parallelism are usually 
critical to the performance of an application and offloading them to 
FPGA accelerators has been demonstrated to be an effective 
solution to meet the performance requirements across many 
application domains \cite{Chung2010}. However, the low productivity 
of developing FPGA-based compute applications has become a major obstacle 
that hinders the widespread of FPGA computing \cite{cong2011high}.

High level synthesis (HLS) tools that allow the designers to express hardware 
designs using familiar high-level description languages greatly alleviate the 
design productivity challenge, but lengthy back-end hardware implementation 
which dramatically limits the number of compile-debug-edit cycles per day is 
still inevitable. Coarse-grained reconfigurable array (CGRA) overlay 
which consists of an array of connected processing elements (PEs) overlaid 
on top of the physical FPGA fabric is emerging as a promising technique to tackle 
the design productivity challenge, as it allows both high-level design entry and 
rapid back-end implementation through a hard-macro based compilation 
\cite{ROB2014} as well as bitstream reuse \cite{scgra-orig}. Nevertheless, 
achieving the increasingly stringent energy and performance requirements 
still requires intensive system customization specifically to a user application, 
while navigating through a labyrinth of architectural design parameters and 
compilation parameters for the customization is extremely slow and contradicts 
with the intension of an high-productivity computing. 

To address both the design productivity and performance challenge, 
we developed a soft CGRA (SCGRA) overlay based 
automatic nested loop acceleration framework targeting a hybrid CPU-FPGA 
system. In this framework, nested loops expressed in high-level languages 
will be compiled and executed on the SCGRA overlay while the rest of the 
user application remains running on the host CPU.
Given an user application and the design constraints, the framework will
automatically explore the design space and tune the design parameters 
specifically to the user application. When the optimal design parameters 
fulfilling the design goals and constraints are acquired, 
corresponding hardware accelerator and communication interfaces 
will be generated. Finally, both hardware accelerator
and software are compiled to the hybrid CPU-FPGA system.  

Since the SCGRA overlay has quite regular structure, its implementation 
scales well and it is relatively straightforward to predict its 
hardware overhead and power consumption with simple 
analytical models. Furthermore, given the SCGRA overlay scheduling 
result, a lot of design metrics such as the nested loop performance, 
energy consumption and energy delay product (EDP) can also be 
estimated easily. While SCGRA overlay scheduling merely 
depends on the overlay size and loop unrolling factor, the overall 
nested loop acceleration customization problem can be reduced to a much simpler 
sub design space exploration (DSE) centering the overlay scheduling. 
Basically, the overall customization can be divided into a small sub DSE step 
and an analytical customization step, and both steps are effective and fast.

After the customization, both the hardware accelerator and communication 
interfaces can be generated. As the accelerator utilizes the regular tiling 
SCGRA overlay as the backbone, it can be implemented rapidly using macro based 
compilation techniques as presented in \cite{ROB2014}. Moreover, when minor 
modification is applied to the high level nested loop description during 
the design iterations, the FPGA accelerator implementation can be reused 
and the compilation process can further be reduced to seconds as demonstrated in 
\cite{scgra-orig}. Therefore, the rapid back-end compilation together with the fast 
the application specific customization will guarantee the design productivity 
of the overall design automation framework. 

We performed a series of experiments to evaluate the efficiency 
and quality of the proposed design framework using a real-world 
benchmark. Compared to an exhaustive search, the proposed 
customization achieves similar results while reducing its 
runtime by 2 orders of magnitude on average. When compared to 
HLS implementations with moderate manual optimization, 
the customized accelerators produced using the proposed framework 
exhibit competitive performance as well. 

In \secref{sec:relatedwork}, related work is briefly introduced. 
The automatic SCGRA overlay based nested loop acceleration 
framework is illustrated in \secref{sec:acc-framework}. Then the 
SCGRA overlay customization are further detailed in 
\secref{sec:customization-framework}. Finally, experimental results are 
presented in \secref{sec:result} and the paper is concluded in \secref{sec:conclusion}.

