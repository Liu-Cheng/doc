\section{Related Work} \label{sec:relatedwork}
Overlay architecture which is a virtual intermediate architecture overlaid on 
top of off-the-shelf FPGA is increasingly applied as a way to address the 
productivity challenge. 

Various overlays with diverse configuration granularities and flexibility 
ranging from virtual FPGAs \cite{Grant2011Malibu, ZUMA2012, Coole2010Intermediate}, 
array-of-FUs \cite{mesh-FUs,ferreira2011fpga,dspoverlay}, soft 
CGRA \cite{kissler2006dynamically, scgra}, soft GPU \cite{Guppy2012GPU-Like}, 
vector processors\cite{Yiannacouras2009FPS, MXP2013} to 
configurable processors or multi-core processors \cite{unnikrishnan2009application, 
MARC2010, Yiannacouras2007Exploration, Capalija2009coarse-grain, OCTAVO2012, iDEA2012} 
have been developed over the years. SCGRA overlay provides unique 
advantages on compromising hardware implementation 
and performance for compute intensive nested loops as demonstrated 
by numerous ASIC CGRAs \cite{tessier2001reconfigurable, compton2002reconfigurable}.
Most importantly, it allows both rapid compilation by taking advantage of 
the overlays' tiling structure \cite{ROB2015} and efficient bitstream 
reuse within the design iterations of an application \cite{scgra}, 
thus it is particularly promising for high productivity nested loop acceleration.

In addition, customizing the CGRA specifically for an application or a domain of application
provides promising performance improvement while saving the hardware resource at the same time as
demonstrated in CGRA work targeting ASIC design \cite{totem, zhou2014application, miniskar2014retargetable}. 
While CGRA customization on ASIC is relatively limited due to the tape-out cost, CGRA overlays
allow more intensive architectural customization providing just enough hardware
to the target application or application domains because of the FPGA's inherent programmability. 
In \cite{adjustable2015}, Coole and Stitt proposed to provide the overlay with limited flexibility
instead of full configurability specifically to a group of design. With this customization, the area
overhead was reduced significantly. The authors in \cite{Lin:2012:EDC:2460216.2460227} developed an SCGRA topology
customization method using genetic algorithm and showed the potential benefits of the SCGRA overlay
customization. Nevertheless, the rest of the system design parameters were not covered. In \cite{BondhugulaRS07},
the authors formalized the loop acceleration on a regular processing array overlay on FPGA. They
focused on the hardware resource constrain, IO bandwidth constrain and the loop parallelism partition while
processing architectural design parameters were not included.
In order to achieve both high design productivity and high performance with low overhead,  
a complete nested loop acceleration framework targeting CPU-FPGA system 
is developed in this work. It supports intensive application-specific
customization including the overlay architectural customization, 
the compilation customization and communication interface customization 
for optimized performance.  

