%%%%%%%%%%%%%%%%%%%%%%%%%%%%%%%%%%%%%%%%%
% Beamer Presentation
% LaTeX Template
% Version 1.0 (10/11/12)
%
% This template has been downloaded from:
% http://www.LaTeXTemplates.com
%
% License:
% CC BY-NC-SA 3.0 (http://creativecommons.org/licenses/by-nc-sa/3.0/)
%
%%%%%%%%%%%%%%%%%%%%%%%%%%%%%%%%%%%%%%%%%

%----------------------------------------------------------------------------------------
%	PACKAGES AND THEMES
%----------------------------------------------------------------------------------------

\documentclass{beamer}

\mode<presentation> {

% The Beamer class comes with a number of default slide themes
% which change the colors and layouts of slides. Below this is a list
% of all the themes, uncomment each in turn to see what they look like.

%\usetheme{default}
%\usetheme{AnnArbor}
%\usetheme{Antibes}
%\usetheme{Bergen}
\usetheme{Berkeley}
%\usetheme{Berlin}
%\usetheme{Boadilla}
%\usetheme{CambridgeUS}
%\usetheme{Copenhagen}
%\usetheme{Darmstadt}
%\usetheme{Dresden}
%\usetheme{Frankfurt}
%\usetheme{Goettingen}
%\usetheme{Hannover}
%\usetheme{Ilmenau}
%\usetheme{JuanLesPins}
%\usetheme{Luebeck}
%\usetheme{Madrid}
%\usetheme{Malmoe}
%\usetheme{Marburg}
%\usetheme{Montpellier}
%\usetheme{PaloAlto}
%\usetheme{Pittsburgh}
%\usetheme{Rochester}
%\usetheme{Singapore}
%\usetheme{Szeged}
%\usetheme{Warsaw}

% As well as themes, the Beamer class has a number of color themes
% for any slide theme. Uncomment each of these in turn to see how it
% changes the colors of your current slide theme.

%\usecolortheme{albatross}
%\usecolortheme{beaver}
%\usecolortheme{beetle}
%\usecolortheme{crane}
%\usecolortheme{dolphin}
%\usecolortheme{dove}
%\usecolortheme{fly}
%\usecolortheme{lily}
%\usecolortheme{orchid}
%\usecolortheme{rose}
%\usecolortheme{seagull}
%\usecolortheme{seahorse}
%\usecolortheme{whale}
%\usecolortheme{wolverine}

%\setbeamertemplate{footline} % To remove the footer line in all slides uncomment this line
%\setbeamertemplate{footline}[page number] % To replace the footer line in all slides with a simple slide count uncomment this line

%\setbeamertemplate{navigation symbols}{} % To remove the navigation symbols from the bottom of all slides uncomment this line
}

\usepackage{graphicx} % Allows including images
\usepackage{booktabs} % Allows the use of \toprule, \midrule and \bottomrule in tables

%----------------------------------------------------------------------------------------
%	TITLE PAGE
%----------------------------------------------------------------------------------------

\title[]{Loop Acceleration For Tightly-Coupled CPU+FPGA System} 
\author[]{
    Cheng Liu 
    \\Supervisor: Dr. Hayden Kwok-Hay So 
    \\Co-supervisor: Dr. Ngai Wong}
\institute {
    Department of Electrical and Electronic Engineering 
    \\The University of Hong Kong
\medskip
}
\date{\today} % Date, can be changed to a custom date

\begin{document}

\begin{frame}
\titlepage % Print the title page as the first slide
\end{frame}

%----------------------------------------------------------------------------------------
%	PRESENTATION SLIDES
%----------------------------------------------------------------------------------------

%------------------------------------------------
\section{Background} 
%------------------------------------------------
\begin{frame}[t]
\frametitle{Loop and performance acceleration}
\begin{itemize}
\item Loop will be the performance bottleneck in many programs executed on a sequential computation architecture.
\item Loop has ample data parallelism and it is one of the most preferable optimization tagrets.
\item Loop provides a great opportunity for FPGA computation.
\end{itemize}
\end{frame}

%------------------------------------------------
\begin{frame}[t]
\frametitle{FPGA design over CPU based software design}

\textbf{Advantages}
\begin{itemize}
\item Inherent parallel computation capability
\item Energy efficiency (application specific design, ...)
\item Low latency
\item ...
\end{itemize}

\textbf{Disadvantages}
\begin{itemize}
\item High barrier-to-entry (Hardware engineering skills, ...)
\item Low design productivity (Low level abstraction, Long compilation time, ...)
\item Difficult to handle complex softwares like OS
\item ...
\end{itemize}

\end{frame}

%------------------------------------------------
\begin{frame}[t]
\frametitle{Hybrid architecture: CPU+FPGA accelerator}
CPU+FPGA Contex
\end{frame}

%------------------------------------------------
\begin{frame}[t]
\frametitle{Challenges of the CPU+FPGA accelerator}
\begin{itemize}
\item High barrier-to-entry (Hardware engineering skills, ...)
\item Low design productivity (Low level abstraction, Long compilation time, ...)
\item Complex design space exploration
\item ...
\end{itemize}
\end{frame}

%------------------------------------------------
\section{Related work}
%------------------------------------------------
\begin{frame}
\frametitle{Contemporary FPGA accelerator design methods}
\textbf{Classification through design entry}
\begin{itemize}
\item Register transfer level (RTL) design flow
\item High level synthesis based design flow
\end{itemize}

\textbf{Classification through compilation}
\begin{itemize}
\item Compile to HDL and implementation on FPGA
\item Comple to virtual overlay, implementation on overlay and implementation on FPGA
\end{itemize}

\end{frame}

%------------------------------------------------
\begin{frame}

\frametitle{Comparison of different design methods}
Performance and productivity.

\end{frame}

%------------------------------------------------
\begin{frame}

\frametitle{Limitation of previous SCGRA work}
\begin{itemize}
\item Ignore the loop structure
\item Ignore the communication limitation
\end{itemize}

\end{frame}

%------------------------------------------------
\section{Research scheme}
\begin{frame}

\frametitle{System overview}
SCGRA based CPU+FPGA accelerator.

\end{frame}

%------------------------------------------------
\begin{frame}
\frametitle{Design space overview}
Complex design space

\end{frame}

%------------------------------------------------
\begin{frame}

\frametitle{Research focuses}
\begin{itemize}
\item Loop unrolling
\item Operation scheduling
\item Communication interface
\end{itemize}

\end{frame}

%------------------------------------------------
\begin{frame}

\frametitle{Automatic loop optimization}
\begin{itemize}
\item Loop unrolling analysis
\end{itemize}

\end{frame}

%------------------------------------------------
\begin{frame}

\frametitle{Operation scheduling}
\begin{itemize}
\item IO aware
\item On chip memory aware
\item memory footprint aware
\end{itemize}

\end{frame}

%------------------------------------------------
\begin{frame}

\frametitle{Communication interface}
\begin{itemize}
\item Data prefetching
\item On chip buffer synthesis
\end{itemize}

\end{frame}

%\begin{figure}
%\includegraphics[width=0.8\linewidth]{test}
%\end{figure}

%------------------------------------------------
\section{Current progress}
\begin{frame}

\frametitle{Design productivity oriented design flow}
Design flow developed to improvde design productivity.

\end{frame}

%------------------------------------------------
\begin{frame}

\frametitle{Preliminary loop unrolling analysis}
%\footnotesize{
%\begin{thebibliography}{99} % Beamer does not support BibTeX so references must be inserted manually as below
%\bibitem[Smith, 2012]{p1} John Smith (2012)
%\newblock Title of the publication
%\newblock \emph{Journal Name} 12(3), 45 -- 678.
%\end{thebibliography}
%}
\end{frame}

%------------------------------------------------
\begin{frame}

\frametitle{HW/SW communication on Zedboard}
HW/SW communication on Zedboard

\end{frame}

%------------------------------------------------
\section{Conclusion}
\begin{frame}
\Huge{\centerline{The End}}
\end{frame}

%------------------------------------------------

\end{document} 
