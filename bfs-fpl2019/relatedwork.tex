\section{Related work} \label{sec:relatedwork}
The growing importance of efficient BFS traverse on large graphs 
have attracted attentions of many researchers. Different BFS 
optimization algorithms and accelerators have been proposed in 
literature. In this section, we will particularly focus on the 
FPGA based BFS acceleration. 

BFS is memory bound, so many BFS accelerator optimizations 
target on alleviating the memory bandwidth constrain. The authors in 
\cite{zhang2017boosting, khoram2018accelerating} 
took advantage of Hybrid Memory Cube (HMC) with much higher memory 
bandwidth for BFS acceleration. The authors in \cite{attia2014cygraph} 
proposed to change the compressed sparse row (CSR) format slightly to 
improve the memory access efficiency. The authors in \cite{umuroglu2015hybrid} 
chose to sacrifice some redundant memory accesses for sequential memory 
access such that the overall memory bandwidth utilization improves.
The authors in \cite{attia2014cygraph, betkaoui2012reconfigurable} explored 
Convey HC-2 \cite{bakos2010high} with multiple memory instances for BFS acceleration. 

On top of the BFS accelerators, many general graph processing acceleration frameworks 
have been proposed \cite{engelhardt2016gravf, jun2018grafboost, yao2018efficient, 
oguntebi2016graphops, Dai2017foregraph, dai2016fpgp, nina2018performance, engelhardt2017towards}. The authors \cite{kapre2015custom, 
wang2010message} developed customized soft processors on FPGAs for graph processing and further 
presented a distributed software solution on a number of connected FPGAs.
One of the main goals of these works is to achieve high graph processing performance and make the 
FPGA based accelerator design easier. Along with this goal, we argue that using OpenCL to build highly 
optimized graph processing accelerator also provides high performance and improves the 
accelerator design productivity. The authors in \cite{gautier2016spector, chen_fpl2019} 
presented OpenCL based vertex-centric and edge-centric BFS accelerators 
respectively. They have demonstrated the portability of OpenCL based BFS 
accelerators on FPGAs, which is confirmed by our study,
%and the portability also enables us to reproduce their 
%designs on Intel Harp-v2 in a short time. 
though their performance remains much lower than the RTL designs. 
Still, with a number of optimizations, our study 
shows that OpenCL based BFS accelerator with comprehensive optimizations will be a promising approach to 
gain both high performance and high-level software features.
