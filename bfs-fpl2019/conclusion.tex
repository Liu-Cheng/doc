\section{Conclusion} \label{sec:conclusion}
%\subsection{Limitations}
%In this work, we mainly explore the OpenCL based BFS accelerator 
%design on FPGAs, though Harp-v2 is a heterogeneous CPU-FPGA computing platform. 
%We may further explore co-design on the heterogeneous architecture.

In this work, we assume that the bitmap can be fully buffered on 
the FPGA on-chip RAM blocks. 
Basically, we limit the number of vertices in the graph that 
can be processed using this work. While the latest FPGAs has over 
500Mb on-chip memory \cite{xilinxFPGA}, we are supposed to handle graphs with hundreds 
of millions of vertices on a single FPGA card. Note that, the number of edge does 
not have this constraint. To handle graphs with more vertices, we need to further explore 
the graph partitioning, while this work exhibits 
the efficiency of the BFS processing of a single graph partition.

%\subsection{Conclusion}
Handcrafted BFS accelerators with HDL usually suffer high portability and maintenance cost despite the relatively 
good performance. OpenCL based BFS accelerator can greatly alleviate these problems, but it is 
challenging to achieve satisfactory performance due to the inherent irregular memory accesses. 
In this work, we propose a series of high-level optimization approaches to improve the irregular 
memory access efficiency in BFS. Compared to reference BFS implementations with a vertex-centric 
framework and an edge-centric framework, 
the proposed OBFS achieves 9.5X and 5.5X performance speedup on average respectively. 
When compared to the prior HDL based BFS accelerators, OBFS also achieves competitive 
performance.


