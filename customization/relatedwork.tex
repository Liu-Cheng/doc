\section{Related Work} \label{sec:relatedwork}
FPGA design productivity has become a major obstacle that hinders the 
widespread adoption of FPGA as commodity computing devices. To address 
the problem, the community has developed a number of design tools that 
can compile high level applications all the way to FPGA 
\cite{zhang2008autopilot, Legup, VivadoHLS, ROCCC} and design 
frameworks that can provide automatic customization to achieve better 
design trade-off \cite{dataflowcustomization, 
mlcustomization, genericcustomization, DCcustomization, onlinecustomization, 
RCcustomization, colinheart}.

High level synthesis (HLS) is one of the most widely adopted methods to 
compile high level applications to FPGA circuits. Thus previous automatic 
customization frameworks were mostly developed on top of the HLS tools 
\cite{dataflowcustomization, mlcustomization, 
genericcustomization, DCcustomization, onlinecustomization}. 
They typically considered an HLS tool as a black box. Generic algorithms 
were adopted to perform the DSE and customization in
\cite{mlcustomization, genericcustomization, onlinecustomization}. 
Divide and conquer method was used in \cite{DCcustomization} and 
A calibration tree algorithm was used in \cite{RCcustomization}.  

Overlay which is an intermediate layer built on top of the 
off-the-shelf FPGA devices has been demonstrated to be an 
effective way to improve the FPGA design productivity\cite{scgra, ferreira2011fpga}. 
A number of different types of overlays \cite{Lebedev2010, kissler2006dynamically, 
    ferreira2011fpga, Grant2011Malibu, Guppy2012GPU-Like, unnikrishnan2009application, 
Yiannacouras2009FPS, Capalija2009coarse-grain} have been developed for different 
trade-offs between flexibility and overhead. SCGRA overlay is one of the most 
promising overlays for compute intensive application acceleration on FPGA 
\cite{scgra, colinheart}. The authors in \cite{colinheart} focused
on topology customization and showed the potential benefits of the SCGRA overlay 
customization. This work will provide a more intensive customization for nested loop
acceleration on a hybrid CPU-FPGA system. It will not only include the overlay 
fabric customization but also the compilation customization such as loop unrolling 
and communication customization between the FPGA and host CPU.

